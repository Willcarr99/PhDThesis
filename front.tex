%% ------------------------------ Abstract ---------------------------------- %%
\begin{abstract}

The asymptotic giant branch (AGB) phase of stellar evolution involves complex nucleosynthesis processes responsible for the production of many chemical elements observed in the universe. The signatures of active AGB nucleosynthesis have been discovered through the observation of both enhanced rubidium abundance in the atmospheres of massive O-rich AGB stars and abundance anticorrelations of light-element pairs in low-mass globular cluster stars. Massive AGB stars have even been considered a candidate nucleosynthesis host for the extreme potassium enrichment and magnesium depletion observed in the red giant stars of the globular cluster NGC 2419. Slow neutron capture during AGB thermal instabilities (the s-process) and hydrogen burning at the base of the convective hydrogen envelope (hot-bottom burning) have been shown to be sites of the nucleosynthesis for these abundance patterns. These sites are investigated in this thesis through nuclear reaction network calculations involving the reaction rates of all relevant nuclei. The discrepancy between observed and predicted rubidium abundance in massive AGB stars is investigated with a Monte Carlo reaction network. All reaction rates in the network are varied within their uncertainties to determine the rates most correlated with rubidium abundance. The discrepancy is determined to be unlikely a result of reaction rate uncertainties alone, but the $^{86}\mathrm{Rb}(n,\gamma)^{87}$Rb, $^{13}\mathrm{N}(n,p)^{13}\mathrm{C}$, and $^{14}\mathrm{N}(n,p)^{14}\mathrm{C}$ reaction rates are found to be the most in need of constraint. The Mg--K anticorrelation in the globular cluster NGC 2419 is investigated by constraining the key potassium-destroying reaction $^{39}\mathrm{K}(p,\gamma)^{40}\mathrm{Ca}$ using nuclear structure input from the $^{39}\mathrm{K}(^{3}\mathrm{He},d)^{40}\mathrm{Ca}$ proton-transfer reaction measurement at the Triangle Universities Nuclear Laboratory. This measurement has resolved a key $^{39}\mathrm{K}+p$ resonance, 154 keV, for the first time, among others, leading to a significant impact on the $(p,\gamma)$ reaction rate at temperatures $T \lesssim 110$ MK. Potassium production is found to decrease at these temperatures as a result, and the temperatures and densities of the hydrogen-burning candidate sites responsible for the Mg--K anticorrelation are constrained compared to the previous $^{39}\mathrm{K}(p,\gamma)^{40}\mathrm{Ca}$ rate.

%Asymptotic giant branch (AGB) stars are responsible for the production of about half of all heavy elements in the universe, as well as many of the anomalous abundance patterns observed in globular clusters. These observations are signatures of complex nucleosynthesis processes, ranging from slow neutron capture due to thermal instabilities (the s-process) to high-temperature hydrogen burning (hot-bottom burning). 

%Nuclear reaction network calculations can offer insight into these processes by determining which reaction rates are important and under which stellar conditions the observations are reproduced. Laboratory nuclear reaction experiments, such as transfer reactions, can then be used to constrain the important reaction rates. 

%In this thesis,

%updating the nuclear structure inputs that go into these network calculatons.

%In particular, observations of enhanced rubidium abundance

\end{abstract}


%% ---------------------------- Copyright page ------------------------------ %%
%% Comment the next line if you don't want the copyright page included.
\makecopyrightpage

%% -------------------------------- Title page ------------------------------ %%
\maketitlepage

%% -------------------------------- Dedication ------------------------------ %%
\begin{dedication}
\centering To Mom, Dad, Grandma, Lindsay, and Alex.
\end{dedication}

%% -------------------------------- Biography ------------------------------- %%
\begin{biography}
William Fox was born in Raleigh, N.C. in 1993 to his parents Daniel and Tammy Fox, who were both the inspiration for his academic pursuits. He attended Sanderson High School, where he was first exposed to physics in his senior year, unknowingly shaping his entire career. As an undergraduate at the University of North Carolina at Wilmington, he developed a deep connection with both physics and mathematics, studying general relativity and astrophysics. He conducted research in gravitational wave theory in the midst of the first gravitational wave detection by LIGO, inspiring him to pursue graduate studies. As a graduate student at North Carolina State University, he discovered a newfound interest in experimental nuclear physics and has been drawn to its connection with astrophysics ever since.
\end{biography}

%% ----------------------------- Acknowledgements --------------------------- %%
\begin{acknowledgements}
% Alex
First, I would like to thank my wonderful wife, Alex. She has been a relentless encouragement through my entire graduate studies, and I am eternally grateful for her support. Being an expert in literature, she has provided an artistic perspective in my life not often encountered in this field. I cannot thank her enough for her countless hours listening to me ramble on about the latest figure or paragraph in my thesis. She always enthusiastically discussed the material with me and helped me perfect it, regardless of whether I provided context. She has been an inspiration to me through her own career, and I look forward to supporting her as she has certainly done for me.

% Richard
Richard Longland has been an exceptional advisor, mentor, and friend over the last five years. His advising style has coupled well with my own researching style, giving me the freedom to create my own path toward solutions. He truly brings out the creativity in his students by allowing them to develop their own analysis pipelines, distinguishing themselves from each other. I am grateful for Richard taking me in, even with minimal experimental background. Through his guidance, I now have the skills and mindset of an experimental physicist, something I never thought I would achieve. For this and much more, I could not be more grateful.

% Christian
Working closely with Christian Iliadis has been one of the highlights of my time as a graduate student. He has been an outstanding mentor and friend since my early days at TUNL. Even brief encounters with Christian are always memorable due to his personable nature. He has made my time at TUNL not only academically rewarding, but also enjoyable. I have referenced his nuclear astrophysics book and course notes countless times in my research, and this thesis would not exist without him.

% Committee Members
I am very grateful for the rest of my committee members, Carla Fr\"{o}hlich, Matthew Green, and John Classen, who have taken the time to read my thesis and provided constructive feedback during my defense. From our discussions, many positive changes have been implemented into the thesis.

% Dr. Herman
I would have pursued a different career if not for Russell Herman. As my undergraduate advisor and mentor, he inspired me to be a physicist. I look back on those days learning general relativity and gravitational wave theory from him with complete awe at how he could make such complicated subjects so simple. He is one of the greatest physics and mathematics educators I have ever known, and I owe him lifetimes of gratitude for getting me to where I am today.

% Caleb, Federico, and Kiana
Caleb Marshall, Federico Portillo Chaves, and Kiana Setoodehnia comprised Richard's research group when I joined, and they were each essential to my assimilation into the group. Caleb taught me everything I needed to know about the quirks of the tandem accelerator and the Direct Extraction Negative Ion Source (DENIS) at TUNL. Federico taught me how to make evaporated targets, how to perform maintenance on the focal plane detector, and how to use \texttt{FRESCO} for distorted-wave Born approximation (DWBA) calculations. I only briefly got to know Kiana before she left the group, but she has helped me immensely whether she knows it or not. Her meticulous notes on the focal plane detector, the Enge Split-Pole Spectrograph, and the Enge beamline were vital to the success of the $^{39}\mathrm{K}(^{3}\mathrm{He},d)^{40}\mathrm{Ca}$ experiment presented in this thesis, among many other experiments performed at TUNL.

% TUNL Nuclear Astro Group
The Nuclear Astrophysics group at TUNL has provided me with a sense of community unlike any other. Its informal nature has helped ease the everyday stress of being a graduate student and allowed me to create many new friendships among colleagues. Sharing my research with students, postdocs, and professors via countless presentations and research updates has also greatly enhanced my communication skills. I am very grateful for this close group of people, and I expect to remain close to each of them throughout my career.

% TUNL Staff
Finally, I would like thank the TUNL staff, without whom the $^{39}\mathrm{K}(^{3}\mathrm{He},d)^{40}\mathrm{Ca}$ experiment presented in this thesis would not have been possible. In particular, I would like to thank Chris Westerfeldt, John Dunham, Bret Carlin, and Tom Calisto for each working closely with me and others in Richard's group on various technical concerns in the lab. Chris has worked tirelessly to maintain our cryopumps, allowing us to achieve excellent vacuum in the Enge beamline, target chamber, and Enge magnet. He has also been crucial in maintaining the evaporation chamber in the physics department at Duke, used for the evaparated potassium iodine targets in this thesis. John has been the ``go-to'' expert for the tandem accelerator and DENIS at TUNL, and has trained our entire group on their operation and maintenance. Bret has been the expert on the electronic components in our experimental setup at the Enge, as well as for the LabView graphical user interface used to operate the tandem beamline components. Finally, Tom has used his engineering expertise to tackle countless projects and issues with our experimental setup at the Enge for us.

\end{acknowledgements}

\thesistableofcontents

\thesislistoftables

\thesislistoffigures
