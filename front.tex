%% ------------------------------ Abstract ---------------------------------- %%
\begin{abstract}

The asymptotic giant branch (AGB) phase of stellar evolution involves complex nucleosynthesis processes responsible for the production of many chemical elements observed in the universe. The signatures of active AGB nucleosynthesis have been discovered through the observation of both enhanced rubidium abundance in the atmospheres of massive O-rich AGB stars and abundance anticorrelations of light-element pairs in low-mass globular cluster stars. Massive AGB stars have even been considered a candidate nucleosynthesis host for the extreme potassium enrichment and magnesium depletion observed in the red giant stars of the globular cluster NGC 2419. Slow neutron capture during AGB thermal instabilities (the s-process) and hydrogen burning at the base of the convective hydrogen envelope (hot-bottom burning) have been shown to be sites of the nucleosynthesis for these abundance patterns. These sites are investigated in this thesis through nuclear reaction network calculations involving the reaction rates of all relevant nuclei. The discrepency between observed and predicted rubidium production in massive AGB stars is investigated with a Monte Carlo reaction network, varying all reaction rates within their uncertainties to determine the rates most correlated with rubidium abundance. The $^{86}\mathrm{Rb}(n,\gamma)^{87}$Rb reaction rate is found to be one of the most in need of constraint. The Mg--K anticorrelation in the globular cluster NGC 2419 is investigated by constraining the key potassium-destroying reaction $^{39}\mathrm{K}(p,\gamma)^{40}\mathrm{Ca}$ using nuclear structure input from the $^{39}\mathrm{K}(^{3}\mathrm{He},d)^{40}\mathrm{Ca}$ proton-transfer reaction measurement at the Triangle Universities Nuclear Laboratory. This measurement has resolved a key $^{39}\mathrm{K}+p$ resonance, 154 keV, for the first time, among others, leading to a significant impact on the $(p,\gamma)$ reaction rate at temperatures $T \lesssim 110$ MK. Potassium production is found to decrease at these temperatures as a result, and the temperatures and densities of the hydrogen-burning candidate sites responsible for the Mg--K anticorrelation are mildly constrained.

%Asymptotic giant branch (AGB) stars are responsible for the production of about half of all heavy elements in the universe, as well as many of the anomalous abundance patterns observed in globular clusters. These observations are signatures of complex nucleosynthesis processes, ranging from slow neutron capture due to thermal instabilities (the s-process) to high-temperature hydrogen burning (hot-bottom burning). 

%Nuclear reaction network calculations can offer insight into these processes by determining which reaction rates are important and under which stellar conditions the observations are reproduced. Laboratory nuclear reaction experiments, such as transfer reactions, can then be used to constrain the important reaction rates. 

%In this thesis,

%updating the nuclear structure inputs that go into these network calculatons.

%In particular, observations of enhanced rubidium abundance

\end{abstract}


%% ---------------------------- Copyright page ------------------------------ %%
%% Comment the next line if you don't want the copyright page included.
\makecopyrightpage

%% -------------------------------- Title page ------------------------------ %%
\maketitlepage

%% -------------------------------- Dedication ------------------------------ %%
%\begin{dedication}
% \centering To my parents.
%\end{dedication}

%% -------------------------------- Biography ------------------------------- %%
\begin{biography}
The author was born in a small town \ldots
\end{biography}

%% ----------------------------- Acknowledgements --------------------------- %%
%\begin{acknowledgements}
%Thanks to family, advisor, committee, office friends, etc. ...
%\end{acknowledgements}


\thesistableofcontents

\thesislistoftables

\thesislistoffigures
