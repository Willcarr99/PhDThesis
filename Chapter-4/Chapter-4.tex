\chapter{Experiments at the Triangle Universities Nuclear Laboratory} 
\label{ch:exp}

% The beam was tuned using steering magnets and quadrupole focusing magnets positioned along the beamline, with the beam current being read from various Faraday cup beam stops. The beamline was under high vacuum ($\sim 10^{-6}-10^{-7}$ torr). Upon entering the target chamber, the beam made contact with the target, composed of natural potassium iodide (KI).

% The $d$ and $^{3}$He particles that were ejected at the angle $\theta_{\mathrm{lab}}$ passed through the magnetic field $B$ of the Enge Split-Pole Spectrograph, where they were then focused at a point along the focal plane. The focal-plane detector was preemptively moved via dual motors to align itself with the focal plane. Meanwhile, the $^{39}\mathrm{K}(^{3}\mathrm{He},^{3}\mathrm{He})^{39}\mathrm{K}$ reaction was similtanesouly measured with a silicon monitor detector inside the target chamber positioned at a constant $45^{\circ}$ from the beamline to monitor potential target degradation and other target properties.

% Carbon foils were purchased from the Arizona Carbon Foil Co., Inc. The carbon foils originally came arc-evaporated onto glass substrates with a parting agent. They were cut into pieces and allowed to float in warm water. Each piece attached to an aluminum target frame by dipping the frame underwater and raising it up from underneath the carbon foil piece. The surface tension from the water allowed for adherence of the foil onto the frame, and the resulting composite was allowed to dry overnight.

\section{Introduction}

This chapter details the equipment and experimental considerations that went into the successful $^{39}\mathrm{K}(^{3}\mathrm{He},d)^{40}\mathrm{Ca}$ transfer reaction experiment presented in Chapter \ref{ch:GC}, performed at the Triangle Universities Nuclear Laboratory (TUNL). Experimental preparation in terms of target making is presented, followed by an overview of the Tandem accelerator facility, where the present experiment was conducted. Finally, the Enge Split-Pole Spectrograph and the focal plane detector package are described in the context of the present experiment.

\section{Target Preparation}

\subsection{Carbon Backing}

\subsection{Evaporation}

\section{TUNL Tandem Accelerator and Beamline Components}

\section{Enge Split-Pole Spectrograph}

\section{Focal Plane Detector}
