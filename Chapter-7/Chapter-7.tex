\chapter{Summary and Conclusions}
\label{chap:conc}

% Massive AGB Stars connecting Chs 3 and 6. Digital DAQ to enhance future proton transfer experiments

%The asymptotic giant branch (AGB) phase of stellar evolution involves complex nucleosynthesis processes responsible for the production of many chemical elements observed in the universe. The signatures of active AGB nucleosynthesis have been discovered through the observation of both enhanced rubidium abundance in the atmospheres of massive O-rich AGB stars and abundance anticorrelations of light-element pairs in low-mass globular cluster stars. AGB stars have even been considered a candidate nucleosynthesis host for the extreme potassium enrichment and magnesium depletion observed in the red giant stars of the globular cluster NGC 2419. Slow neutron capture during AGB thermal pulses (s-process) and hydrogen burning at the base of the convective hydrogen envelope (hot-bottom burning) have been shown to be the sites of the nucleosynthesis for the above abundance patterns. These sites have been investigated in this thesis through nuclear reaction network models, involving the reaction rates of all relevant nuclei and their uncertainties.

%Slow neutron capture (the s-process) and high-temperature hydrogen burning (hot-bottom burning) in massive AGB stars have been investigated in this thesis. The latter process is a known origin for the abundance anticorrelations of light-element pairs in low-mass globular cluster stars. 

Key reaction rates correlated with enhanced rubidium abundance in the s-process have been determined in this thesis through Monte Carlo sampling of the reaction rate probability densities. The rates of the $^{86}\mathrm{Rb}(n,\gamma)^{87}\mathrm{Rb}$, $^{13}\mathrm{N}(n,p)^{13}\mathrm{C}$, and $^{14}\mathrm{N}(n,p)^{14}\mathrm{C}$ reactions were found to be the most correlated with the ratio of rubidium abundance to nearby s-process elements. This ratio is an indicator of s-process branching activation associated with the neutron source $^{22}\mathrm{Ne}(\alpha,n)^{25}\mathrm{Mg}$ occurring during thermal pulses, which is expected to be activated for massive O-rich AGB stars. The discrepancy between observed and theoretical rubidium abundance, known as the ``rubidium problem'', was investigated, and the rate uncertainties of the relevant reactions were found to not play a role in this discrepancy.

The Mg--K anticorrelation in the globular cluster NGC 2419 was investigated by experimentally constraining the rate of the key potassium-destruction reaction $^{39}\mathrm{K}(p,\gamma)^{40}\mathrm{Ca}$. This constraint was established through the first-ever resolution of key $^{39}\mathrm{K}+p$ resonances in the proton-transfer reaction $^{39}\mathrm{K}(^{3}\mathrm{He},d)^{40}\mathrm{Ca}$ performed at the Triangle Universities Nuclear Laboratory (TUNL) with the Enge Split-Pole Spectrograph. The resonances at 154 keV, 415 keV, 439 keV, and 521 keV were resolved for the first time, and new proton partial widths were calculated from their measured angular distributions. The impact on the $^{39}\mathrm{K}(p,\gamma)^{40}\mathrm{Ca}$ reaction rate from these new resonances, as well as that of the many other resonances resolved in this experiment, was found to be significant. A large increase in the previously calculated rate was found for $T \lesssim 110$ MK, due primarily to the new resolution of the 154 keV resonance. The rate increased by as much as a factor of 13 at 70 MK, and its total $1\sigma$ width decreased by as much as a factor of 42 at 80 MK. These massive constraints were tested in a nuclear reaction network model reproducing the Mg--K anticorrelation in NGC 2419. The $T$-$\rho$ conditions of the hydrogen burning nucleosynthesis environment that reproduce the Mg--K anticorrelation were investigated through a grid search. Several $T$-$\rho$ solutions that were acceptable with the previously calculated reaction rate were found to no longer be acceptable, and vice versa. However, massive AGB and super-AGB stars were not found to be any more or less likely to be candidates for the Mg--K anticorrelation as a result of this new reaction rate.

In order to enhance future proton-transfer measurements at TUNL, a new digital data acquisition system (DAQ) was developed that will replace the traditional analog module system for the focal plane detector package at the Enge Split-Pole Spectrograph. Both frontend software and a sort routine were developed to extract energy and timing information from events collected by a new digitizer, the CAEN V1730. Initial tests were performed on the energy and timing resolution of the DAQ, and coincidence and triggering logic were developed for use with the focal plane detector. The commissioning of this DAQ will mark a new era for the Enge Split-Pole Spectrograph at TUNL.