\chapter{\texorpdfstring{$^{39}\mathrm{\textbf{K}}(\MakeLowercase{p},\gamma)^{40}\mathrm{\textbf{C\MakeLowercase{a}}}$}{39K(p,g)40Ca} Constraints for Globular Cluster NGC 2419}
\label{ch:GC}

%%%%%%%%%%%%%%%%%%%%%%%%%%%%%%%%%%%%%%%%%%%%%%%%%%%%%%%%%%%%%%%%%%%%%%%%%%%%%%%%%%%%%%%%%%%%%
\section{Introduction}

% Globular clusters and anticorrelations in general (See Gratton et al. 2019)
%  3.1) Introduction to globular cluster abundance anticorrelations in general
%  - What is a globular cluster?
%  - Defined by multiple stellar populations (Bedin et al. 2004, Piotto 2007). Show the HR diagram for an example Milky Galaxy
%  - Defined by abundance anticorrelations. Show an example anticorrelation, such as O-Na
%  - Abundance anticorrelations result from pollution/mixing from older stars in the globular cluster (polluter stars) ["The (anti-)correlation among light elements is simply a manifestation of the multiple stellar populations." (Gratton et al. 2019)]
%  - Hydrogen burning in the polluter stars is the mechanism behind these anticorrelations (

This chapter details the original research constraining hydrogen-burning conditions responsible for abundance anomalies in the globular cluster NGC 2419. In particular, the observed potassium enrichment and magnesium depletion of red giants of NGC 2419 is reproduced with nuclear reaction network calculations using new nuclear physics information presented in this chapter. A new reaction rate of the key potassium-destroying reaction, $^{39}\mathrm{K}(p,\gamma)^{40}\mathrm{Ca}$, is calculated from the new proton partial-widths and resonance energies of astrophysically-important, proton-unbound $^{40}$Ca states measured in the proton-transfer reaction $^{39}\mathrm{K}(^{3}\mathrm{He},d)^{40}\mathrm{Ca}$. The reaction rate is significantly constrained for $T \lesssim 110$ MK. Reaction network calculations will show new temperature-density constraints for hydrogen-burning, compared with that of the previously calculated reaction rate. Hot-bottom burning in Super-AGB stars is investigated as a polluter candidate driving the Mg--K anticorrelation in NGC 2419.

%%%%%%%%%%%%%%%%%%%%%%%%%%%%%%%%%%%%%%%%%%%%%%%%%%%%%%%%%%%%%%%%%%%%%%%%%%%%%%%%%%%%%%%%%%%%%
\section{NGC 2419 and the Mg--K Anticorrelation} \label{sec:NGC2419}
% Iliadis et al. 2016
% Derhmingny et al. 2017

% PARAGRAPH 1: Mg-K abundance observations (2012)

% The globular cluster NGC 2419, imaged by the Hubble Space Telescope in Fig. \ref{fig:NGC2419}, was recently found to exhibit some puzzling abundance patterns.

The O--Na anticorrelation is ubiquitous in globular clusters, as stated in Section \ref{subsec:GC_Candidate}. However, there are other abundance patterns that are unique to individual or a small subset of globular clusters. The globular cluster NGC 2419 was recently found to exhibit some puzzling abundance patterns. About 30$\%$ of the red giants observed show a strong K enrichment correlated with a considerable Mg depletion \cite{Mucciarelli2012,Cohen2012}. Fig. \ref{fig:MgK} shows the observed elemental potassium and magnesium abundances, with respect to iron, for the red giants in NGC 2419 sampled by Ref. \cite{Mucciarelli2012} in red and Ref. \cite{Cohen2012} in blue. This was the first discovery of a Mg--K anticorrelation in any globular cluster. Only one other cluster to date, NGC 2808, has exhibited such an anticorrelation in a small portion of its stars \cite{Mucciarelli2015,Mucciarelli2017}, but the extent of the K enrichment is about 7 times less than that of NGC 2419. Meanwhile, the extent of the Mg depletion is about 3 times less. The evolutionary history of NGC 2419 must therefore be rather unique for it to exhibit such abundance patterns not seen in other globular clusters.

%\begin{figure}[t]
%\centering
%\includegraphics[width=3in]{Chapter-6/figs/NGC2419.jpg}
%\caption{\label{fig:NGC2419}An image of the globular cluster NGC 2419 taken by the Hubble Space Telescope \cite{NGC2419Hubble,Larsen2019}.}
%\end{figure}

\begin{figure}[t]
\centering
\includegraphics[width=2.5in]{Chapter-6/figs/MgK_Observation.png}
\caption{\label{fig:MgK}The observed Mg and K elemental abundances of red giants in NGC 2419 from Refs. \cite{Mucciarelli2012} (red) and \cite{Cohen2012} (blue). Figure adapted from \cite{Iliadis2016}.}
\end{figure}

% PARAGRAPH 2: Reaction network leading to K abundance (from 36Ar to 39K)

Just as the nucleosynthesis mechanism for the O--Na anticorrelation involves a series of $(p,\gamma)$ reactions and radioactive decays, the same is true for the Mg-K anticorrelation. The hydrogen burning schemes are illustrated in Fig. \ref{fig:MgKHBurning}. The Mg--Al cycle, shown on the left-hand side of the figure, starts from $^{24}$Mg at about 70 MK. After the subsequent $^{24}\mathrm{Mg}(p,\gamma)^{25}\mathrm{Al}$ and $^{25}\mathrm{Al}(\beta^{+}\nu)^{25}\mathrm{Mg}$ reactions, there is a chance the $^{26}\mathrm{Al}^{m}$ isomer is populated. The isomer preferentially $\beta^{+}$ decays to $^{26}$Mg, while the ground state $^{26}\mathrm{Al}^{g}$ preferentially captures a proton. Either scenario eventually synthesizes $^{27}$Al, which can repeat the cycle with $^{27}\mathrm{Al}(p,\alpha)^{24}\mathrm{Mg}$ or proceed with $^{27}\mathrm{Al}(p,\gamma)^{28}\mathrm{Si}$ if the temperature is more than about 80 MK. The overall effect is a depletion of magnesium and an enrichment of aluminum and silicon.

\begin{figure}[t]
\centering
\noindent
\begin{minipage}{.44\linewidth}
\begin{tikzpicture}[scale=1.5, every node/.style={transform shape}]

% (0,0) is at the bottom left corner of plot, where 24Mg is.
\filldraw[fill = light-gray] (0,0) rectangle (1,1); % 24Mg box
\filldraw[fill = light-gray] (1+\BoxSpace,0) rectangle (2+\BoxSpace,1); % 25Mg box
\filldraw[fill = light-gray] (2+\BoxSpacetwo,0) rectangle (3+\BoxSpacetwo,1); % 26Mg box
\draw (0,1+\BoxSpace) rectangle (1,2+\BoxSpace); % 25Al box
\draw (1+\BoxSpace,1+\BoxSpace) rectangle (2+\BoxSpace,2+\BoxSpace); % 26Al box
\filldraw[fill = light-gray] (2+\BoxSpacetwo,1+\BoxSpace) rectangle (3+\BoxSpacetwo,2+\BoxSpace); % 27Al box
\draw (1+\BoxSpace,2+\BoxSpacetwo) rectangle (2+\BoxSpace,3+\BoxSpacetwo); % 27Si box
\filldraw[fill = light-gray] (2+\BoxSpacetwo,2+\BoxSpacetwo) rectangle (3+\BoxSpacetwo,3+\BoxSpacetwo); % 28Si box

\node[text=blue] at (0.5,0.5) {$^{24}\mathrm{Mg}$};
\node at (1+\BoxSpace+0.5,0.5) {$^{25}\mathrm{Mg}$};
\node at (2+\BoxSpacetwo+0.5,0.5) {$^{26}\mathrm{Mg}$};
\node at (0.5,1+\BoxSpace+0.5) {$^{25}\mathrm{Al}$};
\node at (1+\BoxSpace+0.5,1+\BoxSpace+0.5) {$^{26}\mathrm{Al}$};
\node[text=red] at (2+\BoxSpacetwo+0.5,1+\BoxSpace+0.5) {$^{27}\mathrm{Al}$};
\node at (1+\BoxSpace+0.5,2+\BoxSpacetwo+0.5) {$^{27}\mathrm{Si}$};
\node[text=red] at (2+\BoxSpacetwo+0.5,2+\BoxSpacetwo+0.5) {$^{28}\mathrm{Si}$};

%\node at (1+\BoxSpace+0.65,2+\BoxSpace-0.2) {\scriptsize{g}};
%\node at (1+\BoxSpace+0.65,1+\BoxSpace+0.22) {\scriptsize{m}};

\node at (1+\BoxSpace+0.62,2+\BoxSpace+\BoxSpacehalf) {\tiny{g}};
%\node at (2+\BoxSpacetwo+0.12,1+\BoxSpacehalf+0.02) {\tiny{(m)}};
\node at (2+\BoxSpace+\BoxSpacehalf,1-0.05) {\tiny{m}};

\draw[line width = \AW mm, -{Triangle[]}] (0.5,1-\AOS) -- (0.5,1+\BoxSpace+\AOS); % 24Mg(p,g)
\draw[line width = \AW mm, -{Triangle[]}] (1-\AOS,1+\BoxSpace+\AOS) -- (1+\BoxSpace+\AOS, 1-\AOS); % 25Al(B+)
\draw[line width = \AW mm, -{Triangle[]}] (1+\BoxSpace+0.5,1-\AOS) -- (1+\BoxSpace+0.5,1+\BoxSpace+\AOS); % 25Mg(p,g)
\draw[line width = \AW mm, -{Triangle[]},dashed] (2+\BoxSpace-\AOS,1+\BoxSpace+\AOS) -- (2+\BoxSpacetwo+\AOS,1-\AOS); %*m26Al(B+)
\draw[line width = \AW mm, -{Triangle[]}] (2+\BoxSpacetwo+0.5,1-\AOS) -- (2+\BoxSpacetwo+0.5,1+\BoxSpace+\AOS); %26Mg(p,g)
\draw[line width = \AW mm, -{Triangle[]},dashed] (2+\BoxSpacetwo+0.5,2+\BoxSpace-\AOS) -- (2+\BoxSpacetwo+0.5,2+\BoxSpacetwo+\AOS); %27Al(p,g)
\draw[line width = \AW mm, -{Triangle[]},dashed] (1+\BoxSpace+0.5,2+\BoxSpace-\AOS) -- (1+\BoxSpace+0.5,2+\BoxSpacetwo+\AOS); % *g26Al(p,g)
\draw[line width = \AW mm, -{Triangle[]}] (2+\BoxSpace-\AOS,2+\BoxSpacetwo+\AOS) -- (2+\BoxSpacetwo+\AOS,2+\BoxSpace-\AOS); % 27Si(B+)
\draw[line width = \AW mm, -{Triangle[]},dashed] (2+\BoxSpacetwo+\AOS,1+\BoxSpace+\AOS) -- (1-\AOS,1-\AOS); %27Al(p,a)

\end{tikzpicture}
\end{minipage}
%
\begin{minipage}{.44\linewidth} \hfill
\centering
\noindent
\begin{tikzpicture}[scale=1.5, every node/.style={transform shape}]

% (0,0) is at the bottom left corner of plot, where 36Ar is.
\filldraw[fill = light-gray] (0,0) rectangle (1,1); % 36Ar box
\draw (1+\BoxSpace,0) rectangle (2+\BoxSpace,1); % 37Ar box
\filldraw[fill = light-gray] (2+\BoxSpacetwo,0) rectangle (3+\BoxSpacetwo,1); % 38Ar box
\filldraw[fill = light-gray] (2+\BoxSpacetwo,-1-\BoxSpace) rectangle (3+\BoxSpacetwo,-\BoxSpace); % 37Cl box
\draw (0,1+\BoxSpace) rectangle (1,2+\BoxSpace); % 37K box
\draw (1+\BoxSpace,1+\BoxSpace) rectangle (2+\BoxSpace,2+\BoxSpace); % 38K box
\filldraw[fill = light-gray] (2+\BoxSpacetwo,1+\BoxSpace) rectangle (3+\BoxSpacetwo,2+\BoxSpace); % 39K box
\filldraw[fill = light-gray] (2+\BoxSpacetwo,2+\BoxSpacetwo) rectangle (3+\BoxSpacetwo,3+\BoxSpacetwo); % 40Ca box

\node[text=blue] at (0.5,0.5) {$^{36}\mathrm{Ar}$};
\node at (1+\BoxSpace+0.5,0.5) {$^{37}\mathrm{Ar}$};
\node at (2+\BoxSpacetwo+0.5,0.5) {$^{38}\mathrm{Ar}$};
\node at (0.5,1+\BoxSpace+0.5) {$^{37}\mathrm{K}$};
\node at (1+\BoxSpace+0.5,1+\BoxSpace+0.5) {$^{38}\mathrm{K}$};
\node[text=red] at (2+\BoxSpacetwo+0.5,1+\BoxSpace+0.5) {$^{39}\mathrm{K}$};
\node at (2+\BoxSpacetwo+0.5,-\BoxSpace-0.5) {$^{37}\mathrm{Cl}$};
\node[text=red] at (2+\BoxSpacetwo+0.5,2+\BoxSpacetwo+0.5) {$^{40}\mathrm{Ca}$};

\draw[line width = \AW mm, -{Triangle[]}] (0.5,1-\AOS) -- (0.5,1+\BoxSpace+\AOS); % 36Ar(p,g)
\draw[line width = \AW mm, -{Triangle[]}] (1-\AOS,1+\BoxSpace+\AOS) -- (1+\BoxSpace+\AOS, 1-\AOS); % 37K(B+)
\draw[line width = \AW mm, -{Triangle[]}] (1+\BoxSpace+0.5,1-\AOS) -- (1+\BoxSpace+0.5,1+\BoxSpace+\AOS); % 37Ar(p,g)
\draw[line width = \AW mm, -{Triangle[]}] (2+\BoxSpace-\AOS,1+\BoxSpace+\AOS) -- (2+\BoxSpacetwo+\AOS,1-\AOS); %38K(B+)
\draw[line width = \AW mm, -{Triangle[]}] (2+\BoxSpacetwo+0.5,1-\AOS) -- (2+\BoxSpacetwo+0.5,1+\BoxSpace+\AOS); %38Ar(p,g)
\draw[line width = \AW mm, -{Triangle[]},dashed] (2+\BoxSpace-\AOS,\AOS) -- (2+\BoxSpacetwo+\AOS,-\BoxSpace-\AOS); %37Ar(EC)
\draw[line width = \AW mm, -{Triangle[]},dashed] (2+\BoxSpacetwo+0.5,-\BoxSpace-\AOS) -- (2+\BoxSpacetwo+0.5,\AOS); % 37Cl(p,g)
\draw[line width = \AW mm, -{Triangle[]}] (2+\BoxSpacetwo+0.5,2+\BoxSpace-\AOS) -- (2+\BoxSpacetwo+0.5,2+\BoxSpacetwo+\AOS); %39K(p,g)

\end{tikzpicture}
\end{minipage}
\vspace{0.75 cm}
\caption{\label{fig:MgKHBurning}The hydrogen burning mechanism driving the Mg--K anticorrelation in the globular cluster NGC 2419. The nucleosynthesis chain for the Mg depletion is the Mg--Al cycle, shown on the left-hand side. Whether the $^{26}\mathrm{Al}^{m}$ isomer or the $^{26}\mathrm{Al}^{g}$ ground state is populated, the result is $^{27}$Al production, which repeats the cycle with $^{27}\mathrm{Al}(p,\alpha)^{24}\mathrm{Mg}$. The main nucleosynthesis chain for the K enrichment is shown on the right-hand side with solid arrows. The $^{37}\mathrm{Ar}(p, \gamma)^{38}\mathrm{K}$ reaction proceeds at a much higher rate than $^{37}\mathrm{Ar}(e^{-},\nu)^{37}\mathrm{Cl}$ electron capture at the stellar densities of interest. Nuclides in blue are destroyed and nuclides in red are produced in the overall nucleosynthesis. Shaded nuclides are stable.}
\end{figure}

The hydrogen burning scheme for the K enrichment in NGC 2419 is illustrated on the right-hand side of Fig. \ref{fig:MgKHBurning}. It has been shown to start from $^{36}$Ar, an $\alpha$-nucleus and the most abundant argon isotope at the low metallicities of the first stellar generation \cite{Ventura2012}. Some authors \cite{Ventura2012,Mucciarelli2015} have incorrectly claimed that the main K enrichment chain involves the decay of $^{37}$Ar to $^{37}$Cl, then the subsequent $^{37}\mathrm{Cl}(p,\gamma)^{38}\mathrm{Ar}$ reaction, indicated by the dashed arrows in Fig. \ref{fig:MgKHBurning}. However, as Ref. \cite{Iliadis2016} indicates, the $^{37}\mathrm{Ar}(p,\gamma)^{38}\mathrm{K}$ reaction has a much larger decay constant than the electron capture reaction $^{37}\mathrm{Ar}(e^{-}, \nu)^{37}\mathrm{Cl}$ at the stellar densities of interest. Therefore, the main chain must proceed to $^{38}$K before the subsequent $^{38}\mathrm{K}(\beta^{+}\nu)^{38}\mathrm{Ar}$ and $^{38}\mathrm{Ar}(p,\gamma)^{39}\mathrm{K}$ reactions. The former chain is expected to contribute only marginally to $^{39}$K nucleosynthesis. Additionally, unlike the mechanisms for the O--Na anticorrelation and the Mg depletion, the sequence leading from $^{36}$Ar to $^{39}$K is not a cycle because the $^{39}\mathrm{K}(p,\alpha)^{36}\mathrm{Ar}$ reaction has a much smaller decay constant than $^{39}\mathrm{K}(p,\gamma)^{40}\mathrm{Ca}$. However, as it will become clear in Section \ref{sec:NGC2419}, the $^{39}\mathrm{K}(p,\gamma)^{40}\mathrm{Ca}$ reaction rate is currently very uncertain in the astrophysical temperature range of interest.

% PARAGRAPH 3: Ventura et al. (2012) polluter candidates

The nucleosynthesis mechanism for the Mg--K anticorrelation may be well-established, but the site of this nucleosynthesis is much more uncertain. Ref. \cite{Ventura2012} were the first to propose polluter star candidates for the Mg--K anticorrelation in NGC 2419, as well as the other observed abundance patterns. Of consideration were massive AGB stars and Super-AGB (SAGB) stars of $\sim 6$ $\mathrm{M}_{\odot}$, where hydrogen burning occurs during hot-bottom burning (HBB) at the base of the convective hydrogen envelope. Agreement was found with the observed intermediate K-enriched population only in an artificially optimized scenario. The $^{38}\mathrm{Ar}(p,\gamma)^{39}\mathrm{K}$ reaction rate would need to increase by a factor of 100, and the poorly known mass loss rate in the AGB models would need to simultaneously decrease by a factor of 4. However, Ref. \cite{Ventura2012} does not investigate the effect of varying other relevant reaction rates, which could have the same effect as increasing the $^{38}\mathrm{Ar}(p,\gamma)^{39}\mathrm{K}$ rate.

% PARAGRAPH 4: Iliadis et al. (2016) monte carlo network calculations constraining temperature-density conditions and suggesting polluter candidates

Ref. \cite{Iliadis2016} expanded the search for polluter candidates by performing a Monte Carlo nuclear reaction network calculation that reproduced all of the observed abundances in NGC 2419, including the Mg--K anticorrelation. The parameters that were randomly sampled include the temperature $T$, density $\rho$, final hydrogen mass fraction $X_{\mathrm{H}}$, all reaction rate probability densities in the network, and initial abundances. $T$ and $\rho$ were held constant for each network calculation. The reaction rates were obtained by truncating the Starlib \cite{Sallaska2013} library for the relevant rates. The resulting $T$--$\rho$ conditions that matched all of the observed abundances are shown by the blue dots in Fig. \ref{fig:Trho_Iliadis}. A narrow band of possible $T$--$\rho$ conditions was found between $\approx80-260$ MK and $\approx10^{-4}-10^{8}$ $\mathrm{g/cm}^{3}$, with additional solutions not shown in the figure up to $10^{11}$ $\mathrm{g/cm}^{3}$. The black curves  represent the $T$--$\rho$ hydrogen burning conditions of several polluter star candidates. These include the cores of 15 $\mathrm{M}_{\odot}$ and 120 $\mathrm{M}_{\odot}$ main sequence (MS) stars, the hydrogen-burning (H) shell of a 1 $\mathrm{M}_{\odot}$ red giant star, hot-bottom burning of a 6 $\mathrm{M}_{\odot}$ and 8 $\mathrm{M}_{\odot}$ thermally-pulsing AGB and SAGB star, respectively, a 1.15 $\mathrm{M}_{\odot}$ ONe classical nova (CN), and 0.8 $\mathrm{M}_{\odot}$ and 1.0 $\mathrm{M}_{\odot}$ CO classical novae. The only candidates that nominally overlap with this band are ONe and CO classical novae. However, another candidate Ref. \cite{Iliadis2016} considers is SAGB stars because of the large uncertainty in AGB stellar model parameters, such as the mass loss rate and the prescription of convective mixing. Adjusting these parameters within their current uncertainties could increase the temperature of HBB enough to match the observed abundances.

\begin{figure}[t]
\centering
\includegraphics[width=4.5in]{Chapter-6/figs/Trho_Iliadis2016.png}
\caption{\label{fig:Trho_Iliadis}Temperature-density conditions reproducing the Mg--K anticorrelation and other abundance patterns in NGC 2419, obtained by sampling $T$, $\rho$, $X_{\mathrm{H}}$, reaction rate probability densities, and initial abundances from a nuclear reaction network (see text). The $T$--$\rho$ conditions for several polluter star candidates are represented by the black curves. Adapted from Ref. \cite{Iliadis2016}.}
\end{figure}

% PARAGRAPH 5: Dermigny et al. (2017) reaction rate sensitivities suggesting 4 reactions are important

While Ref. \cite{Iliadis2016} simultaneously sampled the reaction rate probability densities of all the reactions in the network, they did not investigate the effect of individual reaction rates. Ref. \cite{Dermigny2017} investigated the sensitivity of the acceptable $T$--$\rho$ conditions for NGC 2419 polluter stars on unknown systematic effects of individual reaction rates. Only 4 reactions were found to make a significant impact on the acceptable $T$--$\rho$ conditions, $^{30}\mathrm{Si}(p,\gamma)^{31}\mathrm{P}$, $^{37}\mathrm{Ar}(p,\gamma)^{38}\mathrm{K}$, $^{38}\mathrm{Ar}(p,\gamma)^{39}\mathrm{K}$, and $^{39}\mathrm{K}(p,\gamma)^{40}\mathrm{Ca}$. If the reaction rate of $^{39}\mathrm{K}(p,\gamma)^{40}\mathrm{Ca}$, for example, were systematically larger than its recommended rate, by even as little as a factor of 5, the acceptable $T$--$\rho$ band would have significantly reduced scatter and would be constrained to the low-temperature side for all densities. This scenario is shown in Fig. \ref{fig:39K_p_g_Sens} on the bottom-left panel, where the black dots represent the new rate multiplied by the indicated variation factor $\alpha$, and the red dots represent the recommended rate ($\alpha = 1$). This makes sense in theory because potassium abundance would be reduced if the rate of its primary destruction reaction were increased. The high temperature conditions would therefore destroy too much potassium compared to its observed abundance. In contrast, if the rate were systematically smaller by a factor of 5, the solutions interestingly only have increased scatter. These same effects are also true for $^{30}\mathrm{Si}(p,\gamma)^{31}\mathrm{P}$, but the opposite is true for $^{37}\mathrm{Ar}(p,\gamma)^{38}\mathrm{K}$ and $^{38}\mathrm{Ar}(p,\gamma)^{39}\mathrm{K}$, as these latter reactions increase the potassium abundance.

\begin{figure}[t]
\centering
\noindent
\begin{tikzpicture}[scale=1.3, every node/.style={transform shape}]
\node at (0,0) {\includegraphics[scale=0.45]{Chapter-6/figs/39K_p_g_Sens1.png}};
\node at (0.04,-3.7) {\includegraphics[scale=0.45]{Chapter-6/figs/39K_p_g_Sens2.png}};

\node at (0.3,-5.9) {Temperature (GK)};
\node[rotate=90] at (-5.7,-1.85) {Density ($\mathrm{g}/\mathrm{cm}^{3}$)};

\end{tikzpicture}
\caption{\label{fig:39K_p_g_Sens}Systematic effects of the $^{39}\mathrm{K}(p,\gamma)^{40}\mathrm{Ca}$ reaction rate influencing temperature-density conditions. The indicated variation factors ($\alpha = 1/10, 1/5, 5, 10$) are applied to the reaction rate in each panel, and the black dots show the resulting temperature-density conditions that provide an acceptable match with observed abundances. The red dots represent the case where no systematic effects ($\alpha=1$) have been added. Figure adapted from \cite{Dermigny2017}.}
\end{figure}

%%%%%%%%%%%%%%%%%%%%%%%%%%%%%%%%%%%%%%%%%%%%%%%%%%%%%%%%%%%%%%%%%%%%%%%%%%%%%%%%%%%%%%%%%%%%%
\section{The Previous $^{39}\mathrm{\textbf{K}}(p,\gamma)^{40}\mathrm{\textbf{Ca}}$ Reaction Rate} \label{sec:prev_39K_p_g_rate}

The remainder of this chapter will focus specifically on the key potassium-destroying reaction $^{39}\mathrm{K}(p,\gamma)^{40}\mathrm{Ca}$, found by Ref. \cite{Dermigny2017} to be one of the few important reactions able to constrain the potassium abundance in NGC 2419 and therefore constrain the possible $T$--$\rho$ conditions of its polluter stars. 

\subsection{Longland \emph{et al.} (2018) Evaluation} \label{subsec:longland_rate}

The most recent $^{39}\mathrm{K}(p,\gamma)^{40}\mathrm{Ca}$ reaction rate evaluation is that of Ref. \cite{Longland2018}, which Ref. \cite{Dermigny2017} used while it was in preparation. Its probability density was calculated using a Monte Carlo reaction rate formalism \cite{Longland2010a} with resonance strengths and resonance energies provided by the direct measurements of Refs. \cite{Kikstra1990,Cheng1981,Leenhouts1966}. The lowest directly measured resonance strength from these experiments is for the resonance at $E^{\mathrm{c.m.}}_{r} = 606$ keV. For resonances below this, indirect $^{39}\mathrm{K}(^{3}\mathrm{He},d)^{40}\mathrm{Ca}$ \cite{Cage1971} and $^{39}\mathrm{K}(d,n)^{40}\mathrm{Ca}$ \cite{Fuchs1969} proton-transfer measurements provided proton partial-widths, when available. Additionally, the indirect $^{36}\mathrm{Ar}(^{6}\mathrm{Li},d)^{40}\mathrm{Ca}$ $\alpha$-transfer measurement of Ref. \cite{Yamaya1994} provided $\alpha$-particle partial widths for the $(p,\alpha)$ channel, when available. Otherwise, measured or theoretical upper limits were given to the single particle reduced widths of the remaining resonances. These upper limits are described by the Porter-Thomas distribution \cite{Porter1956,Weidenmuller2009}, truncated at the upper limit value in the case of measured upper limits. This provided a physically motivated probability density from which to sample unknown or upper-limit partial widths.

\begin{figure}[t]
\centering
\includegraphics[width=4in]{Chapter-6/figs/39K_p_g_Longland2018.png}
\caption{\label{fig:39K_p_g_Longland}The recent $^{39}\mathrm{K}(p,\gamma)^{40}\mathrm{Ca}$ reaction rate probability density calculation of \cite{Longland2018} as a function of temperature. The median, recommended rate is shown as the dotted normalization line. The thick and thin black lines represent the $68\%$ ($1\sigma$) and $95\%$ ($2\sigma$) uncertainty bands, respectively. The color scale shows the continuous nature of the probability density, with darker red colors closer to the recommended rate. The green line represents the previous calculation of Ref \cite{Cheng1981}.}
\end{figure}

The reaction rate probability density evaluated by Ref. \cite{Longland2018} is shown in Fig. \ref{fig:39K_p_g_Longland}. The reaction rate uncertainty is represented on the $y$-axis as a factor of the mean, recommended rate, which has been normalized to unity. The 68$\%$ (1$\sigma$) and 95$\%$ (2$\sigma$) uncertainty bands are represented by the thick and thin black lines, respectively, while the color scale indicates the continuity of the probability density. Clearly, the rate uncertainty is very large between about $0.05-0.2$ GK ($50-200$ MK), where the total $1\sigma$ width peaks at a factor of 84 at about 80 MK. Recall from Section \ref{sec:NGC2419} that the astrophysically-relevant temperature range spanning the $T$--$\rho$ band in Figs. \ref{fig:Trho_Iliadis} and \ref{fig:39K_p_g_Sens} is $\approx80-260$ MK. The most recent $^{39}\mathrm{K}(p,\gamma)^{40}\mathrm{Ca}$ reaction rate is therefore very uncertain in the astrophysical temperature range of interest. This is concerning, given that Ref. \cite{Dermigny2017} found that the $T$--$\rho$ conditions are sensitive to systematic variations in the $^{39}\mathrm{K}(p,\gamma)^{40}\mathrm{Ca}$ rate, by even as little as a factor of 5. This motivates the necessity of constraining the $^{39}\mathrm{K}(p,\gamma)^{40}\mathrm{Ca}$ rate, particularly for the contributing resonances between $\approx50-260$ MK.

\subsection{Contributing Resonances}

The $^{39}\mathrm{K}+p$ resonances that contribute to the reaction rate in the $\approx50-260$ MK range span resonance energies of $E^{\mathrm{c.m.}}_{r} \approx 90-450$ keV (See Section \ref{sec:rates}) and $^{40}$Ca excited state energies of about $E_{x} \approx 8400-8800$ keV, where the most recent $^{40}\mathrm{Ca}$ proton separation energy measurement is $S_{p} = 8328.18(2)$ keV from Ref. \cite{Wang2021}. This resonance energy range is below the lowest directly measured resonance at $E^{\mathrm{c.m.}}_{r} = 606$ keV ($E_{x} = 8935$ keV), indicating that this region is entirely described by inferred transfer reaction partial widths or upper limits at present.

\begin{figure}[!p]
\centering
\includegraphics[width=6in]{Chapter-6/figs/Contrib_Longland2018.png}
\caption{\label{fig:contrib_longland2018}The individual resonance contributions to the $^{39}\mathrm{K}(p,\gamma)^{40}\mathrm{Ca}$ reaction rate from Ref. \cite{Longland2018} as a fraction of unity. Measured resonances are represented with shading or hatched lines and are shown with their $1\sigma$ uncertainty bands. Resonances from upper limits are represented by dashed lines and show their 84$\%$ (+$1\sigma$) value. The black line represents the total resonance contribution of all resonances that individually contribute less than $20\%$ to the total reaction rate.}
\end{figure}

The contributing resonances from the reaction rate evaluation of Ref. \cite{Longland2018} are shown in Fig. \ref{fig:contrib_longland2018}. The individual, fractional resonance contribution for each resonance is shown as a function of temperature. Measured resonances are shown with their $1\sigma$ width, representative of the Monte Carlo formalism \cite{Longland2010a} and are shaded or have hatched lines for clarity. Upper limit resonances are represented by dashed lines at their $84\%$ (+$1\sigma$) value. All resonances that individually contribute at least $20\%$ to the total reaction rate are shown in the figure with corresponding resonance energy labels, while the remaining resonance contributions sum to the black line. Of note are the 96 keV, 156 keV, and 337 keV resonances, which overwhelmingly contribute to the reaction rate at their respective temperatures. The 337 keV resonance, in particular, is the most important resonance between $\approx 0.15-0.4$ GK ($150-400$ MK), which makes up most of the astrophysical temperature range of interest, $80-260$ MK. However, the discrepancy between the $84\%$ ($+1\sigma$) contribution and the $16\%$ (-$1\sigma$) contribution is very large. The large uncertainty in the reaction rate around $50-200$ MK in Fig. \ref{fig:39K_p_g_Longland} is partly due to the uncertainty in this resonance contribution, but it is mostly a result of the many unknown resonances with only upper limit values between the 96 keV resonance and the 337 keV resonance.

Low-lying resonances at astrophysically-important proton energies are difficult to populate for $^{39}\mathrm{K}(p,\gamma)^{40}\mathrm{Ca}$, hence why the lowest directly measured resonance strength is that of the 606 keV resonance. At energies lower than this, an alternative to populating these same states is to use proton-transfer reactions, such as $(^{3}\mathrm{He},d)$ or $(d,n)$, which are not constrained by the Coulomb barrier in the same manner as $(p,\gamma)$. The $^{39}\mathrm{K}(^{3}\mathrm{He},d)^{40}\mathrm{Ca}$ reaction was measured in this thesis for the purpose of constraining the $^{39}\mathrm{K}(p,\gamma)^{40}\mathrm{Ca}$ reaction rate.

\subsection{Previous Proton-Transfer Measurements}

The previous $^{39}\mathrm{K}(^{3}\mathrm{He},d)^{40}\mathrm{Ca}$ \cite{Erskine1966,Seth1967,Forster1970,Cage1971} and $^{39}\mathrm{K}(d,n)^{40}\mathrm{Ca}$ \cite{Fuchs1969} proton-transfer measurements were not capable of achieving high resolution, particularly for unbound $^{40}$Ca states, i.e. at energies above the proton separation energy. The ($^{3}\mathrm{He},d$) measurements of Refs. \cite{Erskine1966,Seth1967,Cage1971} obtained spectroscopic factors ($C^{2}S$) for only 2 unbound $^{40}$Ca states, with ENSDF \cite{Chen2017} excitation energies of $E_{x} = 8425$ keV ($E^{\mathrm{c.m.}}_{r} = 96$ keV) and 8551 keV ($E^{\mathrm{c.m.}}_{r} = 223$ keV), which are both below the lowest directly measured resonance at $E_{x} = 8935$ ($E^{\mathrm{c.m.}}_{r} = 606$ keV). Meanwhile, Ref. \cite{Forster1970} did not observe unbound $^{40}$Ca states due to their focus on lower energy excited states. The $(d,n)$ measurement of Ref. \cite{Fuchs1969} obtained $C^{2}S$ for 14 unbound $^{40}$Ca states, with only 4 of these states below the lowest directly measured resonance, including $8359$ keV ($E^{\mathrm{c.m.}}_{r} = 30$ keV), $8425$ keV ($E^{\mathrm{c.m.}}_{r} = 96$ keV), $8551$ keV ($E^{\mathrm{c.m.}}_{r} = 223$ keV), and $8665$ keV ($E^{\mathrm{c.m.}}_{r} = 337$ keV). Hence, there is an overlap between the $(^{3}\mathrm{He},d)$ and $(d,n)$ measurements of 2 unbound $^{40}$Ca states, 8425 keV ($E^{\mathrm{c.m.}}_{r} = 96$ keV) and 8551 keV ($E^{\mathrm{c.m.}}_{r} = 223$ keV). The $^{39}\mathrm{K}(p,\gamma)^{40}\mathrm{Ca}$ reaction rate calculation of Ref. \cite{Longland2018} used the $C^{2}S$ measurements of Ref. \cite{Cage1971} for these 2 states, due to their more sophisticated coupled-channel DWBA calculations. They used the $C^{2}S$ measurement of Ref. \cite{Fuchs1969} for the 8665 keV ($E^{\mathrm{c.m.}}_{r} = 337$ keV) state. However, the reaction rate calculation did not include the 8359 keV ($E^{\mathrm{c.m.}}_{r} = 30$ keV) $C^{2}S$ measurement of Ref. \cite{Fuchs1969}, as this state has an energy discrepancy with ENSDF \cite{Chen2017} of 12 keV. This resonance was instead given an upper limit. Several unknown resonances remain below $E_{x} = 8935$ keV ($E^{\mathrm{c.m.}}_{r} = 606$ keV). Ref. \cite{Longland2018} uses upper limits for 18 resonances in this region. Hence, this was strong motivation to revisit the proton-transfer reaction at modern resolution to potentially resolve the currently unknown resonances.

%%%%%%%%%%%%%%%%%%%%%%%%%%%%%%%%%%%%%%%%%%%%%%%%%%%%%%%%%%%%%%%%%%%%%%%%%%%%%%%%%%%%%%%%%%%%%
\section{The $^{39}\mathrm{\textbf{K}}(^{3}\mathrm{\textbf{He}},d)^{40}\mathrm{\textbf{Ca}}$ Experiment}

% PARAGRAPH 1: Basically like the methods section in my 39K(3He,d)40Ca PRL paper, but go into much more detail about the tandem, enge, and detector setup. Note: I will have already covered these in detail in Ch 4
% PARAGRAPH 2: Show an example spectrum with peaks labeled
% PARAGRAPH 3: Different spectra from the FP and Si detectors, including DE/E for particle discrimination.

The $^{39}\mathrm{K}(^{3}\mathrm{He}, d)^{40}\mathrm{Ca}$ experiment was performed in 2019 over 5 days using the Enge Split-Pole Spectrograph at the Triangle Universities Nuclear Laboratory (TUNL). The 10 MV FN tandem Van de Graaff accelerator at TUNL accelerated a fully-ionized $^{3}$He beam to  $E_{\mathrm{lab}} = 21$ MeV, where the energy was stabilized using a pair of high-resolution slits between two $90^{\circ}$ dipole magnets. The $^{39}\mathrm{K}(^{3}\mathrm{He},d)^{40}\mathrm{Ca}$ proton transfer reaction and $^{39}\mathrm{K}(^{3}\mathrm{He},^{3}\mathrm{He})^{39}\mathrm{K}$ elastic scattering were measured separately between $\theta_{\mathrm{lab}} = 5-20^{\circ}$ and $ \theta_{\mathrm{lab}} = 15-59^{\circ}$, respectively, using the focal plane detector package described by Ref. \cite{Marshall2019}. The $d$ and $^{3}$He particles that were ejected from the target at the angle $\theta_{\mathrm{lab}}$ passed through the magnetic field $B$ of the Enge Split-Pole Spectrograph, where they were then focused at a position along the focal plane, based on their magnetic rigidity $B\rho$. The focal-plane detector was preemptively moved via dual motors to align itself with the focal plane. Meanwhile, for every $^{39}\mathrm{K}(^{3}\mathrm{He},d)^{40}\mathrm{Ca}$ or $^{39}\mathrm{K}(^{3}\mathrm{He},^{3}\mathrm{He})^{39}\mathrm{K}$ focal-plane measurement, elastic scattering was simultaneously measured with a silicon detector telescope inside the target chamber, positioned at a constant $45^{\circ}$ from the beamline to monitor potential target degradation and other target properties.

\subsection{Targets}

A total of 7 natural KI ($93.26\%$ $^{39}$K, $6.73\%$ $^{41}$K) targets were made throughout the course of the experiment, in 2 separate evaporation batches. The targets labeled KI $\#$1, KI $\#$2, and KI $\#$3 were made in the first batch, while those labeled KI $\#$4, KI $\#$5, KI $\#$6, and KI $\#$7 were made in the second batch, on the second day of the experiment. For each batch, the targets were simultaneously produced by evaporating approximately 75 $\mu\mathrm{g/cm}^{2}$ of natural KI onto aluminum target frames, each with a 21 $\mu\mathrm{g/cm}^{2}$ natural carbon ($98.84\%$ $^{12}$C, $0.96\%$ $^{13}$C) foil backing. The KI target thickness was measured by a thickness monitor inside the evaporation chamber, placed at roughly the same distance from the evaporation material as were the aluminum target frames with carbon backings. Fig. \ref{fig:KI_Evap} shows a photograph of the chamber just before the evaporation process, with 3 natural carbon targets in position. A separate 21 $\mu\mathrm{g/cm}^{2}$ natural carbon target was also used in the experiment for calibrations with the $^{12}\mathrm{C}(^{3}\mathrm{He},d)^{13}\mathrm{N}$, $^{13}\mathrm{C}(^{3}\mathrm{He},d)^{14}\mathrm{N}$, and $^{16}\mathrm{O}(^{3}\mathrm{He},d)^{17}\mathrm{F}$ reactions, among other contaminants.

\begin{figure}[!p]
\centering
\begin{tikzpicture}[scale=1.0, every node/.style={transform shape}]
\node[rotate=270] at (0,0) {\includegraphics[width=5.4in]{Chapter-6/figs/KI_Evap.jpg}};
\end{tikzpicture}
\caption{\label{fig:KI_Evap}The evaporation chamber before the production of KI targets. The 3 targets shown in position become the KI $\#$1, KI $\#$2, and KI $\#$3 targets after evaporation. The thickness monitor is also shown in position around a fourth opening in the 16-slot target holder.}
\end{figure}

Potassium iodide is hygroscopic, meaning it will eventually oxidize at atmospheric pressure. To prevent this, care was taken to transport the targets from the evaporation chamber to the target chamber in a vacuum-sealed box at rough vacuum ($\sim 10^{-2}$ torr), where they were then exposed to high vacuum ($\sim 10^{-6}$ torr) for the remainder of the experiment. However, as will be described later, the first batch of targets (KI $\#$1, KI $\#$2, and KI $\#$3) oxidized at some point during the first day of the experiment, for the $\theta_{\mathrm{lab}} = 5^{\circ}$ and $7^{\circ}$ ($^{3}\mathrm{He},d$) measurements. The focal-plane spectra for these runs, exclusively using the KI $\#$1 target, consisted of peaks with high-energy tails, expected of targets that undergo oxidation \cite{Landau1944}. These angles were measured again toward the end of the experiment with the non-oxidized KI $\#$6 target, but the oxidized KI $\#$1 data was not discarded nevertheless. Section \ref{subsec:oxidation} details a novel technique for fitting these spectra, an example of which is shown in Section \ref{subsec:oxidized_spectra}.

\subsection{Focal-Plane Spectra}

\def\yTS{1.29} % y tick space
\def\xTS{2.165} % x tick space

\begin{figure}[t]
\centering
\begin{tikzpicture}[scale=1.0, every node/.style={transform shape}]
\node at (0,0) {\includegraphics[width=5.5in]{Chapter-6/figs/5deg_DEvsE.png}};
\node at (-8.0,-2) {\rlap{\includegraphics[scale=0.6]{Chapter-6/figs/5deg_DEvsE_Scale.png}}};

\node[rotate=90] at (-8.0,0) {$\Delta$E};
\node at (0,-3.95) {E};

\node at (-7.4,-2.15) {\rlap{\footnotesize{50}}};
\node at (-7.55,-2.15+\yTS) {\rlap{\footnotesize{100}}};
\node at (-7.55,{-2.15+(2*\yTS)}) {\rlap{\footnotesize{150}}};
\node at (-7.55,{-2.15+(3*\yTS)}) {\rlap{\footnotesize{200}}};
\node at (-7.55,{-2.15+(4*\yTS)}) {\rlap{\footnotesize{250}}};

\node at (-5.2,-3.55) {\rlap{\footnotesize{50}}};
\node at (-5.2+\xTS,-3.55) {\rlap{\footnotesize{100}}};
\node at ({-5.2+(2*\xTS)},-3.55) {\rlap{\footnotesize{150}}};
\node at ({-5.2+(3*\xTS)},-3.55) {\rlap{\footnotesize{200}}};
\node at ({-5.2+(4*\xTS)},-3.55) {\rlap{\footnotesize{250}}};
\node at ({-5.2+(5*\xTS)},-3.55) {\rlap{\footnotesize{300}}};

\node at (-8.65,-1.35) {\rlap{\scriptsize{1296}}};
\node at (-8.5,-1.62) {\rlap{\scriptsize{216}}};
\node at (-8.35,-1.92) {\rlap{\scriptsize{36}}};
\node at (-8.20,-2.22) {\rlap{\scriptsize{6}}};
\node at (-8.20,-2.52) {\rlap{\scriptsize{1}}};
\node at (-8.20,-2.82) {\rlap{\scriptsize{0}}};

\node at (1,-0.8) {\rlap{\LARGE{$\textbf{d}$}}};
\node at (-3.5,-0.1) {\rlap{\LARGE{$\textbf{t}$}}};
\node at (1.9,0.65) {\rlap{\LARGE{$^{\textbf{3}}$\textbf{He}}}};

\end{tikzpicture}
\caption{\label{fig:5deg_DEvsE}A 2D histogram of the focal-plane $\Delta$E vs E detector signals. Different groups correspond to different particles depending on their mass and charge. The deuteron group, highlighted in red, is being gated on in the figure.}
\end{figure}

The focal-plane detector package (see Ref. \cite{Marshall2019}) consists of two position-sensitive avalanche counters, one located in the front of the detector (P1 section) and one located near the back (P2 section), a gas proportionality counter ($\Delta\mathrm{E}$ section), and a residual energy scintillator (E section). The data acquisition system (DAQ) triggers on the E section, located at the back of the detector, so that coincidences with all sections are established. The combination of the $\Delta$E and E sections enables particle discrimination based on mass and charge, allowing a given particle to be gated on and therefore filtering out undesired particles and detector noise in other spectra. In the $^{39}\mathrm{K}(^{3}\mathrm{He},d)^{40}\mathrm{Ca}$ experiment, the $\Delta\mathrm{E}/\mathrm{E}$ spectrum was used to gate on the deuteron group for $^{39}\mathrm{K}(^{3}\mathrm{He},d)^{40}\mathrm{Ca}$ and the $^{3}\mathrm{He}$ group for $^{39}\mathrm{K}(^{3}\mathrm{He},^{3}\mathrm{He})^{39}\mathrm{K}$. Figure \ref{fig:5deg_DEvsE} shows an example 2D histogram of $\Delta$E vs E in the offline \texttt{Jam} software package \cite{Swartz2001,Jam} for the KI $\#$6 target at $\theta_{\mathrm{lab}} = 5^{\circ}$ and an Enge magnetic field of $B = 1.14$ T. The $d$, $t$, and $^{3}$He particles appear in groups, with the $d$ group highlighted in red to represent a gate. The $p$ group is joined by noise in the $\Delta$E detector below the $\Delta$E threshold and are filtered out along with the other particle groups. Both the $\Delta$E and E signals are compressed from their original 4096 channels to 512 channels in the 2D spectrum. 

The 1D histogram of the P1 section, gated on the deuteron group of Fig \ref{fig:5deg_DEvsE}, is shown in Fig. \ref{fig:5deg_fp} in red. The overlayed black spectrum is from a carbon target to show overlapping peaks, indicating contaminants in the KI target. The deuteron peaks shown only in red were ejected from $^{39}\mathrm{K}(^{3}\mathrm{He},d)^{40}\mathrm{Ca}$ reactions in the KI target, where different excited states of the $^{40}$Ca recoil nuclei were populated, resulting in different kinematics for each deuteron depending on the excited state. High-energy deuterons are associated with high-energy $^{40}$Ca excited states, and vice versa. The deuterons are momentum-separated by the Enge Split-Pole Spectrograph, focusing the higher energy ones on the left side of the position sections and the lower energy ones on the right side. Thus, the P1 channel number in Fig. \ref{fig:5deg_fp} is inversely related to the $^{40}$Ca excited state energies. The same is true of the excited states from the contaminants.

\def\yTS{0.72}
\def\xTS{2.12}

\begin{figure}[t]
\centering
\begin{tikzpicture}[scale=1.0, every node/.style={transform shape}]
\node[rotate=90] at (-8.50,0) {P1 Counts};
\node at (0,-3.6) {Channel};
\node at (0,0) {\includegraphics[width=6in]{Chapter-6/figs/5deg_KI6.png}};
\node at (-7.85,-3.0) {\rlap{\footnotesize{0}}};
\node at (-8.00,-3.0+\yTS) {\rlap{\footnotesize{20}}};
\node at (-8.00,{-3.0+(2*\yTS)}) {\rlap{\footnotesize{40}}};
\node at (-8.00,{-3.0+(3*\yTS)}) {\rlap{\footnotesize{60}}};
\node at (-8.00,{-3.0+(4*\yTS)}) {\rlap{\footnotesize{80}}};
\node at (-8.20,{-3.0+(5*\yTS)}) {\rlap{\footnotesize{100}}};
\node at (-8.20,{-3.0+(6*\yTS)}) {\rlap{\footnotesize{120}}};
\node at (-8.20,{-3.0+(7*\yTS)}) {\rlap{\footnotesize{140}}};
\node at (-8.20,{-3.0+(8*\yTS)}) {\rlap{\footnotesize{160}}};

\node at (-5.95,-3.18) {\rlap{\footnotesize{1000}}};
\node at (-5.95+\xTS,-3.18) {\rlap{\footnotesize{1500}}};
\node at ({-5.95+(2*\xTS)},-3.18) {\rlap{\footnotesize{2000}}};
\node at ({-5.95+(3*\xTS)},-3.18) {\rlap{\footnotesize{2500}}};
\node at ({-5.95+(4*\xTS)},-3.18) {\rlap{\footnotesize{3000}}};
\node at ({-5.95+(5*\xTS)},-3.18) {\rlap{\footnotesize{3500}}};
\node at ({-5.95+(6*\xTS)},-3.18) {\rlap{\footnotesize{4000}}};
\end{tikzpicture}
\caption{\label{fig:5deg_fp}A histogram spectrum of the front (P1) focal-plane detector position section at $\theta_{\mathrm{lab}} = 5^{\circ}$ and $E_{\mathrm{lab}} = 21$ MeV with the KI $\#$6 target (in red) and the carbon target (in black), both gated on the deuteron group from the the $\Delta$E vs. E spectrum. Overlapping peaks indicate contaminants.}
\end{figure}

\subsection{Energy Shifts} \label{subsec:energy_shifts}

% I want to cover all the angles, targets, magnetic fields, and sort groups. Introduce what I mean by sort groups. Cover energy shifts for various reasons, the Pos1 vs Event spectrum, etc.

The P1 peak centroid positions typically remain constant over time. However, certain changes in the beam can introduce shifts in these positions, broadening the peaks over more channels than usual. For instance, adjusting the beam steerers or focusing elements at any point along the beamline can cause very slight kinematic differences to occur in the target. This is because the target material is never perfectly uniformly-distributed. Sometimes these differences in kinematics are large enough to cause a noticeable drift in the P1 spectrum. Retuning the beam between runs at the same Engle angle $\theta_{\mathrm{lab}}$ is the most common example of energy shifts, followed by switching between targets from a different evaporation batch. Any unintentional beam drift can also cause them, albeit usually to a lesser degree.

To visualize these energy shifts, the P1 spectrum must be traced over time. This feature was recently added to the offline version of the \texttt{EngeSpec} \cite{EngeSpec} graphical user interface (GUI), and I implemented the same feature in my local \texttt{Jam} source code. For each event, the P1 data is collected and the event number is recorded during the sort routine. The total number of events is unknown until the end of the sort routine, which unfortunately makes an online version of this feature nearly impossible. The 4096-channel P1 data and the corresponding event numbers are compressed into 512 bins, where they each get incremented into a 2D histogram.

\def\yTS{0.635} % y tick space
\def\xTS{2.48} % x tick space

\begin{figure}[t]
\centering
\begin{tikzpicture}[scale=1.12, every node/.style={transform shape}]
\node at (0,0) {\includegraphics[width=5in]{Chapter-6/figs/9deg_AllRuns_Pos1vsEvent.png}};
\node at (0,-3.8) {Event Number};
\node[rotate=90] at (-7.5,0) {P1};
\node at (-6.7,2.4) {\footnotesize{304}};
\node at (-6.7,2.4-\yTS) {\footnotesize{302}};
\node at (-6.7,{2.4-(2*\yTS)}) {\footnotesize{300}};
\node at (-6.7,{2.4-(3*\yTS)}) {\footnotesize{298}};
\node at (-6.7,{2.4-(4*\yTS)}) {\footnotesize{296}};
\node at (-6.7,{2.4-(5*\yTS)}) {\footnotesize{294}};
\node at (-6.7,{2.4-(6*\yTS)}) {\footnotesize{292}};
\node at (-6.7,{2.4-(7*\yTS)}) {\footnotesize{290}};
\node at (-6.7,{2.4-(8*\yTS)}) {\footnotesize{288}};

\node at (-3.9,-3.3) {\footnotesize{100}};
\node at (-3.9+\xTS,-3.3) {\footnotesize{200}};
\node at ({-3.9+(2*\xTS)},-3.3) {\footnotesize{300}};
\node at ({-3.9+(3*\xTS)},-3.3) {\footnotesize{400}};
\node at ({-3.9+(4*\xTS)},-3.3) {\footnotesize{500}};
\node at (-7.72,-1.05) {\scriptsize{243}};
\node at (-7.68,-1.35) {\scriptsize{81}};
\node at (-7.68,-1.65) {\scriptsize{27}};
\node at (-7.62,-1.95) {\scriptsize{9}};
\node at (-7.62,-2.25) {\scriptsize{3}};
\node at (-7.62,-2.55) {\scriptsize{1}};
\node at (-7.62,-2.85) {\scriptsize{0}};

%\hspace{-0.2cm}
% Fix first \includegraphics figure in place, and place second figure to the bottom left (right) with \rlap (\llap). Use \rlap{\raisebox{3cm}{\include ...}} to raise the figure
\node at (-7.5,-2) {\rlap{\includegraphics[scale=0.6]{Chapter-6/figs/9deg_Scale.png}}};

\end{tikzpicture}
\caption{\label{fig:9deg_P1vsEvt}A 2D histogram of the front (P1) focal-plane position section vs event number over all $^{39}\mathrm{K}(^{3}\mathrm{He},d)^{40}\mathrm{Ca}$ runs ($\#23-\#27$) at $\theta_{\mathrm{lab}} = 9^{\circ}$ with the KI $\#$5 target. Both axes are compressed to 512 channels. The P1 section is zoomed in on the $^{13}$N ground state to show the effect of energy shifts due to beam retunes. A beam retune after run $\#$25 caused the shift in the P1 spectrum shown at event (bin) number 320. The unknown phenomenon at event (bin) number 100 occurred during run $\#$23, which caused a reduction in counts but no energy shift.}
\end{figure}

There were a few instances of energy shifts during the $^{39}\mathrm{K}(^{3}\mathrm{He},d)^{40}\mathrm{Ca}$ experiment. One happened due to a necessary beam-retune after a tour group was scheduled to view the tandem accelerator. The low-energy beamline Faraday cup was put in place for about an hour, and tuning was required to get the beam back on target. The runs before and after this retune are represented by the 2D histogram of P1 vs event number in Fig. \ref{fig:9deg_P1vsEvt}. Depicted in the figure are the runs $\#23-\#27$ at $\theta_{\mathrm{lab}} = 9^{\circ}$ with the KI $\#$5 target. The P1 section is zoomed in on the $^{13}$N ground state, as it had the most counts and is therefore easiest to visualize energy shifts. The $^{13}$N ground state centroid starts at P1 bin number 296, and a clear shift to bin number 297 is seen at event number 320. A 1-bin shift in the 512-bin 2D spectrum corresponds to an 8-channel shift in the 4096-channel P1 spectrum, which was noticeable even during the experiment. Based on the BCI of each run, it was easy to determine that the shift happened between runs $\#$25 and $\#$26, exactly when the retune happened. This is also evident from the shift in counts at this point because more beam current was acquired as a result of the retune. There is also an unknown phenomenon just after event bin number 100, occurring during run $\#$23. The peaks do not appear to be shifted from it, at least not as dramatically as from the retune, but it is clear that the beam current changed abruptly.

\begin{table}[t]
\centering
\caption{\label{tab:run}Information on each run of the $^{39}\mathrm{K}(^{3}\mathrm{He}, d)^{40}\mathrm{Ca}$ experiment, including the angle of the Enge, the target used, the magnetic field of the Enge, and the sort group (see text).}
%\begin{tabularx}{\textwidth}{ccccc}
\begin{tabular}{ccccc}
\hline\midrule
Sort Group&$\theta_{\mathrm{lab}}$ $[^{\circ}]$&Run Numbers&Target&Enge Field [$T$]\\ \midrule
1&5&9, 10&KI $\#$1&1.14\\
2&5&11&KI $\#$1&1.14\\
3&5&$83-87$&KI $\#$6&1.14\\
4&7&$15-20$&KI $\#$1&1.14\\
5&7&72, $75-78$&KI $\#$6&1.14\\
6&9&23&KI $\#$5&1.145\\
7&9&24, 25&KI $\#$5&1.145\\
8&9&26, 27&KI $\#$5&1.145\\
9&11&$31-33$&KI $\#$5&1.145\\
10&13&$39-41$&KI $\#$5&1.145\\
11&13&42&KI $\#$5&1.145\\
12&13&43, 44&KI $\#$5&1.145\\
13&13&45, 47&KI $\#$5, KI $\#$6&1.145\\
14&15&$55-59$&KI $\#$6&1.145\\
15&20&60, 61, 64&KI $\#$6&1.145\\
16&20&65, 66&KI $\#$6&1.145\\
\hline\hline
%\end{tabularx}
\end{tabular}
%\footnotetext{Even though this sort group uses 2 different targets, the focal-plane peaks do not shift between them. The targets are from the same evaporation batch and therefore should have the same thickness.}
\end{table}

It would not be appropriate to sort all runs at $\theta_{\mathrm{lab}} = 9^{\circ}$ together for the $^{39}\mathrm{K}(^{3}\mathrm{He},d)^{40}\mathrm{Ca}$ energy calibration and yield analysis. Because of the presence of energy shifts at some angles, it was necessary to group runs together based on where the energy shifts occurred. I refer to these groups as \emph{sort groups}. All of the runs in the $^{39}\mathrm{K}(^{3}\mathrm{He},d)^{40}\mathrm{Ca}$ experiment, as well as their sort groups, are shown in Table \ref{tab:run}. Because there were 7 angles measured in this experiment ($\theta_{\mathrm{lab}} = 5^{\circ}, 7^{\circ}, 9^{\circ}, 11^{\circ}, 13^{\circ}, 15^{\circ}$ and $20^{\circ}$), the ideal scenario would be to have 7 sort groups, 1 for each angle. However, there are 16 sort groups for various reasons, including beam retunes, switching targets, and the occasional beam noise from beamline element power supplies failing. In the case of sort group $\#$13 at $\theta_{\mathrm{lab}} = 13^{\circ}$, consisting of runs $\#$45 and $\#$47, 2 different targets are used between them, but they are from the same evaporation batch and therefore did not cause an energy shift. The $^{39}\mathrm{K}(^{3}\mathrm{He}, {}^{3}\mathrm{He})^{39}\mathrm{K}$ P1 spectra fortunately did not contain energy shifts because only a single run was needed per angle, and the same target (KI $\#$7) was used for each run.

In the case of the $^{39}\mathrm{K}(^{3}\mathrm{He},d)^{40}\mathrm{Ca}$ run $\#$23 at $\theta_{\mathrm{lab}} = 9^{\circ}$, and for a few other runs, the energy shift occurred during the run, not between runs. For this reason, it was also necessary to have the ability to gate on the P1 vs Event Number histogram for the entire area before or after a shift. I added this feature to both my local \texttt{EngeSpec} and \texttt{Jam} source codes. Only the events occurring in the gate are used in a new, event-gated P1 spectrum. Unfortunately, this means that the recorded BCI is no longer accurate. An approximation to the new, reduced BCI is to scale the original BCI by the fraction of events in the gate. This is not exact due to the loss of resolution when compressing the events into 512 bins, but it is more than sufficient considering the much larger uncertainty from the BCI measurement itself. The same fraction can be applied to the time-keeping scalars to correct for the new deadtime. The exact same gate is also used in these cases for the simultaneous silicon detector telescope spectra, as introduced in the next section, to ensure consistency with the reduced BCI and deadtime between the focal-plane and silicon detector spectra. This is essential for a relative measurement between the two detectors, as discussed in Section \ref{subsec:SiNorm}.

\subsection{Silicon Detector Telescope} \label{subsec:SiDet}

To minimize the effects of uncertainty in the target thickness, non-uniformity in the target, and target degradation after exposure to the beam, the $^{39}\mathrm{K}(^{3}\mathrm{He}, {}^{3}\mathrm{He})^{39}\mathrm{K}$ elastic scattering and $^{39}\mathrm{K}(^{3}\mathrm{He}, d)^{40}\mathrm{Ca}$ proton-transfer yields from the focal-plane P1 spectra were normalized to the simultaneous $^{39}\mathrm{K}(^{3}\mathrm{He}, {}^{3}\mathrm{He})^{39}\mathrm{K}$ elastic scattering yield of a Si detector telescope positioned at a constant $\theta_{\mathrm{lab}} = 45^{\circ}$ inside the target chamber. The scale for the Si-normalized focal-plane differential cross-section measurements was then corrected using the global $^{3}\mathrm{He}$ optical model potential (OMP) differential cross-section for $^{39}\mathrm{K}$ from Ref. \cite{Liang2009} (see Sections \ref{subsec:SiNorm} and \ref{subsec:global_norm}). The scaling factor was the ratio between the differential cross-section of the global OMP and that of the Si-normalized focal-plane $^{39}\mathrm{K}(^{3}\mathrm{He}, {}^{3}\mathrm{He})^{39}\mathrm{K}$ measurements.

The silicon detector telescope consists of a $\Delta$E detector and a thicker, residual energy E detector, used in coincidence for particle discrimination in the same way as the $\Delta$E and E focal-plane detectors. A representative 2D $\Delta$E vs E histogram for the silicon detectors at $\theta_{\mathrm{lab}} = 45^{\circ}$ with the KI $\#$6 target is shown in Fig. \ref{fig:5deg_SiDEvsSiE}. As before, different groups correspond to different particles depending on mass and charge. The highlighted gate is around the $^{3}$He group. There are 4 regions in this group that are more dense than the background and that 

%\newpage

\def\yTS{1.175} % y tick space
\def\xTS{2.165} % x tick space
\def\sl{0.28} % shift left
\def\su{0.4} % shift up

\begin{figure}
\centering
\begin{tikzpicture}[scale=1.0, every node/.style={transform shape}]
\node at (0,0) {\includegraphics[width=5.5in]{Chapter-6/figs/5deg_SiDEvsSiE.png}};
\node at (-8.0,-2.15) {\rlap{\includegraphics[scale=0.7]{Chapter-6/figs/9deg_Scale.png}}};

\node[rotate=90] at (-8.0,0) {Si $\Delta$E};
\node at (0,-3.95) {Si E};

\node at (-7.4,-2.6) {\rlap{\footnotesize{50}}};
\node at (-7.55,-2.6+\yTS) {\rlap{\footnotesize{100}}};
\node at (-7.55,{-2.6+(2*\yTS)}) {\rlap{\footnotesize{150}}};
\node at (-7.55,{-2.6+(3*\yTS)}) {\rlap{\footnotesize{200}}};
\node at (-7.55,{-2.6+(4*\yTS)}) {\rlap{\footnotesize{250}}};
\node at (-7.55,{-2.6+(5*\yTS)}) {\rlap{\footnotesize{300}}};

\node at (-6.05,-3.55) {\rlap{\footnotesize{100}}};
\node at (-6.05+\xTS,-3.55) {\rlap{\footnotesize{150}}};
\node at ({-6.05+(2*\xTS)},-3.55) {\rlap{\footnotesize{200}}};
\node at ({-6.05+(3*\xTS)},-3.55) {\rlap{\footnotesize{250}}};
\node at ({-6.05+(4*\xTS)},-3.55) {\rlap{\footnotesize{300}}};
\node at ({-6.05+(5*\xTS)},-3.55) {\rlap{\footnotesize{350}}};

\node at (-8.65-\sl,-1.35+\su) {\rlap{\scriptsize{279964}}};
\node at (-8.50-\sl,-1.62+\su) {\rlap{\scriptsize{46654}}};
\node at (-8.35-\sl,-1.92+\su) {\rlap{\scriptsize{7776}}};
\node at (-8.35-\sl,-2.22+\su) {\rlap{\scriptsize{1296}}};
\node at (-8.20-\sl,-2.52+\su) {\rlap{\scriptsize{216}}};
\node at (-8.05-\sl,-2.82+\su) {\rlap{\scriptsize{36}}};
\node at (-7.90-\sl,-3.12+\su) {\rlap{\scriptsize{6}}};
\node at (-7.90-\sl,-3.42+\su) {\rlap{\scriptsize{1}}};
\node at (-7.90-\sl,-3.72+\su) {\rlap{\scriptsize{0}}};

\node at (-6.85,-2.9) {\rlap{\LARGE{$\textbf{p}$}}};
\node at (-6,-1.7) {\rlap{\LARGE{$\textbf{d}$}}};
\node at (-0.72,-2) {\rlap{\LARGE{$\textbf{t}$}}};
\node at (4.3,-1.4) {\rlap{\LARGE{$^{\textbf{3}}$\textbf{He}}}};
\node at (1.1,1) {\rlap{\LARGE{$^{\textbf{4}}$\textbf{He}}}};

\end{tikzpicture}
\caption{\label{fig:5deg_SiDEvsSiE}A 2D histogram of the Si $\Delta$E vs Si E detector signals. Different groups correspond to different particles depending on their mass and charge. The $^{3}$He group, highlighted in red, is being gated on in the figure.}
\end{figure}


\def\yTS{0.686} % y tick space
\def\xTS{1.875} % x tick space
\def\sl{0.38} % shift left
\def\su{0.4} % shift up
\begin{figure}
\centering
\begin{tikzpicture}[scale=1.0, every node/.style={transform shape}]
\node at (0,0) {\includegraphics[width=5.5in]{Chapter-6/figs/5deg_SiDEvsSiTotE_GSiDEvsSiE.png}};
\node at (-8.1,-2.35) {\rlap{\includegraphics[scale=0.75]{Chapter-6/figs/9deg_Scale.png}}};
\node[rotate=90] at (-8.0,0) {Si $\Delta$E};
\node at (0,-3.95) {Si Total E};
\node at (-7.4,-2.9) {\rlap{\footnotesize{80}}};
\node at (-7.55,-2.9+\yTS) {\rlap{\footnotesize{100}}};
\node at (-7.55,{-2.9+(2*\yTS)}) {\rlap{\footnotesize{120}}};
\node at (-7.55,{-2.9+(3*\yTS)}) {\rlap{\footnotesize{140}}};
\node at (-7.55,{-2.9+(4*\yTS)}) {\rlap{\footnotesize{160}}};
\node at (-7.55,{-2.9+(5*\yTS)}) {\rlap{\footnotesize{180}}};
\node at (-7.55,{-2.9+(6*\yTS)}) {\rlap{\footnotesize{200}}};
\node at (-7.55,{-2.9+(7*\yTS)}) {\rlap{\footnotesize{220}}};
\node at (-7.55,{-2.9+(8*\yTS)}) {\rlap{\footnotesize{240}}};
\node at (-7.55,{-2.9+(9*\yTS)}) {\rlap{\footnotesize{260}}};
\node at (-6.63,-3.55) {\rlap{\footnotesize{140}}};
\node at (-6.63+\xTS,-3.55) {\rlap{\footnotesize{160}}};
\node at ({-6.63+(2*\xTS)},-3.55) {\rlap{\footnotesize{180}}};
\node at ({-6.63+(3*\xTS)},-3.55) {\rlap{\footnotesize{200}}};
\node at ({-6.63+(4*\xTS)},-3.55) {\rlap{\footnotesize{220}}};
\node at ({-6.63+(5*\xTS)},-3.55) {\rlap{\footnotesize{240}}};
\node at ({-6.63+(6*\xTS)},-3.55) {\rlap{\footnotesize{260}}};
\node at ({-6.63+(7*\xTS)},-3.55) {\rlap{\footnotesize{280}}};
\node at (-8.65-\sl,-1.35+\su) {\rlap{\scriptsize{390614}}};
\node at (-8.50-\sl,-1.62+\su) {\rlap{\scriptsize{78126}}};
\node at (-8.50-\sl,-1.92+\su) {\rlap{\scriptsize{15625}}};
\node at (-8.35-\sl,-2.22+\su) {\rlap{\scriptsize{3125}}};
\node at (-8.20-\sl,-2.52+\su) {\rlap{\scriptsize{625}}};
\node at (-8.20-\sl,-2.82+\su) {\rlap{\scriptsize{125}}};
\node at (-8.05-\sl,-3.12+\su) {\rlap{\scriptsize{25}}};
\node at (-7.90-\sl,-3.42+\su) {\rlap{\scriptsize{5}}};
\node at (-7.90-\sl,-3.72+\su) {\rlap{\scriptsize{1}}};
\node at (-7.90-\sl,-4.02+\su) {\rlap{\scriptsize{0}}};
\node at (4.0,0.1) {\LARGE{I}};
\node at (3.3,0.2) {\LARGE{K}};
\node at (1.15,0.38) {\LARGE{O}};
\node at (-3.27,1.75) {\LARGE{C}};
\end{tikzpicture}
\caption{\label{fig:5deg_SiDEvsSiTotE}A 2D histogram of the Si $\Delta$E vs Si total E detector signals, gated on the $^{3}$He group and at $\theta_{\mathrm{lab}} = 45^{\circ}$. This is similar to Fig. \ref{fig:5deg_SiDEvsSiE}, except the total Si energy has been calculated based on the slope of the peaks in the Si $\Delta$E vs Si E spectrum (see Eqn. \ref{eqn:Si_Slope}). Each peak is labeled with its corresponding elastic scattering element.}
\end{figure}

\newpage

\noindent
have an even more negative slope than the trend of the overall group. These different regions correspond to scattering from different elements in the target. From left to right, they are carbon, oxygen, potassium, and iodine, as the more massive elements will deposit more energy in the Si E detector. 

To obtain the $^{39}\mathrm{K}(^{3}\mathrm{He}, {}^{3}\mathrm{He})^{39}\mathrm{K}$ Si detector yield, the gated Si $\Delta$E vs Si E spectrum is first transformed. The regions corresponding to the individual scattering elements all have the same slope, and by rotating them so that they vertically align with the $y$-axis, they can be projected onto the $x$-axis to create a new 1D histogram with narrow peaks. This transformation is equivalent to finding the total energy deposited in the silicon detectors,
\begin{equation} \label{eqn:Si_Slope}
E^{\mathrm{tot}} = \frac{E + p_{\mathrm{Si}}\Delta E}{1 + p_{\mathrm{Si}}},
\end{equation}

\def\yTS{1.85} % y tick space
\def\xTS{1.858} % x tick space
\def\sl{0.38} % shift left
\def\su{0.4} % shift up
\begin{figure}[t]
\centering
\begin{tikzpicture}[scale=0.8, every node/.style={transform shape}]
\node at (0,0) {\includegraphics[width=5.5in]{Chapter-6/figs/5deg_SiTotE_GSiDEvsSiE.png}};
\node[rotate=90] at (-8.2,0) {Si Total E Counts};
\node at (0,-5.7) {Channel};
\node at (-7.75,-3.15) {\rlap{\footnotesize{2000}}};
\node at (-7.75,-3.15+\yTS) {\rlap{\footnotesize{4000}}};
\node at (-7.75,{-3.15+(2*\yTS)}) {\rlap{\footnotesize{6000}}};
\node at (-7.75,{-3.15+(3*\yTS)}) {\rlap{\footnotesize{8000}}};
\node at (-7.9,{-3.15+(4*\yTS)}) {\rlap{\footnotesize{10000}}};
\node at (-5.85,-5.28) {\rlap{\footnotesize{1400}}};
\node at (-5.85+\xTS,-5.28) {\rlap{\footnotesize{1500}}};
\node at ({-5.85+(2*\xTS)},-5.28) {\rlap{\footnotesize{1600}}};
\node at ({-5.85+(3*\xTS)},-5.28) {\rlap{\footnotesize{1700}}};
\node at ({-5.85+(4*\xTS)},-5.28) {\rlap{\footnotesize{1800}}};
\node at ({-5.85+(5*\xTS)},-5.28) {\rlap{\footnotesize{1900}}};
\node at ({-5.85+(6*\xTS)},-5.28) {\rlap{\footnotesize{2000}}};
\node at (6,2) {\LARGE{I}};
\node at (4,1) {\LARGE{K}};
\node at (1.65,-4) {\LARGE{O}};
\node at (-5.5,-4) {\LARGE{C}};
\end{tikzpicture}
\caption{\label{fig:5deg_SiTotE}A histogram spectrum of Si Total E gated on the $^{3}$He group at $\theta_{\mathrm{lab}} = 45^{\circ}$, obtained by projecting Fig. \ref{fig:5deg_SiDEvsSiTotE} onto its $x$-axis. Each peak is labeled with its corresponding elastic scattering element. The individual $^{39}$K and $^{41}$K isotopes are unresolvable.}
\end{figure}

\noindent
where $p_{\mathrm{Si}}$ is the Si slope parameter defined by the user in either \texttt{EngeSpec} or \texttt{Jam}. The transformed 2D Si $\Delta$E vs Si total E spectrum, gated on the $^{3}$He group, is shown in Fig. \ref{fig:5deg_SiDEvsSiTotE}. The scattering elements are labeled with their element symbols. Finally, the projected 1D histogram for Si total E is presented in Fig. \ref{fig:5deg_SiTotE}, where the potassium (K) peak can be seen on the shoulder of the iodine (I) peak. The yield of the potassium peak ideally remains constant throughout the experiment, for a given target, since the Si detector telescope is positioned at a constant $\theta_{\mathrm{lab}} = 45^{\circ}$. However, target degradation occurs naturally over time, decreasing the yield of both focal-plane and Si detector measurements. Hence, the normalization corrects for this, as well as the other unknown target properties. 

One complication is the poor resolution between different isotopes for elastic scattering, especially at low angles. In the Si detectors, the potassium peak always consisted of $^{41}$K in addition to $^{39}$K, since a natural KI target was used. In the focal-plane P1 detector, which has better energy resolution, the elastic scattering $^{39}$K peak could be resolved only for $\theta_{\mathrm{lab}} \geq 40^{\circ}$. The presence of $^{41}$K was corrected for in the differential cross-section calculations of $^{39}\mathrm{K}(^{3}\mathrm{He}, {}^{3}\mathrm{He})^{39}\mathrm{K}$ by taking into account both the isotopic abundance ratio of potassium ($93.26\%$ $^{39}$K, $6.73\%$ $^{41}$K) and the relative differential cross-sections of global $^{3}$He OMPs for $^{39}$K and $^{41}$K. The former contribution cancels out in the Si detector normalization for $\theta_{\mathrm{lab}} < 40^{\circ}$, but provides a $\sim 7\%$ increase in the differential cross section ratio for $\theta_{\mathrm{lab}} \geq 40^{\circ}$. The latter contribution is almost negligible due to the similar theoretical angular distributions of both differential cross sections over the angles of interest.

%, as detailed in Section \ref{subsec:SiNorm}.

\subsection{Oxidized Spectra} \label{subsec:oxidized_spectra}

% the first batch of targets (KI $\#$1, KI $\#$2, and KI $\#$3) oxidized at some point during the first day of the experiment, for the $\theta_{\mathrm{lab}} = 5^{\circ}$ and $7^{\circ}$ ($^{3}\mathrm{He},d$) measurements. The focal-plane spectra for these runs consisted of peaks with high-energy tails, expected of targets that undergo oxidation \cite{Landau}. These angles were measured again toward the end of the experiment with the non-oxidized KI $\#$6 target, but the oxidized KI $\#$1 data was not discarded nevertheless. Section \ref{subsec:oxidation} details a novel technique for fitting these spectra.

The KI $\#$1 target was the first one to be used in the $^{39}\mathrm{K}(^{3}\mathrm{He}, d)^{40}\mathrm{Ca}$ experiment at $\theta_{\mathrm{lab}} = 5^{\circ}$ and $7^{\circ}$. It was immediately clear that this target was producing peaks that were not gaussian in the focal-plane P1 spectra, as were the other targets in the first evaporation batch, KI $\#$2 and KI $\#$3. They were transported together and all placed in the target chamber simultaneously. At some point, they were presumably exposed to atmosphere suddenly or for an extended period of time, despite the care taken in transporting them to the target chamber in a sealed box at rough vacuum. The oxidized focal-plane P1 spectrum for all KI $\#$1 runs ($\#15-\#20$) at $\theta_{\mathrm{lab}} = 7^{\circ}$, gated on the deuteron group, are shown sorted together in Fig. \ref{fig:7deg_oxidized}. The low-channel (high-energy) side of each peak has a tail that is potentially smeared over other nearby peaks. This makes fitting peaks very challenging for yield measurements, when high resolution is required. However, Bayesian Monte Carlo techniques were implemented to obtain yield measurements with realistic uncertainties from these runs, as presented in Section \ref{subsec:oxidation}.

\def\yTS{0.69}
\def\xTS{1.945}

\begin{figure}[t]
\centering
\begin{tikzpicture}[scale=1.0, every node/.style={transform shape}]
\node[rotate=90] at (-8.6,0) {P1 Counts};
\node at (0,-3.6) {Channel};
\node at (0,0) {\includegraphics[width=6in]{Chapter-6/figs/7deg_oxidized.png}};
\node at (-7.90,-3.0) {\rlap{\footnotesize{0}}};
\node at (-8.05,-3.0+\yTS) {\rlap{\footnotesize{25}}};
\node at (-8.05,{-3.0+(2*\yTS)}) {\rlap{\footnotesize{50}}};
\node at (-8.05,{-3.0+(3*\yTS)}) {\rlap{\footnotesize{75}}};
\node at (-8.20,{-3.0+(4*\yTS)}) {\rlap{\footnotesize{100}}};
\node at (-8.20,{-3.0+(5*\yTS)}) {\rlap{\footnotesize{125}}};
\node at (-8.20,{-3.0+(6*\yTS)}) {\rlap{\footnotesize{150}}};
\node at (-8.20,{-3.0+(7*\yTS)}) {\rlap{\footnotesize{175}}};
\node at (-8.20,{-3.0+(8*\yTS)}) {\rlap{\footnotesize{200}}};

\node at (-7.32,-3.18) {\rlap{\footnotesize{1250}}};
\node at (-7.32+\xTS,-3.18) {\rlap{\footnotesize{1500}}};
\node at ({-7.32+(2*\xTS)},-3.18) {\rlap{\footnotesize{1750}}};
\node at ({-7.32+(3*\xTS)},-3.18) {\rlap{\footnotesize{2000}}};
\node at ({-7.32+(4*\xTS)},-3.18) {\rlap{\footnotesize{2250}}};
\node at ({-7.32+(5*\xTS)},-3.18) {\rlap{\footnotesize{2500}}};
\node at ({-7.32+(6*\xTS)},-3.18) {\rlap{\footnotesize{2750}}};
\node at ({-7.32+(7*\xTS)},-3.18) {\rlap{\footnotesize{3000}}};

\end{tikzpicture}
\caption{\label{fig:7deg_oxidized}A histogram spectrum of the focal-plane P1  section at $\theta_{\mathrm{lab}} = 7^{\circ}$ with the KI $\#$1 target, gated on the deuteron group. The peaks are clearly not gaussian because of the presence of high-energy (low-channel) tails, a result of oxidation in the KI $\#$1 target.}
\end{figure}

%%%%%%%%%%%%%%%%%%%%%%%%%%%%%%%%%%%%%%%%%%%%%%%%%%%%%%%%%%%%%%%%%%%%%%%%%%%%%%%%%%%%%%%%%%%%%
\section{Bayesian Peak Fitting and Target Oxidation} \label{sec:peak_fitting}

% 1 paragraph overview first here...
The energy loss of light, charged particles through a medium is a statistical process. Particles with the same initial energy will traverse the same length in the medium with a distribution of final energies. This phenomenon is known as energy straggling and was theoretically described by Landau \cite{Landau1944}. Thin-film targets are ideal for high resolution spectral analysis because the energy loss distribution is approximately gaussian. Thick targets, however, alter the energy loss distribution in ways that present challenges to fitting peaks in a spectrum. Section \ref{subsec:peak_fitting_gaus} presents the Bayesian peak fitting procedure implemented in the analysis of the $^{39}\mathrm{K}(^{3}\mathrm{He}, d)^{40}\mathrm{Ca}$ experiment for the case of thin targets, and Section \ref{subsec:oxidation} presents a modified procedure for the case of thick targets, resulting from target oxidation which produces tails in the energy loss distribution.

\subsection{Bayesian Peak Fitting} \label{subsec:peak_fitting_gaus}

Transfer reaction analysis requires precise experimental cross section measurements for each relevant excited state of the residual nucleus. Each cross section is proportional to the yield of the transfer reaction populating the given excited state. An essential ingredient in the yield measurement, described in more detail in Section \ref{sec:cs_calc}, is the number of counts of ejectile particles measured by the focal-plane detector. Each count contributes to a peak along the focal-plane spectrum corresponding to the given excited state with a finite width. The peaks are distinguished by their excitation energies, which are determined through an energy calibration based on their relative centroid positions along the focal-plane, as is done in Section \ref{subsec:cal}. A typical transfer reaction spectrum will include many such peaks, as well as peaks from reactions in the target not necessarily of interest, known as contaminants. As mentioned in the introduction of this section, excited state peaks are typically gaussian-distributed, assuming a thin-film target is used. The spectrum will also include background counts, represented by a straight line. A precise determination of the number of counts in a given peak, in the simplest case, is therefore done by fitting a gaussian with a background line.

The target for the $^{39}\mathrm{K}(^{3}\mathrm{He}, d)^{40}\mathrm{Ca}$ experiment was natural potassium iodine on a natural carbon film backing. The most prominent contaminants were from $(^{3}\mathrm{He}, d)$ reactions on $^{12}$C, $^{13}$C, $^{14}$N, and $^{16}$O, producing excited states from the residual nuclei $^{13}$N, $^{14}$N, $^{15}$O, and $^{17}$F, respectively. The chosen magnetic field of the Enge Split-Pole Spectrograph was such that deuterons from $(^{3}\mathrm{He}, d)$ reactions with iodine were not at all present along the focal-plane, and deuterons from $^{41}\mathrm{K}(^{3}\mathrm{He}, d)$ were only minimally present. However, given the presence of the many other contaminants, and to distinguish doublets, triplets, and other multiplets, it was deemed necessary to use a more sophisticated technique to fit the peaks from this experiment than a simple chi-squared minimization. Bayesian Monte Carlo sampling was a natural choice, considering its realistic uncertainty handling and flexibility when dealing with non-gaussian fits, presented in Section \ref{subsec:oxidation}. I used the \texttt{BayeSpec} \cite{BayeSpec} graphical user interface (GUI), along with custom Bayesian sampling routines described below, to acquire fits to the $^{39}\mathrm{K}(^{3}\mathrm{He}, d)^{40}\mathrm{Ca}$ data.

% BayeSpec GUI, MCMCGaus
Most peaks from transfer reactions on a thin-film target are gaussian-distributed,
\begin{equation}
    f(x;A,\mu,\sigma) = A \, \exp \Big( -\frac{(x-\mu)^{2}}{2\sigma^{2}} \Big),
\end{equation}
where $A$ is the peak intensity, $\mu$ is its mean, and $\sigma$ is its standard deviation.
%A chi-squared minimization would optimize these 3 parameters to fit the data. Alternatively, 
Bayesian sampling can be used to optimize these 3 parameters to match the focal-plane data if appropriate prior distributions are known. In this case, each prior distribution is approximately normal, $\mathcal{N}(\mu_{0},\sigma_{0}^{2})$, with its own mean $\mu_{0}$ and standard deviation $\sigma_{0}$. Posterior distributions can then be computed from these priors to achieve parameter values with statistically realistic uncertainties that match the data. These more realistic uncertainties make it possible to handle difficult fitting scenarios that would normally cause problems for chi-square minimizations, such as distinguishing peaks in a multiplet with appropriate uncertainties.

Prior means $\mu_{0}$ for $A$ and $\mu$ are particularly simple to obtain if the peak apex is visible. Even if it is not visible, as in the case of a multiplet, a guess can be made for each peak apex location, $(x,y)$, based on the curvature of their sum, where $x$ is the mean guess and $y$ is the intensity guess. Each guess is established by selecting a point using the \texttt{BayeSpec} GUI. The prior standard deviations $\sigma_{0}$ for $A$ and $\mu$ are given constant values conservative enough for sampling to be successful in even the most uncertain cases. These can be adjusted by the user, but are typically left as their defaults, listed below.

The prior mean $\mu_{0}$ and standard deviation $\sigma_{0}$ for $\sigma$ are ideally given constant values for a given reaction representative of most peaks from that reaction in focal-plane spectra. This is because energy loss is nearly equivalent for peaks with similar excitation energies from the same reaction, but it is not equivalent for peaks from different reactions. Therefore, peaks from each reaction should ideally have a shared $\sigma$ prior and a shared $\sigma$ posterior in a single fit if multiple peaks are fit simultaneously. In practice, a single $\sigma$ prior can be used for different reactions if it is conservative enough for sampling to be successful, but the $\sigma$ posteriors should remain different for different reactions. 

A background line must also be included in the model, where its intensity $y_{\mathrm{bg}}$ and slope $m_{\mathrm{bg}}$ have their own $\mathcal{N}(\mu_{0}, \sigma_{0}^{2})$ priors. Put together, the prior distributions for gaussian peaks on a background line in a typical focal-plane spectrum are
\begin{align}
    A_{i} &\sim \mathcal{N}(y_{i}, 10.0^{2}) \nonumber \\
    \mu_{i} &\sim \mathcal{N}(x_{i}, 1.0^{2}) \nonumber \\
    \sigma &\sim \mathcal{N}(5.0, 1.0^{2}) \nonumber \\
    y_{\mathrm{bg}} &\sim \mathcal{N}(\mathrm{max}(1, \mathrm{min}(d)), 0.1^{2}) \nonumber \\
    m_{\mathrm{bg}} &\sim \mathcal{N}(0, 0.01), \label{eqn:gaus_priors}
\end{align}
where $\sigma$ is a constant prior for all reactions, but it can easily be adjusted by the user for different reactions if needed, and $d$ refers to the focal-plane data in the specified fit range. The prior for $m_{\mathrm{bg}}$ can also be adjusted if the background line is clearly not horizontal. The model function to be fit is
\begin{equation}
    f(x;A,\mu,\sigma,y_{\mathrm{bg}},m_{\mathrm{bg}}) = \sum_{i}^{N} A_{i} \, \exp \Big( -\frac{(x-\mu_{i})^{2}}{2\sigma_{\hspace{-0.08cm} j}^{2}} \Big) + \Big(y_{\mathrm{bg}} + m_{\mathrm{bg}} (x - \mathrm{min}(x))\Big), \label{eqn:nGausBkg}
\end{equation}
where $N$ is the total number of gaussian peaks to be fit and $\sigma_{\hspace{-0.08cm} j}$ refers to the shared standard deviation for peaks from reaction $j$.

The posterior computation in the custom Bayesian sampling routine uses the \texttt{quap} function in \texttt{R}, part of the Bayesian \texttt{rethinking} package \cite{Rethinking}. This function finds a quadratic approximation to each full posterior distribution at its mode. It is less sophisticated than Markov Chain Monte Carlo, but its relative simplicity makes it a more efficient option to use with a GUI. The model composed of Eqns. \ref{eqn:gaus_priors} and \ref{eqn:nGausBkg} is provided to \texttt{quap}, which returns the posterior quadratic approximations with built-in uncertainties. Samples from these posteriors are then used to compute the area of each gaussian, given by $\sqrt{2\pi} \, \sigma_{\hspace{-0.08cm} j} A_{i}$, along with their uncertainties. A small number of posterior samples are also used to graphically display the gaussian fits, with different colors representing different reactions.

\begin{figure}[t]
\centering
\includegraphics[width=6.5in]{Chapter-6/figs/Gaus_Bayesian.png}
\caption{\label{fig:Gaus_Bayesian}A Bayesian multi-gaussian fit with \texttt{BayeSpec} for the $^{40}$Ca excited states (in red) 6025 keV and 5903 keV from $^{39}\mathrm{K}(^{3}\mathrm{He},d)^{40}\mathrm{Ca}$ and the $^{14}$N excited states (in orange) 5106 keV and 4915 keV from $^{13}\mathrm{C}(^{3}\mathrm{He},d)^{14}\mathrm{N}$ at $\theta_{\mathrm{lab}} = 5^{\circ}$. In red and orange are 50 random samples of the gaussian distributions from the $\sigma$, $\mu$, and $A$ posteriors for each peak. The $^{40}$Ca peaks share an identical $\sigma$ posterior, while the $^{14}$N peaks share their own as well. In blue are the sums of the peaks plus the background line for each of those 50 samples.}
\end{figure}

Figure \ref{fig:Gaus_Bayesian} shows an example of this Bayesian fitting procedure for a $^{39}\mathrm{K}(^{3}\mathrm{He},d)^{40}\mathrm{Ca}$ focal-plane spectrum at $\theta_{\mathrm{lab}} = 5^{\circ}$. Excited states of $^{40}\mathrm{Ca}$ are shown in red and $^{14}\mathrm{N}$ excited state contaminants are shown in orange. In blue is the sum of the gaussians and the background line, given by Eqn. \ref{eqn:nGausBkg}. Each gaussian, as well as the blue curve, consists of 50 individual curves that were sampled from the posteriors to represent the uncertainties. The reason it is useful to combine these 4 peaks into one fit in this example is to better resolve the 5903 keV $^{40}$Ca state in red from the 4915 keV $^{14}$N state in orange in the double gaussian on the right side of the spectrum. The addition of the 6025 keV $^{40}$Ca peak in red and the 5106 keV $^{14}$N peak in orange on the left constrains the $\sigma$ posteriors for the double gaussian. While this may seem like a minor correction, and it is minor in the present example, it is crucial in situations where the double gaussian or multiplet is otherwise impossible to resolve.

%%*****************************************************************************************%%
\subsection{Fitting Peaks with Target Oxidation} \label{subsec:oxidation}

Potassium iodine is hygroscopic, meaning it easily absorbs moisture from the environment. When exposed to atmospheric pressure, the salt slowly oxidizes, forming potassium carbonate and molecular iodine. This oxidation increases the thickness of thin-film targets. The probability of particles traversing the target with large final energies is decreased, introducing a high energy tail in the ejected particle spectrum. The Landau distribution describes this energy loss, but it is computationally challenging to implement. An approximation that has been found to fit oxidized-target spectra well is the exponentially-modified gaussian (EMG) distribution \cite{Babu2016}. The power of the EMG approximation is that its area calculation is the exact same as that of a gaussian distribution, $\sqrt{2\pi} \, \sigma A$, where $\sigma$ and $A$ are parameters of both gaussians and EMGs, but they have slightly different definitions as it will become clear below. This makes it very simple to determine the number of counts for EMG distributions, unlike Landau distributions, where the full area integral must be calculated.

% What to do when the target is oxidized and therefore the peaks have high energy tails. This is called energy straggling and is physically described by the Landau distribution. An approximation to this is an exponentially-modified gaussian 
% Cite: [Measurement of Energy Loss Straggling of Relativistic Electrons in Thin Aluminum Foils, S. Ramesh Babu and N.M. Badiger, ACTA Physica Polonica A. Vol. 129 (2016) DOI: 10.12693/AphysPolA.129.1118  | Page 1119 Section 2.2: Straggling theories of thin absorbers]

The probability density function (PDF) of an EMG distribution $f$ is a convolution of exponential $g$ and gaussian $h$ PDFs,
\begin{align}
    f(x;\sigma,\lambda,\mu,A) = (g*h)(x) &= \int_{-\infty}^{\infty}g(x')h(x-x') \, dx' \\
             &= \int_{0}^{\infty}\lambda \exp\big(-\lambda x'\big) \, A\exp\Big(-\frac{1}{2}\Big(\frac{x-x'-\mu}{\sigma}\Big)^{2}\Big) \, dx' \nonumber \\
             &= A \, \sigma\lambda\sqrt{\frac{\pi}{2}}\exp\Big(\frac{1}{2}(\sigma\lambda)^{2} - (x-\mu)\lambda\Big) \, \mathrm{erfc}\Big(\frac{1}{\sqrt{2}}\Big(\sigma\lambda - \frac{x-\mu}{\sigma}\Big)\Big), \nonumber
\end{align}
where $\lambda$ is the exponential-component rate, $\mu$, $\sigma$, and $A$ are the gaussian-component mean, standard deviation, and peak intensity, respectively, and $\mathrm{erfc}$ is the complimentary error function, defined as $\mathrm{erfc} \, z = 1 - \mathrm{erf} \, z$ for the complex variable $z$, where $\mathrm{erf}$ is the error function,
\begin{equation}
    \mathrm{erf} \, z = \frac{2}{\sqrt{\pi}}\int_{0}^{z}\exp\big(-t^2\big) \, d t.
\end{equation}
The standard EMG distribution, as a function of $x$, has a high--$x$ tail. Since the focal-plane position spectrum channel number is inversely related to energy, the EMG must be modified to have a low--$x$ tail. This is done by simply replacing $x-\mu$ with $\mu-x$ to reflect the standard EMG distribution about its gaussian-component mean, $\mu$. The following discussion of EMG distributions assumes this reflection has been performed.

Previously it was shown for non-oxidized targets that the Bayesian fitting procedure for a gaussian peak requires gaussian mean $\mu$ and peak intensity $A$ priors. The means of these priors are provided by the user when selecting an estimate at the peak apex with the \texttt{BayeSpec} GUI. However, this presents a problem when extending the procedure to EMG fits. An EMG is defined by the $\mu$ and $A$ parameters of its gaussian component, not the mean and peak intensity of the EMG itself. To obtain reasonable priors for the $\mu$ and $A$ parameters, they must be derived from the attributes of the EMG peak apex, where the user selects. This apex defines the mode $x_{m}$ and peak intensity $y_{m}$ of the EMG, which can both be calculated by determining the coordinates where the derivative of the PDF is equal to zero. These are
%The simplest attributes of an EMG from the perspective of the user are the mode $x_{m}$ and peak intensity $y_{m}$, which can also be calculated by determining the coordinates where the derivative of the PDF is equal to zero. These are
\begin{align}
    x_{m} &= \mu - \sigma^{2}\lambda + \sqrt{2} \, \sigma \, z, \nonumber \\
    y_{m} &= A \, \exp \Big( -\frac{1}{2} \Big( \frac{\mu - x_{m}}{\sigma} \Big)^{2} \Big), \label{eqn:mode_peak_intensity}
\end{align}
where $z$ is defined such that
\begin{equation}
    \exp(z^{2}) \, \mathrm{erfc}(z) = \sqrt{\frac{2}{\pi}} \, \frac{1}{\sigma\lambda},
\end{equation}
which can be solved numerically. With $x_{m}$ and $y_{m}$ provided by the user, $\mu$ and $A$ become
\begin{align}
    \mu(\sigma, \lambda) &= x_{m} + \sigma^{2} \lambda - \sqrt{2} \sigma z, \label{eqn:mu_before} \\
    A(\sigma, \lambda) &= y_{m} \exp \Big( \frac{1}{2} \Big( \frac{\mu-x_{m}}{\sigma} \Big)^{2} \Big). \label{eqn:A_before}
\end{align}
The priors would be fully determined if $\sigma$ and $\lambda$, the parameters related to the EMG width and skewness, can be constrained. Fortunately, for a given transfer reaction observed in a focal-plane spectrum, there is little variation between these properties, much like the gaussian $\sigma$ parameter of Section \ref{subsec:peak_fitting_gaus}. From a sample of EMG fits for $^{39}\mathrm{K}(^{3}\mathrm{He},d)^{40}\mathrm{Ca}$ peaks, reasonable priors for $\sigma$ and $\lambda$ were determined to be
\begin{align} 
    \sigma &\sim \mathcal{N}(5.0, \, 1.0^{2}), \label{eqn:sig_prior} \\
    \lambda &\sim \mathcal{N}(0.09, \, 0.02^{2}), \label{eqn:lambda_prior}
\end{align} 
where these are also usually sufficient for other reactions. Priors for Eqns. \ref{eqn:mu_before} and \ref{eqn:A_before} can then be constructed by sampling from Eqns. \ref{eqn:sig_prior} and \ref{eqn:lambda_prior}. The resulting prior distributions for $\mu - x_{m}$ and $A / y_{m}$ are shown in Figure \ref{fig:mu_and_A} in black after taking 10,000 samples from Eqns. \ref{eqn:sig_prior} and \ref{eqn:lambda_prior}. The gaussian approximations (blue) were constructed from the mean and standard deviation of the $\mu - x_{m}$ and $A/y_{m}$ samples, while the lognormal approximations (red) were similarly constructed from the mean and standard deviation of the natural logarithm of those samples. Note that the priors are not gaussian. In fact, the prior for $A / y_{m}$ is neither gaussian nor lognormal. However, a lognormal approximation suffices for both in practice and is preferable to the actual distributions, which are either non-analytical or at least exceedingly complicated. The Bayesian quadratic approximation function, \texttt{quap}, used with \texttt{BayeSpec} requires named prior distributions, forbidding the use of non-analytical distributions. The lognormal approximations also correctly prevent negative sample values and are therefore taken as the priors for $\mu$ and $A$.

\begin{figure}[t]
\centering
\includegraphics[width=6.5in]{Chapter-6/figs/mu_and_A.png}
\caption{\label{fig:mu_and_A}The constructed priors for $\mu - x_{m}$ (left) and $A/y_{m}$ (right) from 10,000 samples of the $\sigma$ and $\lambda$ priors of Eqns. \ref{eqn:sig_prior} and \ref{eqn:lambda_prior}. The gaussian (blue) approximations were derived from the mean and standard deviation of the samples. The lognormal (red) approximations were derived from the mean and standard deviation of the natural logarithm of those samples.}
\end{figure}

\begin{figure}[t]
\centering
\includegraphics[width=6.5in]{Chapter-6/figs/ExpModGauss_Mode_and_Intensity.png}
\caption{\label{fig:EMG_Mode}A Bayesian exponentially-modified gaussian fit with \texttt{BayeSpec} for the 6285 keV $^{40}\mathrm{Ca}$ state from $^{39}\mathrm{K}(^{3}\mathrm{He},d)^{40}\mathrm{Ca}$ at $\theta_{\mathrm{lab}} = 5^{\circ}$ with an oxidized potassium iodine target (KI $\#$1). In blue are 50 random samples of the exponentially-modified gaussian distribution from the $\sigma$, $\lambda$, $\mu$, and $A$ posteriors, plus the background line. In green are the gaussian components of the exponentially-modified gaussian samples. The new mode and peak intensity are represented by the black lines, where the solid line represents their mean and the dashed lines represent their standard deviation.}
%The mode and peak intensity means of the 10,000 samples are shown by solid black lines, while their standard deviations are represented by dashed lines.
\end{figure}

As in Section \ref{subsec:peak_fitting_gaus}, once the priors have been determined, the quadratic approximations to the posteriors are computed with \texttt{quap}, and the fit is constructed from samples of the EMG distribution with those posteriors. An example of this Bayesian EMG fitting method is shown in Figure \ref{fig:EMG_Mode}, where the 6285 keV $^{40}$Ca state from $^{39}\mathrm{K}(^{3}\mathrm{He},d)^{40}\mathrm{Ca}$ at $\theta_{\mathrm{lab}} = 5^{\circ}$ is fit with an EMG distribution because the potassium iodine target for this run (KI $\#$1) had undergone oxidation. %The $\mu$ and $A$ priors of Eqns. \ref{eqn:mu_before} and \ref{eqn:A_before} were obtained by taking 10,000 samples from the $\sigma$ and $\lambda$ priors of Eqns. \ref{eqn:sig_prior} and \ref{eqn:lambda_prior} and approximating this as a lognormal distribution. 
The closely-packed blue lines are 50 representative samples of the EMG posteriors, plus that of the background line, while the green lines are the corresponding gaussian components of those EMG samples with mean $\mu$, standard deviation $\sigma$, and peak intensity $A$. The new mode and peak intensity from Eqn \ref{eqn:mode_peak_intensity}, obtained by sampling from the posteriors, are shown by the black lines, where the solid line represents its mean and the dashed lines represent its standard deviation. Note that the ($x_{m}$, $y_{m}$) coordinate lies along its gaussian component, a ubiquitous property of EMG distributions. The number of counts in the EMG distribution is simply equivalent to the area of its gaussian component, $\sqrt{2\pi} \, \sigma A$, constructed by sampling from the $\sigma$ and $A$ posteriors.

Figure \ref{fig:EMG_Multiplet} shows how powerful this method can be for high resolution spectral analysis with an oxidized target. The same focal-plane spectrum as Figure \ref{fig:EMG_Mode} is shown here, but it is focused on a region with multiple peaks. The blue summed fit consists of 5 EMG distributions, shown individually by the red and orange lines, and a small, virtually horizontal background line. As before, each distribution shows 50 lines drawn from samples of the EMG posteriors. From a simple energy calibration, the red peaks were found to be $^{40}$Ca excited states with excitation energies 7694 keV, 7658 keV, 7623 keV, and 7532 keV, from left to right, and the orange peak corresponds to the $^{14}$N 6446 keV excited state. Because we expect energy loss to be nearly equivalent for peaks with similar energies from the same reaction, the $\sigma$ and $\lambda$ width and skewness parameter posteriors are fixed between the $^{40}$Ca peaks, while the $^{14}$N peak has its own $\sigma$ and $\lambda$ posteriors. The custom Bayesian sampling routines for both gaussian and EMG fits can be used for a simultaneous fit of any number of peaks from up to 3 different reactions at present, and this could easily be extended to any number of reactions when the need arises.

\begin{figure}
\centering
\includegraphics[width=6.5in]{Chapter-6/figs/EMG_Multiplet.png}
\caption{\label{fig:EMG_Multiplet}A Bayesian exponentially-modified gaussian fit with \texttt{BayeSpec} for the $^{40}$Ca excited states (left to right, in red) 7694 keV, 7658 keV, 7623 keV, and 7532 keV from $^{39}\mathrm{K}(^{3}\mathrm{He},d)^{40}\mathrm{Ca}$ and the $^{14}$N excited state (in orange) 6446 keV from $^{13}\mathrm{C}(^{3}\mathrm{He},d)^{14}\mathrm{N}$ at $\theta_{\mathrm{lab}} = 5^{\circ}$ with an oxidized potassium iodine target (KI $\#$1). In red and orange are 50 random samples of the exponentially-modified gaussian distributions from the $\sigma$, $\lambda$, $\mu$, and $A$ posteriors for each peak. The $^{40}$Ca peaks share identical $\sigma$ and $\lambda$ posteriors, whereas the $^{14}$N peak has its own. In blue are the sums of the peaks plus the background line for each of those 50 samples.}
\end{figure}

\newpage

%Since $x_{m}$ and $y_{m}$ are constant, they can be added and multiplied, respectively, to the resulting normal distributions from the following properties. Consider a normally distributed random variable $X$ with mean $a$ and variance $b^2$. Then the properties
%\begin{align}
%    X + c &\sim \mathcal{N}(a+c,b^{2}), \label{norm_prop1} \\
%    cX &\sim \mathcal{N}(ca,c^{2}b^{2}), \label{norm_prop2}
%\end{align}
%follow from the PDF of the normal distribution for a constant $c$. 

%Taking 10,000 samples from Eqns. \ref{sig_prior} and \ref{lambda_prior}, and using Eqns. \ref{norm_prop1} and \ref{norm_prop2}, Eqns. \ref{mu_before} and \ref{A_before} become
%\begin{align}
%    \mu &\sim \mathcal{N}(x_{m} + 5.37, \, 0.862^{2}) \\
%    A &\sim \mathcal{N}(1.87 y_{m}, \, 0.362^{2} y^{2}_{m}).
%\end{align}

%\begin{equation}
%    z = \mathrm{erfcxinv}\Bigg(\sqrt{\frac{2}{\pi}} \, \frac{1}{\sigma\lambda}\Bigg).
%\end{equation}
%The function $\mathrm{erfcxinv}$ is the inverse of the scaled complementary error function $\mathrm{erfcx}$, defined as
%\begin{equation}
%    \mathrm{erfcx}(z) = \exp(z^{2}) \, \mathrm{erfc}(z).
%\end{equation}
%In other words, $z$ is obtained by sol

%%*****************************************************************************************%%

%%%%%%%%%%%%%%%%%%%%%%%%%%%%%%%%%%%%%%%%%%%%%%%%%%%%%%%%%%%%%%%%%%%%%%%%%%%%%%%%%%%%%%%%%%%%%
\section{Cross Sections} \label{sec:cs_calc}

% Short introduction here...
An essential ingredient in the calculation of thermonuclear reaction rates is the nuclear reaction cross section $\sigma$, introduced in Section \ref{sec:rates}. The reaction cross section is an imaginary, effective area around the target nucleus where the reaction proceeds with a probability of unity if an incident particle passes through it. It is therefore a measure of the probability for a reaction to occur, varying with the incident particle energy. The cross section is closely related to the yield $Y$ of the reaction
\begin{equation}
Y = \frac{N_{\mathrm{R}}}{N_{\mathrm{b}}},
\end{equation}
where $N_{\mathrm{R}}$ is the total number of reactions and $N_{\mathrm{b}}$ is the total number of incident particles. The total number of reactions $N_{\mathrm{R}}$ that occur is related to the number of counts $N_{\mathrm{c}}$, or the area, of a given peak in a detector. However, this peak count is affected by the deadtime $t_{\mathrm{dead}}$ of the data acquisition system. To correct for the deadtime, and therefore get an accurate measure of the number of reactions, the livetime $t_{\mathrm{live}} = 1 - t_{\mathrm{dead}}$ is used as in
\begin{equation}
N_{\mathrm{R}} = \frac{N_{\mathrm{c}}}{t_{\mathrm{live}}}.
\end{equation}
The total number of incident particles $N_{\mathrm{b}}$ is
\begin{equation}
N_{\mathrm{b}} = \frac{q}{Z_{p} \, e},
\end{equation}
where $q$ is the total charge deposited by the incident particles, $Z_{p}$ is the unit charge of the incident particles, and $e$ is the electronic charge (1.6 $\times$ $10^{-19}$ C). The total charge $q$ is derived from the beam-integrated current (BCI) scalar value collected for each pulse during a given run. The BCI has an associated $\mathrm{BCI}_{\mathrm{scale}}$ setting that determines the scale of the deposited charge, which can be adjusted on the beam current readout module. For the $^{39}\mathrm{K}(^{3}\mathrm{He},d)^{40}\mathrm{Ca}$ experiment, the $\mathrm{BCI}_{\mathrm{scale}}$ setting was $10^{-10}$ C/pulse. The number of incident particles $N_{\mathrm{b}}$ is therefore
\begin{equation}
N_{\mathrm{b}} = \frac{\mathrm{BCI} \times \mathrm{BCI_{\mathrm{scale}}}}{Z_{p} \, e}.
\end{equation}
As described in Section \ref{subsec:cs_calc}, the conversion between the yield and the cross section depends on the target thickness $\Delta$E (in energy units) and the stopping power $\mathcal{E} = -(1/N_{\mathrm{t}})dE/dx$ (in units of eV $\mathrm{cm}^{2} / \mathrm{atom}$), where $N_{\mathrm{t}}$ denotes the number density of target nuclei.

A measurement of the total cross section or total yield would require the full $4\pi$ sr solid angle coverage of detectors around the target to ensure that no emitted particles from the reaction are missed. Since this is usually not feasible, the alternative is the measurement of a differential cross section $d\sigma/d\Omega_{\theta}$ or differential yield $dY/d\Omega_{\theta}$ from a detector covering a solid angle $\Omega$ collecting the particles ejected at an angle $\theta$ from the incident beam. The differential cross section can be measured at several angles to determine the angular distribution of the reaction. In the case of charged-particle reactions, such as $^{39}\mathrm{K}(^{3}\mathrm{He},d)^{40}\mathrm{Ca}$, each excited state of the recoil nucleus $^{40}$Ca has its own angular distribution that depends theoretically on the single-particle quantum numbers associated with the quantum mechanical selection rules for the transition, presented in Section \ref{subsec:DWBA_Spec}.

\subsection{Yield to Cross Section Conversion} \label{subsec:cs_calc}

For thin targets, where the cross section $\sigma$ and stopping power $\mathcal{E}$ are approximately constant over the target thickness $\Delta$E, the yield $Y$ is proportional to $\sigma$, with a proportionality constant $n_{\mathrm{X}} = \Delta \mathrm{E} / \mathcal{E}$ representing the number of active target nuclei per $\mathrm{cm}^{2}$ in the target compound $\mathrm{X}_{a}\mathrm{Y}_{b}$ \cite{Iliadis2015}. The term \emph{active nuclei} in this sense means the nuclei of interest $\mathrm{X}$ in the target, whereas the term \emph{inactive nuclei} refers to the nuclei not of interest $\mathrm{Y}$ in the target. The number of inactive target nuclei per unit area is defined as $n_{\mathrm{Y}}$, and the ratio $n_{\mathrm{Y}}/n_{\mathrm{X}}$ is equivalent to the target molecule stoichiometry, $b/a$. For compound targets, the effective stopping power $\mathcal{E}_{\mathrm{eff}} = \mathcal{E}_{\mathrm{X}} + n_{\mathrm{Y}}\mathcal{E}_{\mathrm{Y}} / n_{\mathrm{X}}$ must be used. The same proportionality constant 
\begin{equation}
n_{\mathrm{X}} = \frac{\Delta E}{\mathcal{E}_{\mathrm{eff}}}
\end{equation}
also applies to the differential yield and differential cross section. Additionally, as some of the energy of the incident particles $E_{0}$ is lost in the target, the cross section is evaluated at an effective, reduced energy $E_{\mathrm{eff}} = E_{0} - \Delta\mathrm{E}/2$. The conversion between the differential yield and differential cross section therefore becomes
\begin{equation} \label{eqn:yield_to_cross}
\left[ \frac{dY(E_{0})}{d\Omega} \right]_{\theta} = n_{\mathrm{X}} \, \left[ \frac{d\sigma(E_{\mathrm{eff}})}{d\Omega} \right]_{\theta}.
\end{equation}

The number of active nuclei per $\mathrm{cm}^{2}$, $n_{\mathrm{X}}$, can also be expressed in a form more tractable for experiments in the lab system as \cite{Rolfs1988,SetoodehniaThesis}\footnote{The ``yield'' $Y$ defined in Eqns. 3.30 and 3.31 of Ref. \cite{SetoodehniaThesis} is $N_{\mathrm{R}}$, not $N_{\mathrm{R}}/N_{\mathrm{b}}$, even though it describes it as such in the text.}
\begin{equation}
n_{\mathrm{X}} = \frac{\nu N_{A} \Delta x}{A_{t, \, \mathrm{mol}}} \,\,\, [\mathrm{in \,\, cm}^{-2}],
\end{equation}
where $\nu$ is the number of $\mathrm{X}$ atoms per molecule in the target material, $N_{A}$ is Avogadro's number (6.023 $\times$ $10^{23}$ $\mathrm{atoms}/\mathrm{mole}$), $\Delta x$ is the thickness of the target (in g/$\mathrm{cm}^{2}$), and $A_{t, \, \mathrm{mol}}$ is the molecular mass of the target material (in grams).

Combining everything, the differential cross section in units of mb/sr, where 1 b $ = 10^{-24}$ $\mathrm{cm}^{-2}$, measured in the lab system is calculated as
\begin{equation}
\left[ \frac{d\sigma}{d\Omega} \right]_{\theta} = \frac{N_{c} \, Z_{p} \, A_{t, \, \mathrm{mol}}}{(3.75 \times 10^{9}) \, \Omega \, t_{\mathrm{live}} \, \mathrm{BCI} \times \mathrm{BCI}_{\mathrm{scale}} \, \nu \, \Delta x} \, \, \, [\mathrm{in} \, \, \mathrm{mb}/\mathrm{sr}],
\end{equation}
where the solid angle $\Omega$ is in units of msr, all other variables are defined as before, and the factor $e/N_{A}$ has been evaluated and converged into the numerical factor in the denominator.

\subsection{Rutherford Scattering} \label{subsec:Ruth}

For elastic scattering cross section measurements, it is typical to report the cross section as a ratio to the theoretical Rutherford scattering cross section at the same angle. Theoretical elastic scattering cross sections from optical model potentials (OMPs) are also typically given in terms of the Rutherford ratio, depending on the nuclear reaction code. In the coupled-reaction channels code Fresco \cite{Fresco} for example, the Rutherford ratio is the default output of the elastic scattering cross section. 

In the center-of-mass system, the Rutherford scattering cross section is \cite{Iliadis2015}
\begin{align}
\left[ \frac{d\sigma}{d\Omega} \right]^{\mathrm{Ruth}}_{\theta} &= \left( \frac{Z_{p}Z_{t}e^{2}}{4E_{\mathrm{c.m.}}} \right)^{2} \frac{1}{\sin^{4}(\theta_{\mathrm{c.m.}}/2)} \nonumber \\
&= 1.296 \left( \frac{Z_{p}Z_{t}}{E_{\mathrm{c.m.}}} \right)^{2} \frac{1}{\sin^{4}(\theta_{\mathrm{c.m.}}/2)} \, \, \, [\mathrm{in} \, \, \mathrm{mb}/\mathrm{sr}],
\end{align}
where $Z_{p}$ is the unit charge of the incident particles, as before, $Z_{t}$ is the unit charge of the active target nuclei, $E_{\mathrm{c.m.}}$ is the total kinetic energy in the center-of-mass system, in units of MeV, and $\theta_{\mathrm{c.m.}}$ is the angle measured in the center-of-mass system. The elastic scattering cross section measured in the lab system must first be converted to the center-of-mass system before taking the ratio to Rutherford scattering. This conversion is described in Section \ref{subsec:lab_to_cm}.

\subsection{Cross Sections in the Center-of-Mass System} \label{subsec:lab_to_cm}

The incident beam energy $E_{\mathrm{lab}}$, scattering angle $\theta_{\mathrm{lab}}$, and the differential cross section itself $d\sigma/d\Omega_{\theta_{\mathrm{lab}}}$ must be converted to the center-of-mass frame to apply the Rutherford ratio and to compare with theoretical differential cross sections, which are usually given in the center-of-mass frame.

The conversion from the incident beam energy in the lab system $E_{\mathrm{lab}}$ to the total kinetic energy in the center-of-mass system $E_{\mathrm{c.m.}}$ is \cite{Iliadis2015}
\begin{equation}
E_{\mathrm{c.m.}} = E_{\mathrm{lab}} \frac{A_{t}}{A_{t} + A_{p}},
\end{equation}
where $A_{t}$ and $A_{p}$ are the nuclear masses of the active target nucleus and the incident particle nucleus, respectively. However, mass evaluations typically report atomic masses, not nuclear masses. In general, the nuclear masses $A_{\mathrm{Nuc}}$ can be calculated from the reported atomic masses $A_{\mathrm{Atom}}$ as \cite{Wang2021}
\begin{equation}
A_{\mathrm{Nuc}} = A_{\mathrm{Atom}} - Z \, m_{e} + B_{e}(Z),
\end{equation}
where $Z$ is the unit charge of the nucleus, $m_{e}$ is the electron mass  (548579.909065(16) $\times 10^{-9}$ $\mathrm{g}/\mathrm{mole}$ \cite{Huang2021}), and $B_{e}$ is the electron binding energy, found from Ref. \cite{Huang2021} to be calculated as
\begin{equation}
B_{e}(Z) = 14.4381 \, Z^{2.39} + 1.55468 \times 10^{-8} \, Z^{5.35} \, \mathrm{eV}.
\end{equation} 

The lab angle $\theta_{\mathrm{lab}}$ can be converted to the center-of-mass angle $\theta_{\mathrm{c.m.}}$ using the kinematics relation \cite{Iliadis2015}
\begin{equation} \label{eqn:angle_cm}
\cos \theta_{\mathrm{lab}} = \frac{\gamma + \cos \theta_{\mathrm{c.m.}}}{\sqrt{1 + \gamma^{2} + 2\gamma\cos \theta_{\mathrm{c.m.}}}},
\end{equation}
where the $\gamma$ parameter is defined as
\begin{equation}
\gamma = \sqrt{\frac{A_{p}A_{e}E_{\mathrm{lab}}}{A_{r}(A_{e} + A_{r})Q + A_{r}(A_{r} + A_{e} - A_{p})E_{\mathrm{lab}}}},
\end{equation}
with $A_{e}$ and $A_{r}$ representing the nuclear masses of the ejected particle and recoil nucleus, respectively, and $Q$ representing the $Q$-value of the reaction. For elastic scattering, $A_{e} = A_{p}$, $A_{r} = A_{t}$, and $Q=0$, so that $\gamma = A_{p}/A_{t}$. Eqn. \ref{eqn:angle_cm} can be solved numerically or it can be rewritten as a quadratic equation for $\theta_{\mathrm{c.m.}}$, using the positive solution for forward scattering.

The differential cross section is defined such that the same number of ejected particles cross the solid angle $d\Omega_{\mathrm{lab}}$ in the direction $\theta_{\mathrm{lab}}$ as those crossing the solid angle $d\Omega_{\mathrm{c.m.}}$ in the direction $\theta_{\mathrm{c.m.}}$. That is,
\begin{equation}
\left( \frac{d\sigma}{d\Omega} \right)^{\mathrm{lab}}_{\theta_{\mathrm{lab}}} \, d\Omega_{\mathrm{lab}} = \left( \frac{d\sigma}{d\Omega} \right)^{\mathrm{c.m.}}_{\theta_{\mathrm{c.m.}}} \, d\Omega_{\mathrm{c.m.}}.
\end{equation}
Assuming the cross section does not depend on the azimuthal angle, the conversion between the lab and center-of-mass systems is therefore
\begin{equation}
\frac{\left( d\sigma/d\Omega \right)^{\mathrm{c.m.}}_{\theta_{\mathrm{c.m.}}}}{\left( d\sigma/d\Omega \right)^{\mathrm{lab}}_{\theta_{\mathrm{lab}}}} = \frac{d\Omega_{\mathrm{lab}}}{d\Omega_{\mathrm{c.m.}}} = \frac{d(\cos \theta_{\mathrm{lab}})}{d(\cos \theta_{\mathrm{c.m.}})} = \frac{1 + \gamma \cos \theta_{\mathrm{c.m.}}}{\left( 1 + \gamma^{2} + 2 \gamma \cos \theta_{\mathrm{c.m.}} \right)^{3/2}}.
\end{equation}

\subsection{Si Detector Normalization} \label{subsec:SiNorm}

As mentioned in Section \ref{subsec:SiDet}, a Si detector telescope was used to  measure the $^{39}\mathrm{K}(^{3}\mathrm{He}, {}^{3}\mathrm{He})^{39}\mathrm{K}$ elastic scattering differential cross section at a constant $\theta_{\mathrm{lab}} = 45^{\circ}$, while the focal-plane detector was simultaneously measuring the differential cross sections of both $^{39}\mathrm{K}(^{3}\mathrm{He}, {}^{3}\mathrm{He})^{39}\mathrm{K}$ and $^{39}\mathrm{K}(^{3}\mathrm{He}, d)^{40}\mathrm{Ca}$ over multiple $\theta_{\mathrm{lab}}$ angles. The Si detector elastic scattering cross section was used as a normalization to correct for systematic uncertainties associated with unknown target effects during the focal-plane measurements. Although this is a relative measurement, an absolute scale can be established by comparing to optical model predictions, described in Section \ref{subsec:global_norm}.

The reason the relative measurement is effective is due to the conversion factor $n_{\mathrm{X}}$ between the yield and cross section in Eqn. \ref{eqn:yield_to_cross}. Since $n_{\mathrm{X}}$ only depends on the target properties, taking a relative cross section is equivalent to taking a relative yield, where the target properties cancel out. Because of the angular dependence, however, each differential yield (or equivalently, each differential cross section) must first be converted into the center-of-mass system before taking the ratio. The final form of the differential cross-section ratio $R$ in the center-of-mass system becomes
\begin{equation} \label{eqn:cross_ratio}
R \equiv \frac{\left( d\sigma/d\Omega_{\mathrm{FP}} \right)^{\mathrm{FP}}_{\theta_{\mathrm{c.m.}}}}{\left( d\sigma/d\Omega_{\mathrm{Si}} \right)^{\mathrm{Si}}_{\theta_{\mathrm{Si}}}} = \frac{N^{\mathrm{FP}}_{c} \, \Omega_{\mathrm{Si}}}{\Omega_{\mathrm{FP}} \, N^{\mathrm{Si}}_{c}} \, \frac{1 + \gamma \cos \theta_{\mathrm{c.m.}}}{\left( 1 + \gamma^{2} + 2 \gamma \cos \theta_{\mathrm{c.m.}} \right)^{3/2}} \, \frac{\left( 1 + \gamma^{2} + 2 \gamma \cos \theta_{\mathrm{Si}} \right)^{3/2}}{1 + \gamma \cos \theta_{\mathrm{Si}}},
\end{equation}
where $\theta_{\mathrm{c.m.}}$ is the angle for the focal plane measurement, $\theta_{\mathrm{Si}} = 48.14^{\circ}$ is the center-of-mass angle for the Si measurement ($\theta_{\mathrm{lab}} = 45^{\circ}$), $\Omega_{\mathrm{Si}} = 4.23(4)$ msr, and $\Omega_{\mathrm{FP}} = 1.00(4)$ msr for all runs of the transfer reaction and at angles $\theta_{\mathrm{lab}} \leq 35^{\circ}$ for elastic scattering; otherwise, $\Omega_{\mathrm{FP}} = 0.50(4)$ msr for elastic scattering. Note that the only experimental quantities required with a relative measurement are the number of detector counts $N_{c}$ and the solid angle $\Omega$ that each detector covers, in addition to the quantities making up the center-of-mass conversion. The $u_{R}$ uncertainty is found to be
\begin{equation}
u_{R} = R \sqrt{\left( \frac{u_{N_{c}}}{N_{c}} \right)^{2}_{\mathrm{FP}} + \left( \frac{u_{\Omega}}{\Omega} \right)^{2}_{\mathrm{FP}} + \left( \frac{u_{N_{c}}}{N_{c}} \right)^{2}_{\mathrm{Si}} + \left( \frac{u_{\Omega}}{\Omega} \right)^{2}_{\mathrm{Si}}},
\end{equation}
where the uncertainties in the angle measurement and $\gamma$ are sufficiently small to be neglected.

% Sort groups and weighted average over them at each angle. Variation from average ratio as an additional scatter uncertainty. Oxidation as an additional scatter uncertainty (20%).

Due to the presence of energy shifts in the $^{39}\mathrm{K}(^{3}\mathrm{He},d)^{40}\mathrm{Ca}$ experiment (see Section \ref{subsec:energy_shifts}), the runs usually could not all be sorted together at a given scattering angle. Therefore, there were multiple relative cross sections that needed to be calculated at these angles, one per sort group, and a method of averaging these measurements was necessary. A weighted average over each sort group $i$ at a given angle was deemed appropriate, where the weight is given in terms of the individual relative cross section uncertainties $u_{R}$, as in
\begin{equation}
\bar{R} = \frac{\sum^{n}_{i} R_{i}/u^{2}_{R, i}}{\sum^{n}_{i} 1/u^{2}_{R, i}},
\end{equation}
for $n$ total sort groups at the given angle. This ensures that the more precise measurements are weighted more. The uncertainty in this weighted average is then
\begin{equation}
u_{\bar{R}} = \sqrt{\frac{1}{\sum^{n}_{i} 1/u^{2}_{R, i}}}.
\end{equation}

In calculating the individual relative cross section uncertainties $u_{R}$, it was clear that an additional scatter uncertainty $u_{s}$ was required to accurately account for the variations from the average $R$. The goal was to apply the same fractional scatter uncertainty for all measurements of a given angle, based on the typical deviation from the average $R$ over all $^{40}$Ca states. An example of this is depicted in Fig. \ref{fig:variation}, where the variations from the average $R$, normalized to unity, at $\theta_{\mathrm{lab}} = 13^{\circ}$ are shown over all sort groups and for all $^{40}$Ca states, up to $E_{x} = 8935$ keV. Each color represents a different $^{40}$Ca state. It is clear that the variations from the average are typically no more than about $10\%$, which is also true for the other angles. The variation from the average $R$ is written as
\begin{equation}
\mathrm{var} = \frac{R}{R_{\mathrm{avg}}},
\end{equation}
where each fractional deviation is found from
\begin{equation}
f_{\mathrm{dev}} = | \mathrm{var} - 1 |.
\end{equation}
The average of these fractional deviations $f_{\mathrm{dev,\, avg}}$ over all the $^{40}$Ca states and all sort groups for a given angle is used as the fractional uncertainty applied to the additional scatter,
\begin{equation}
u_{s} = f_{\mathrm{dev,\, avg}} \, R.
\end{equation}
In cases where only one sort group was needed for an angle, the fractional uncertainty was instead obtained by taking $f_{\mathrm{dev,\, avg}}$ for each sort group and $^{40}$Ca state over all angles. 

\begin{figure}[t]
\centering
\includegraphics[width=6in]{Chapter-6/figs/variation_13deg.png}
\caption{\label{fig:variation}The variations from the average differential cross-section ratio $R$, normalized to unity, for each sort group over all $^{40}$Ca states measured at $\theta_{\mathrm{lab}} = 13^{\circ}$. The colored lines correspond to each $^{40}$Ca state up to $E_{x} = 8935$ keV. The variations are typically within $10\%$ of the average, which is true for the other angles as well. The average deviation for a given angle is used as an additional scatter uncertainty, as described in the text.}
\end{figure}

For sort groups where an oxidized target (KI $\#$1) was used, it was determined that the uncertainty was still underestimated, based on the tension between measurements at the same angle with targets that were not oxidized. An additional $f_{\mathrm{ox}} = 20\%$ scatter was added in these cases to account for any unknown variability and to apply more weight to measurements with non-oxidized targets. The resulting uncertainty in each $R$ measurement with the added scatter is thus
\begin{equation}
u'^{\, 2}_{R} = u^{2}_{R} + u^{2}_{s} + u^{2}_{\mathrm{ox}} = u^{2}_{R} + (f^{2}_{\mathrm{dev,\, avg}} + f^{2}_{\mathrm{ox}}) \, R^{2},
\end{equation}
where $f_{\mathrm{dev,\, avg}}$ covers all angles in cases where only one sort group was needed for an angle and $f_{\mathrm{ox}} = 0$ for non-oxidized targets. 

\subsection{Optical Model Normalization} \label{subsec:global_norm}

An absolute scale for the relative transfer differential cross section is established by applying a normalization factor $N_{\mathrm{ES}}$ obtained from the comparison between the relative elastic scattering differential cross section and the absolute differential cross section from a global $^{3}$He optical model potential (OMP) for $^{39}$K (see Section \ref{subsec:Optical_Model}). The parameters of a global OMP are found through an optimization procedure, fitting them to an exhaustive list of experimental elastic scattering data for several target nuclei. Both the elastic scattering and global OMP cross sections are given in terms of a ratio to Rutherford scattering, which is theoretically unity at $\theta_{\mathrm{c.m.}} = 0^{\circ}$.

% Part 1: The Optical Model Potential itself
The global $^{3}$He OMP of Ref. \cite{Liang2009} was chosen as the reference for the absolute scale. This potential includes real and imaginary volume terms, an imaginary surface term, real and imaginary spin-orbit terms, and a Coulomb potential, where the Woods-Saxon form factor $f_{i}(r)$ and its derivative are used, as described in Section \ref{subsec:Optical_Model}. A total of 148 sets of elastic scattering data for 52 target nuclei were used in the parametrization. The potential parameters are expressed as functions of beam energy and target mass. For $^{39}\mathrm{K} + {}^{3}\mathrm{He}$ at $E_{\mathrm{lab}} = 21$ MeV, the potential parameters of Ref. \cite{Liang2009} are given in Table \ref{tab:OMP_3He}, which are defined in the same manner as in Section \ref{subsec:Optical_Model}.

\begin{table}[t]
\centering
\caption{\label{tab:OMP_3He}Global optical model potential parameters for $^{39}\mathrm{K} + {}^{3}\mathrm{He}$ at $E_{\mathrm{lab}} = 21$ MeV from Ref. \cite{Liang2009}.}
\begin{tabular}{ccccccc}
\hline\midrule
$V_{r}$ & $r_{r}$ & $a_{r}$ & $W_{v}$ & $r_{v}$ & $a_{v}$ & $W_{s}$\\ \midrule
117.881 & 1.178 & 0.768 & -0.646 & 1.415 & 0.847 & 20.665\\
\hline\hline
\end{tabular}
\begin{tabular}{ccccccc}
$r_{s}$ & $a_{s}$ & $V_{so}$ & $r_{so}$ & $a_{so}$ & $W_{so}$ & $r_{c}$\\ \midrule
1.198 & 0.852 & 2.083 & 0.738 & 0.946 & -1.159 & 1.289\\
\hline\hline
\end{tabular}
\end{table}

% Part 2: The non-normalized differential cross sections
The nuclear reaction code \texttt{Fresco} \cite{Thompson1988,Fresco} was used to calculate the $^{39}\mathrm{K}(^{3}\mathrm{He}, {}^{3}\mathrm{He})^{39}\mathrm{K}$ differential cross section shown in Fig. \ref{fig:ES_NoNorm} using the OMP parameters in Table \ref{tab:OMP_3He} at $E_{\mathrm{lab}} = 21$ MeV. This differential cross section, represented as the black curve, was calculated in steps of $1^{\circ}$ from $\theta_{\mathrm{c.m.}} = 0^{\circ} - 80^{\circ}$. The present experimental differential cross section is shown in red at the angles $\theta_{\mathrm{lab}} = 15^{\circ} - 55^{\circ}$ in steps of $5^{\circ}$, and $59^{\circ}$, converted to $\theta_{\mathrm{c.m.}}$. The experimental points have been corrected for the presence of $^{41}$K in all of the elastic scattering peaks of the Si spectra and only at $\theta_{\mathrm{lab}} < 40^{\circ}$ in the focal-plane P1 spectra, described in Section \ref{subsec:SiDet}. Both the data and the model are given in terms of the ratio to Rutherford scattering. The difference in scale is attributed to the relative measurement of the focal-plane and the Si detectors.

\begin{figure}[t]
\centering
\begin{tikzpicture}[scale=1.0, every node/.style={transform shape}]
\node at (0,0) {\includegraphics[width=6.5in]{Chapter-6/figs/global_NoNorm.png}};
\node at (0,1.5) {\rlap{\LARGE{$\textbf{N}_{\mathrm{\textbf{ES}}}$}}};
\draw[>=triangle 45, line width = 0.5 mm, <->] (-0.3,3.6) -- (-0.3,-0.6);
\end{tikzpicture}
\caption{\label{fig:ES_NoNorm}The differential cross-section, as a ratio to Rutherford scattering, of the global $^{3}$He optical model potential of Ref. \cite{Liang2009} for $^{39}$K (black line) and that of the present relative $^{39}\mathrm{K}(^{3}\mathrm{He}, {}^{3}\mathrm{He})^{39}\mathrm{K}$ measurements (red data). The scaling factor $N_{\mathrm{ES}}$ is illustrated, which is also applied to the transfer data to establish an absolute scale.}
\end{figure}

% Part 3: The cubic interpolation method for normalization
The elastic scattering scale factor $N_{\mathrm{ES}}$ was calculated by taking the average ratio between each experimental point and the global model,
\begin{equation}
N_{\mathrm{ES}} = \frac{1}{N} \sum_{i=1}^{N} \frac{R_{\mathrm{ES}, i}}{\left( d\sigma/d\Omega_{\mathrm{ES, global}} \right)_{i}},
\end{equation}
where $R_{\mathrm{ES}, i}$ is defined by Eqn. \ref{eqn:cross_ratio} for the simultaneous $^{39}\mathrm{K}(^{3}\mathrm{He}, {}^{3}\mathrm{He})^{39}\mathrm{K}$ measurements of the focal-plane and Si detectors, $\left( d\sigma/d\Omega_{\mathrm{ES, global}} \right)_{i}$ is the global OMP differential cross section evaluated at the experimental angle $\theta_{\mathrm{c.m., i}}$, and $N$ is the number of experimental points. Since the global OMP differential cross section is not continuous, its value at each experimental center-of-mass angle must be interpolated. This is performed by cubic spline interpolation, where each pair of neighboring $d\sigma/d\Omega_{\mathrm{ES, global}}$ points defines a polynomial that ensures its first and second derivatives at each point are equal to that of neighboring polynomials. After interpolating the $N$ points, the scale factor becomes $N_{\mathrm{ES}} = 0.0277(11) \approx 1/36$. Fig. \ref{fig:ES_Norm} shows the global model scaled by $N_{\mathrm{ES}}$ to match the data. No $\chi^{2}$-minimization or Bayesian Monte Carlo algorithm has been performed on the OMP potential parameters. The global $^{3}$He OMP parameters themselves sufficiently reproduce the angular distribution of the experimental points. However, the OMP parameters give rise to large systematic uncertainties in the spectroscopic factor calculations for the transfer reaction, as demonstrated in Section \ref{subsec:DWBA_Spec}.

\begin{figure}[t]
\centering
\includegraphics[width=6.5in]{Chapter-6/figs/global_Norm.png}
\caption{\label{fig:ES_Norm}The same as Fig. \ref{fig:ES_NoNorm}, except the differential cross-section of the global $^{3}$He optical model potential has been scaled by $N_{\mathrm{ES}}$ to match the experimental measurements, as described in the text.}
\end{figure}

The same $N_{\mathrm{ES}}$ normalization is applied to the transfer differential cross sections, since they were also measured relative to the elastic scattering of the Si detector. The ratio between this corrected $^{39}\mathrm{K}(^{3}\mathrm{He}, d)^{40}\mathrm{Ca}$ differential cross section and that of an appropriate distorted-wave Born approximation (DWBA) model is the spectroscopic factor $C^{2}S$ (see Sections \ref{subsec:narrow_resonances} and \ref{subsec:DWBA}). This ratio is performed in the exact same way as the global OMP normalization, including the cubic spline interpolation. The transfer differential cross sections will be presented in Section \ref{subsec:DWBA_Cross}.

%%%%%%%%%%%%%%%%%%%%%%%%%%%%%%%%%%%%%%%%%%%%%%%%%%%%%%%%%%%%%%%%%%%%%%%%%%%%%%%%%%%%%%%%%%%%%
\section{$^{39}\mathrm{\textbf{K}}(^{3}\mathrm{\textbf{He}},d)^{40}\mathrm{\textbf{Ca}}$ Analysis}

\subsection{Energy Calibrations} \label{subsec:cal}

The centroid positions $\mu$ of the $^{40}$Ca states along the P1 focal-plane spectra were used to determine their excitation energies, given several known calibration states. These $\mu$ were obtained with high precision through the Bayesian peak fitting procedure, detailed in Section \ref{sec:peak_fitting}. However, the centroid uncertainty obtained from the fit does not represent the true channel uncertainty. The fit underestimates the uncertainty, as it does not take into account scatter associated with the known focal-plane detector response. An additional scatter uncertainty of $u_{\mathrm{s}} \sim 0.6$ channels was added to rectify this, corresponding to roughly 1 keV in excitation energy. The new channel uncertainty associated with each centroid value is
\begin{equation}
u_{\mathrm{ch}}^{2} = u_{\mu}^{2} + u_{\mathrm{s}}^{2}.
\end{equation}

The oxidized target fits were also not used in the energy calibration. It is not clear whether the mode of the exponentially-modified gaussian (EMG) fits properly represents their energies. The inconsistency between using gaussian means and EMG modes could cause problems. However, the only angles that have oxidized target data, $\theta_{\mathrm{lab}} = 5^{\circ}$ and $7^{\circ}$, also have data from a target that was not oxidized. The EMG fit area was still used for the yield determination because the total number of counts should not be affected by target oxidation. Scatter was added to these yields regardless to put more weight on those from non-oxidized targets in the weighted averages.

% 1.) Energy calibration + Calibration states figure
The $^{40}$Ca calibration states that were used in the energy calibration are highlighted in red in Fig. \ref{fig:Calibration_States} for the $\theta_{\mathrm{lab}} = 5^{\circ}$ focal plane position spectrum. These are the 4491 keV, 5614 keV, 6025 keV, 7532 keV, 8425 keV, and 9454 keV $^{40}$Ca states. The energies are provided by the ENSDF evaluation of Ref. \cite{Chen2017}, except the 8425 keV state uses the more recent energy measurement of Ref. \cite{Gribble2022}. Unfortunately, not all of these states were unobscured throughout the course of the experiment, which made it necessary to use alternative calibration states in those cases. In total, of the 13 sort groups with non-oxidized targets in the $^{39}\mathrm{K}(^{3}\mathrm{He},d)^{40}\mathrm{Ca}$ experiment (see Section \ref{subsec:energy_shifts}), an alternative calibration state had to be used for 4 of them. At both the $\theta_{\mathrm{lab}} = 5^{\circ}$ (sort group $\#$3) and $\theta_{\mathrm{lab}} = 7^{\circ}$ (sort group $\#$5) angles, the 8425 keV state was obscured by the first excited state from $^{16}\mathrm{O}(^{3}\mathrm{He},d)^{17}\mathrm{F}$. The alternative calibration state for these sort groups was the 8484 keV state, as it is closest in energy and was clearly resolved. Similarly, for both sort groups ($\#15-\#16$) at $\theta_{\mathrm{lab}} = 20^{\circ}$, the 9454 keV calibration state did not have enough counts to properly resolve it from the neighboring $9405-9432$ keV multiplet. In these cases, the 9136 keV state was used as the alternative calibration state. The closer 9538 keV state was determined to be too near the edge of the detector to have a reliable centroid and yield measurement. The 9227 keV state was also ruled out due to it being impossible to determine which of the 9226.69 keV or 9227.43 keV states, or both, from ENSDF it belongs to, as they both have a wide range of possible $J^{\pi}$ assignments. It was still possible to report energies for these alternative calibration states over the sort groups where they were not involved in the calibration.

\begin{figure}[t]
\centering
\includegraphics[width=6.5in]{Chapter-6/figs/CalibrationStates.png}
\caption{\label{fig:Calibration_States}The calibration states for the MCMC energy calibration, represented by red vertical lines for the $\theta_{\mathrm{lab}} = 5^{\circ}$ focal plane position spectrum. Their ENSDF \cite{Chen2017} energies are labeled in keV. The 8425 keV state uses the energy of the recent measurement from Ref. \cite{Gribble2022} instead. The 8484 keV and 9136 keV alternative calibration states are used when the 8425 keV and 9454 keV states were obscured at $\theta_{\mathrm{lab}} = 5,7^{\circ}$ and $20^{\circ}$, respectively. The $^{17}$F contaminant appears in exactly the same position as the 8425 keV state in the figure.}
\end{figure}

The energy calibration was performed with a quadratic fit that represents the magnetic rigidity $B\rho$ of the Enge Split-Pole Spectrograph as a function of the focal-plane detector channels $x_{\mathrm{ch}}$,
\begin{equation} \label{eqn:energy_cal}
B\rho = a + b_{1}x_{\mathrm{ch}} + b_{2} x_{\mathrm{ch}}^{2}.
\end{equation}
The calibration uses $x_{\mathrm{ch}}$ and $B\rho$ from the known calibration states to predict $B\rho$ for all other states, known as test states. The excitation energies of the calibration states are first converted to $B\rho$ by interpolation from the kinematics code \texttt{jrelkin} \cite{jrelkin}. After the calibration, the predicted $B\rho$ are converted back to energy in the same manner, with uncertainties propagated through both the calibration and the energy conversion.

Markov chain Monte Carlo (MCMC) fits were performed with Eqn \ref{eqn:energy_cal} using the Hamiltonian Monte Carlo and Metropolis algorithms built into \texttt{rstan} \cite{Stan,Rstan}. This allowed for Bayesian statistical inference of the fit parameters from MCMC samples. The parameters, including an extra scatter parameter $\sigma$ added in quadrature to each $B\rho$ calibration state uncertainty, were given the prior distributions
\begin{align}
a \sim& \, \mathcal{N}(0,1^{2}), \nonumber \\
b_{1} \sim& \, \mathcal{N}(0,1^{2}), \nonumber \\
b_{2} \sim& \, \mathcal{N}(0,1^{2}), \nonumber \\
\sigma \sim& \, \mathrm{Cauchy}(0,1^{2}).
\end{align}
A total of $5 \times 10^{4}$ iterations were performed for each of the four chains that are used by \texttt{rstan}. Samples were extracted from the last $2.5 \times 10^{4}$ iterations, and the first half were used as warmup. Using the new posteriors, $B\rho$ was sampled over the 4096 channels and converted to $^{40}$Ca excitation energy. Energy residuals from the median calibration line were also sampled over the 4096 channels.

% 2.) MCMC Residual figure for 5 deg:
\begin{figure}[t]
\centering
\includegraphics[width=6.5in]{Chapter-6/figs/residuals-5deg.png}
\caption{\label{fig:residuals_5deg}Residuals between the excitation energies predicted from the MCMC energy calibration and that of the median calibration line at $\theta_{\mathrm{lab}} = 5^{\circ}$ (sort group $\#$3) for the calibration states (in red) and the test states (in black). The $68\%$ ($1\sigma$) and $95\%$ ($2\sigma$) confidence intervals for the fit are represented by the dark and light shading, respectively. See text for details.}
\end{figure}

The energy residuals for $\theta_{\mathrm{lab}} = 5^{\circ}$ (sort group $\#$3) are shown in Fig. \ref{fig:residuals_5deg}. The $68\%$ ($1\sigma$) and $95\%$ ($2\sigma$) confidence intervals from the MCMC samples are shown in dark and light shading, respectively, centered on the median calibration line. The red points represent the calibration states, and the black points represent the test states. Uncertainty bars shown for the test states are simply their input centroid uncertainties $u_{\mathrm{ch}}$ converted to energy from the fit, $u_{\mathrm{ch-to-E}}$, added in quadrature with the ENSDF uncertainties. The uncertainties reported for each test state among each sort group are $u_{\mathrm{ch-to-E}}$ added in quadrature with half the $1\sigma$ width from the fit at the given energy. That is,
\begin{equation}
u_{E}^{2} = u_{\mathrm{ch-to-E}}^{2} + \left( \frac{u_{\mathrm{MCMC, upper}} - u_{\mathrm{MCMC, lower}}}{2} \right)^{2},
\end{equation}
where upper and lower refer to the 84th and 16th percentile of the MCMC samples, respectively.

% 3.) Dealing with uncertainties and averaging + 8551 keV Simple Average figure
The calibration was performed for every sort group, except for those where an oxidized target was used (sort groups $\#1$, $\#2$ and $\#4$). Each test state therefore resulted in up to 13 reported energies, 1 for each sort group where the state was resolvable. Several averaging methods were attempted to obtain a final reported energy for each test state. A weighted average $\bar{E}_{wavg}$ was the first attempt, where the weights were $1/u_{E}^{2}$, and the uncertainties were propagated through the weighted average. However, it was clear that for many test states, the scatter was not properly accounted for with the weighted average and its uncertainty. The few sort groups with smaller uncertainties shifted the average unreasonably far from that of the other sort groups. A simple average $\bar{E}$ was then attempted, which seemed to be better suited for the scatter. However, the uncertainty propagated through the simple average also did not reproduce the scatter appropriately. What finally captured the appropriate scatter was taking a simple average of the reported uncertainties. That is,
\begin{equation} \label{eqn:simple_avg}
\bar{E} \pm \bar{u_{E}} = \frac{1}{N}\sum_{i}^{N}{E_{i}} \pm \frac{1}{N}\sum_{i}^{N}{u_{E, i}},
\end{equation}
was determined to be the appropriate final energy and uncertainty for each state.

\begin{figure}[!p]
\centering
\includegraphics[width=6.5in]{Chapter-6/figs/8551keV_comparison.png}
\caption{\label{fig:8551keV_comparison}Excitation energies and their uncertainties of the 8551 keV state from the MCMC energy calibration for each sort group, excluding sort groups $\#$1 and $\#$2 that used an oxidized target. The ENSDF energy and its uncertainty are given as the red line and red band, respectively. A simple average and a weighted average of the energies are given by the black and orange lines, respectively. The uncertainty propagated from the simple average and from the weighted average are given by the blue and orange bands, respectively. The simple average of the MCMC uncertainties is represented by the green band, which best captures the range of the individual uncertainties for each state.}
\end{figure}

\begin{figure}[!p]
\centering
\includegraphics[width=6.5in]{Chapter-6/figs/res.png}
\caption{\label{fig:residuals}Residuals between the final reported energies for each state from the MCMC energy calibration and that of ENSDF \cite{Chen2017}. The residuals and uncertainties for each state result from a simple average of the MCMC energies and uncertainties of each sort group, respectively. The residuals in red represent the astrophysical region, from 8359 keV to 8995 keV. The $x$-axis values correspond to the following states in keV; 1: 5903, 2: 6285, 3: 6582, 4: 6750, 5: 6950, 6: 7113, 7: 7623, 8: 7658, 9: 7694, 10: 7973, 11: 8113, 12: 8188, 13: 8271, 14: 8359, 15: 8484, 16: 8551, 17: 8665, 18: 8748, 19: 8764, 20: 8851, 21: 8935, 22: 8995, 23: 9092, 24: 9136, 25: 9227, 26: 9405, 27: 9419, 28: 9432, 29: 9538, 30: 9603.}
\end{figure}

To illustrate these different averaging methods, the resulting distribution of calibrated energies for the representative 8551 keV state is shown by the black data in Fig. \ref{fig:8551keV_comparison}. This is an example where the scatter was minimized, potentially due to its relatively large yield, but the different averaging methods are clearly distinguished. The black line represents the simple average, while the orange line represents the weighted average. The propagated uncertainty from each of these averages are the blue and orange bands, respectively. The green band shows the simple average uncertainty of Eqn. \ref{eqn:simple_avg}, coming from the black line. The energy from ENSDF is shown in red, with its uncertainty band also in red. Note that the blue band does not agree with ENSDF, and there is also tension between the orange band and ENSDF.

% 4.) Final Residuals from ENSDF for all states
The final residuals between the energy-calibrated states using Eqn. \ref{eqn:simple_avg} and their corresponding ENSDF state are shown in Fig. \ref{fig:residuals}. It is clear that the typical energy uncertainty is in the range of $2-3$ keV. The states highlighted in red are the ones that contribute to the $^{39}\mathrm{K}(p,\gamma)^{40}\mathrm{Ca}$ reaction rate and are therefore of astrophysical interest. These range from 8359 keV to 8995 keV. The excited state labels are provided in the figure caption. There is decent agreement with ENSDF for most states at relatively high excitation energy, but there is considerable scatter at lower energies. In particular, the state that most disagrees with ENSDF is the 8188 keV state, which has a calibrated energy of 8199.3(27) keV, or a residual of 11.8(28) keV. This may correspond to the 8196 keV state in ENSDF instead, but there is a considerable lack of experimental evidence for this state. This hypothesis will be discussed further in Section \ref{subsec:DWBA_Cross}.

% 5.) The Final Energy table
The final energies from the MCMC calibration are reported in Table \ref{tab:energies}. Calibration states are shown in italics, except for the alternative calibration states 8484 keV and 9136 keV that were still able to have their energies determined. Cases where the observed state corresponds to an unresolvable doublet according to ENSDF are represented with a bracket around the doublet. The assigned $J^{\pi}$ values from ENSDF are also given. Fig. \ref{fig:AstroRegion} shows an energy calibrated focal plane spectrum of the proton-unbound $^{40}$Ca states labeled with their ENSDF \cite{Chen2017} energies, where the proton separation energy $S_{p} = 8328.18(2)$ keV \cite{Wang2021} is indicated by the red line.

\newpage

\begingroup % This and the following line are used to get table footnote marks to be alphabetic, but do this locally (not globally)
  \renewcommand*{\thefootnote}{\alph{footnote}}
  \renewcommand*\footnoterule{} % Get rid of the footnote rule
\begin{table}[H]
\centering
\begin{minipage}{\textwidth}
\centering
\caption{\label{tab:energies}Excitation energies $E_{x}$ determined through a Markov Chain Monte Carlo (MCMC) energy calibration for the $^{40}$Ca states observed in the present experiment. For comparison, the ENSDF \cite{Chen2017} excitation energies and $J^{\pi}$ values are also given, unless otherwise indicated. Brackets represent states from ENSDF too close in energy to resolve via the energy calibration alone. Calibration states are shown in italics.}
\hrule width \hsize \kern 1mm \hrule width \hsize height 0.4pt
\vspace{0.1cm}
\begin{tabular}{lll}
%\hline\midrule
$E_{x}$\footnotemark[1] [keV]&$E_{x}$\footnotemark[2] [keV]&$J^{\pi}$\footnotemark[1]\\ \midrule
4491.43(4)&\emph{4491}\footnotemark[3]&$5^{-}$\\
5613.52(3)&\emph{5614}\footnotemark[3]&$4^{-}$\\
5902.63(7)&5900.5(26)&$1^{-}$\\
6025.47(5)&\emph{6025}\footnotemark[3]&$2^{-}$\\
6285.15(4)&6290.0(28)&$3^{-}$\\
6582.47(10)&6582.6(30)&$3^{-}$\\
6750.41(7)&6745.3(34)&$2^{-}$\\
6950.48(7)&6946.1(24)&$1^{-}$\\
&&\\
\hspace{-3mm}\ldelim \{ {1.8}{3mm}7113.1(10)&7109.4(29)&$1^{-}$\\
7113.73(5)&&$4^{-}$\\
&&\\
7532.26(5)&\emph{7532}\footnotemark[3]&$2^{-}$\\
7623.11(8)&7627.8(28)&$(2^{-},3,4^{+})$\\
7658.23(5)&7661.0(28)&$4^{-}$\\
7694.08(4)&7696.3(27)&$3^{-}$\\
7972.5&7976.5(30)&$3^{-}$\\ % No NNDC Uncertainty given
8113.2(5)&8112.2(28)&$1^{-}$\\
8187.5(8)&8199.3(27)&$3,4,5^{-}$\\
8271(1)&8270.6(25)&$3^{-}$\\
8358.9(6)&8357.3(25)&$(0,1,2)^{-}$\\
8424.35(31)\footnotemark[4]& \emph{8425}\footnotemark[3]&$2^{-}$\\
8484.02(13)&8482.3(29)\footnotemark[3]&$(1^{-},2^{-},3^{-})$\\
8551.1(7)&8549.5(25)&$5^{-}$\\
8665.3(8)&8662.9(25)&$1^{-}$\\
8748.59(19)\footnotemark[4]&8743.6(28)&$2^{+}$\\
8764.18(6)&8767.3(27)&$3^{-}$\\
%\hline\hline
\end{tabular}
\quad
\begin{tabular}{lll}
%\hline\midrule
$E_{x}$\footnotemark[1] [keV]&$E_{x}$\footnotemark[2] [keV]&$J^{\pi}$\footnotemark[1]\\ \midrule
8850.6(9)&8849.5(27)&$6^{-},7^{-},8^{-}$\\
&&\\
\hspace{-3mm}\ldelim \{ {1.8}{3mm}8934.81(7)&8936.3(27)&$2^{+}$\\
8938.4(9)&&$0^{+}$\\
&&\\
8994.5(11)&8992.2(25)&$(1^{-},2^{+})$\\
9091.7(6)&9092.2(36)&$3^{-}$\\
9135.66(5)&9137.8(34)\footnotemark[3]&$2^{-},3^{-}$\\
&&\\
\hspace{-3mm}\ldelim \{ {1.8}{3mm}9226.69(5)&9227.8(34)&$(1^{-},2,3^{-})$\\
9227.43(7)&&$(1,2^{+})$\\
&&\\
\hspace{-3mm}\ldelim \{ {1.8}{3mm}9404.85(19)&9403.7(31)&$2^{-}$\\
9406.3(6)&&$0^{+}$\\
&&\\
\hspace{-3mm}\ldelim \{ {1.8}{3mm}9412.3(2)&9416.7(32)&\\
9418.8(2)&&$3^{-}$\\
&&\\
\hspace{-3mm}\ldelim \{ {1.8}{3mm}9429.11(5)&9431.7(32)&$(3,4)^{-}$\\
9432.46(18)&&$1^{-}$\\
&&\\
9453.95(5)&\emph{9454}\footnotemark[3]&$3^{-}$\\
9537.8(5)&9538.4(34)&$1^{-}$\\
&&\\
\hspace{-3mm}\ldelim \{ {1.8}{3mm}9603.0(4)&9605.1(40)&$3^{-}$\\
9604.6(4)&&$1^{-}$\\
% NOTE: Not including the 9662+9669 keV doublet's energy, since it was only extracted for 2 angles
%\hspace{-3mm}\ldelim \{ {2}{3mm}9662.2(2)&9673.4(31)&1345.2(31)&$3^{-}$\footnotemark[3]& & & \\
%9668.71(8)& & &$3^{-}$\footnotemark[3]& & &$1.73 \times 10^{4}$ \\
%\hline\hline
\vspace{0.0001cm}
\end{tabular}
\hrule width \hsize \kern 1mm \hrule width \hsize height 0.4pt
\vspace{-0.2cm}
\footnotetext[1]{From the ENSDF evaluation of Ref. \cite{Chen2017}, unless otherwise indicated.}
\footnotetext[2]{Present experiment.}
\footnotetext[3]{Level used as a calibration point for at least one angle in this work. See text for details.}
\footnotetext[4]{Energy from the nuclear resonance fluorescence measurement of Ref. \cite{Gribble2022}.}
\end{minipage}
\end{table}
\endgroup

\newpage

% Energy-calibrated spectrum in the astrophysical region of interest
\begin{figure}
\centering
\begin{tikzpicture}
\node at (0,0) {\includegraphics[width=6.5in]{Chapter-6/figs/AstroRegion_Calibrated.png}};
\draw[line width = 0.5mm, color=red] (-6.36,-3.5) -- (-6.36,-2.5);
\end{tikzpicture}
\caption{\label{fig:AstroRegion}The energy-calibrated focal plane position spectrum, focused on the $^{40}$Ca states above the proton separation energy at $\theta_{\mathrm{lab}} = 5^{\circ}$ (sort group $\#$3). The $^{40}$Ca states are labeled with their ENSDF \cite{Chen2017} energies in keV, and contaminants at this angle are indicated. The red vertical line indicates the proton separation energy, $S_{p} = 8328.18(2)$ keV from Ref. \cite{Wang2021}.}
\end{figure}

\newpage

\subsection{DWBA Calculations} \label{subsec:DWBA_Cross}

% C2S from DWBA
The $^{39}\mathrm{K}(^{3}\mathrm{He},d)^{40}\mathrm{Ca}$ differential cross sections were normalized by the same amount $N_{\mathrm{ES}}$ as that of the focal plane $^{39}\mathrm{K}(^{3}\mathrm{He}, {}^{3}\mathrm{He})^{39}\mathrm{K}$ measurements, since both were measured relative to the elastic scattering of the Si detector. The remaining normalization,
\begin{equation} \label{eqn:spec_factor_dwba}
\frac{d\sigma}{d\Omega}_{\mathrm{Exp}} = C^{2}S \frac{d\sigma}{d\Omega}_{\mathrm{DWBA}},
\end{equation}
between the scale-corrected transfer differential cross section $d\sigma/d\Omega_{\mathrm{Exp}}$ and that of an appropriate distorted-wave Born approximation (DWBA) model $d\sigma/d\Omega_{\mathrm{DWBA}}$ defines the spectroscopic factor $C^{2}S$. The spectroscopic factor for $^{39}\mathrm{K}(^{3}\mathrm{He},d)^{40}\mathrm{Ca}$ is broken into two parts, corresponding to the $^{39}\mathrm{K}+p$ and $d+p$ overlaps (see Section \ref{subsec:DWBA}),
\begin{equation}
C^{2}S = C^{2}S_{^{39}\mathrm{K}+p} \, C^{2}S_{d+p}.
\end{equation}
The $C^{2}S_{d+p}$ spectroscopic factor is frequently omitted in the literature, as it is a constant factor applied to all $C^{2}S$ for $(^{3}\mathrm{He},d)$ reactions. It can be approximated as $A/2$ for $A \leq 4$ nuclei \cite{Satchler1983}, which implies $C^{2}S_{d+p} = 1.5$. However, \emph{ab-initio} calculations such as those from Ref. \cite{Brida2011} can be more precise, reducing systematic uncertainties associated with $C^{2}S_{^{39}\mathrm{K}+p}$. Ref. \cite{Brida2011} calculates $C^{2}S_{d+p} = 1.32(1)$, which is adopted in the present analysis. The $C^{2}S_{^{39}\mathrm{K}+p}$ spectroscopic factor for proton-unbound states, when converted to a proton partial-width in Section \ref{subsec:partial_widths}, is a direct input in the $^{39}\mathrm{K}(p,\gamma)^{40}\mathrm{Ca}$ reaction rate. This is the main quantity of astrophysical interest in transfer reactions.

% DWBA OMPs for incoming and outgoing channels
DWBA transfer reaction calculations require optical model potentials (OMPs) for both the incoming $^{39}\mathrm{K} + {}^{3}\mathrm{He}$ and the outgoing $d + {}^{40}\mathrm{Ca}$ channels. The potential for the incoming channel is evaluated at the lab beam energy $E_{\mathrm{lab}}$, whereas the outgoing channel potential is evaluated at $E_{\mathrm{lab}} - Q$, where $Q$ is the $Q$-value of the transfer reaction. The global $^{3}$He OMP of Ref. \cite{Liang2009} from the elastic scattering normalization is retained for the incoming channel. The parameters of this OMP were given in Table \ref{tab:OMP_3He}. For the outgoing channel, the global $d$ OMP of Ref. \cite{An2006} is used in the present analysis, where the potential is defined in the exact same way as that of Ref. \cite{Liang2009}.

% DWBA proton-bound state potential and the weak binding approximation
The proton bound-state potential must also be specified for DWBA calculations of proton-bound $^{40}$Ca states. It consists of a real volume potential with a Woods-Saxon form and a Coulomb potential. The Woods-Saxon well depth $V_{r}$ is automatically adjusted to reproduce the proton binding energy,
\begin{equation}
\mathrm{BE}\left(^{A-1}(X-1) + p \right) = S_{p}(^{A}X) - E_{x},
\end{equation}
where $S_{p}$ is the proton-separation energy of the compound nucleus $^{A}X$, and $E_{x}$ is its excitation energy. However, many of the states observed in the present experiment, and indeed all of the states of astrophysical interest, are proton-unbound. In these cases, one could adjust the Woods-Saxon well depth until a resonance is formed at $E_{r}$. This is unfortunately not performed automatically in \texttt{Fresco} \cite{Thompson1988,Fresco}, the nuclear reaction code chosen for the present analysis. An alternative to this is to approximate the resonance scattering wave function as the wave function for a loosely-bound particle, with a binding energy of $\lesssim$ 50 keV. This is known as the weak binding approximation, and it has been shown to reproduce unbound calculations of $C^{2}S$ to within $1\%$ for states within about 500 keV of the particle threshold and $6\%$ for states within about 1800 keV above the particle threshold \cite{Kankainen2016,Kahl2019}. The present analysis uses the weak binding approximation for unbound $^{40}$Ca states. The proton bound-state parameters, along with the global OMP parameters are summarized in Table \ref{tab:OMP_all}.

% Global OMP parameters
\begin{table}[t]
\caption{\label{tab:OMP_all}Global optical model potential parameters for $^{39}\mathrm{K} + \,^{3}\mathrm{He}$ and $^{40}\mathrm{Ca} + d$ and the proton bound-state parameters.}
\resizebox{\linewidth}{!}{%
\begin{tabular}{cllllllllllllll}
\hline\midrule
 $\mathrm{particle}$ & $V_{r}$ & $r_{r}$ & $a_{r}$ & $W_{v}$ & $r_{v}$ & $a_{v}$ & $W_{s}$ & $r_{s}$ & $a_{s}$ & $V_{so}$ & $r_{so}$ & $a_{so}$ & $W_{so}$ & $r_{c}$\\ \midrule
 $^{3}\mathrm{He}$ & 117.881 & 1.178 & 0.768 & -0.646 & 1.415 & 0.847 & 20.665 & 1.198 & 0.852 & 2.083 & 0.738 & 0.946 & -1.159 & 1.289\\
 $d$ & 91.120 & 1.150 & 0.762 & 2.234 & 1.334 & 0.513 & 10.274 & 1.378 & 0.743 & 3.557 & 0.972 & 1.011 & & 1.303\\
 $p$ & $V_{p}\footnotemark[1]$ & 1.250 & 0.650 &  &  &  &  &  &  &  &  &  &  & 1.250\\
\hline\hline
\end{tabular}%
}
\footnotetext[1]{Well depth is adjusted to reproduce the proton binding energy.}
\end{table}

% Overlaps and the different nlj combinations
Another factor that must be considered in DWBA calculations is the transferred-particle overlaps, in this case for $^{39}\mathrm{K}+p$ and $d+p$. Each overlap is specified by a single-particle state $n \ell_{j}$ from the shell model with the quantum numbers $n$, $\ell$, and $j$ corresponding to the node, transferred orbital angular momentum, and total angular momentum ($j = \ell \pm 1/2$), respectively (see Ref. \cite{Krane1987}). Only a select few single-particle states can be populated from a given transition, based on quantum-mechanical selection rules for angular momentum and parity (see Ref. \cite{Iliadis2015}). For example, if the spin-parity $J^{\pi}$ of the final $^{40}$Ca state is known, the possible $n$, $\ell$, and $j$ single-particle quantum numbers can be extracted from the selection rules. This can also be done for final states where $J^{\pi}$ is narrowed down to a few possibilities. This extraction was performed for each excited state of $^{40}$Ca populated in the present experiment, given their known spin-parities, or spin-parity candidates, from the ENSDF evaluation of Ref. \cite{Chen2017}. The $d+p$ overlap was treated as constant for each DWBA calculation, since it was assumed that it occupied only the $^{3}$He ground state.

The nuclear reaction code \texttt{Fresco} \cite{Thompson1988,Fresco} was used to calculate $^{39}\mathrm{K}(^{3}\mathrm{He},d)^{40}\mathrm{Ca}$ DWBA differential cross sections using the zero-range (ZR) approximation introduced in Section \ref{subsec:DWBA}. Calculations were performed for the allowed $n\ell_{j}$ combinations of each excited state, based on their known $J^{\pi}$ values from ENSDF. In cases where none of the angular distributions agreed with the experimental cross section, the $n\ell_{j}$ combinations were expanded to account for different final $J^{\pi}$ values. An example of the normalization procedure and the different $n\ell_{j}$ combinations for the 7532 keV state is shown in Fig. \ref{fig:7532keV_NoNorm}. This state has a spin-parity of $2^{-}$ from ENSDF, and therefore allows $\ell=1$ or $\ell=3$ orbital angular momenta from selection rules. The most likely single-particle states corresponding to these $\ell$-values are $2\mathrm{p}_{3/2}$ or $2\mathrm{p}_{1/2}$ for $\ell=1$ and $1\mathrm{f}_{7/2}$ or $1\mathrm{f}_{5/2}$ for $\ell=3$, as they are the lowest energy transitions from the $^{39}$K ground state at $1d_{3/2}$. The higher $n$-values for a given $\ell$ transition have a much greater transition energy and are therefore less likely. For that reason they are typically omitted in the literature. However, they were included in the analysis for completion and to test their effects. From Fig. \ref{fig:7532keV_NoNorm}, it is clear that the DWBA differential cross sections increase with increasing $n$. It can also be seen that the $n$-$\ell$ pairs produce very similar differential cross sections at low angles. This is especially true at angles below the first minimum, where the DWBA differential cross sections are most reliable. It is common during normalization to omit angles greater than the first DWBA minimum because that is roughly where the distorted-wave Born approximation loses validity. However, the present $^{39}\mathrm{K}(^{3}\mathrm{He}, d)^{40}\mathrm{Ca}$ experiment did not obtain data beyond the first minimum, so normalization was performed with all experimental points.

\begin{figure}[t]
\centering
\begin{tikzpicture}
\node at (0,0) {\includegraphics[width=6.5in]{Chapter-6/figs/7532keV_NoNorm.png}};
\node at (-4.2,0.2) {\rlap{\LARGE{$\textbf{N}_{\mathrm{\textbf{ES}}}$}}};
\node at (-4.2,3.15) {\rlap{\LARGE{$\textbf{C}^{2}\textbf{S}$}}};
\draw[>=triangle 45, line width = 0.5 mm, <->] (-2.7,3.9) -- (-2.7,2.4);
\draw[>=triangle 45, line width = 0.5 mm, <->] (-2.7,2.4) -- (-2.7,-2.0);
\end{tikzpicture}
\caption{\label{fig:7532keV_NoNorm}Normalization between the relative transfer differential cross section of the 7532 keV state and that of all possible DWBA models based on the current ENSDF $J^{\pi} = 2^{-}$ assignment. The normalization $N_{\mathrm{ES}}$ between elastic scattering and the global $^{3}$He OMP is shown to scale, as is the $C^{2}S$ normalization.}
\end{figure}

Fig. \ref{fig:7532keV_Norm} shows the same differential cross sections as in Fig. \ref{fig:7532keV_NoNorm}, but the DWBA models have been scaled down to the experimental points to show all normalizations simultaneously. It is clear that the $\ell=1$ models match the angular distribution of the points. However, it is not possible to tell which of the $\ell=1$ models is the best match. As is done in the literature \cite{Erskine1966,Seth1967,Forster1970,Cage1971,Fuchs1969}, the present analysis chooses the $n$-$j$ pair associated with the lowest energy single-particle state, unless it is forbidden by selection rules. That is, unless they are forbidden, all $\ell=0$, $\ell=1$, $\ell=2$, and $\ell=3$ distributions will be associated with $3\mathrm{s}_{1/2}$, $2\mathrm{p}_{3/2}$, $2\mathrm{d}_{5/2}$, and $1\mathrm{f}_{7/2}$ single-particle states, respectively.

\begin{figure}[t]
\centering
\includegraphics[width=6.5in]{Chapter-6/figs/7532keV_Norm.png}
\caption{\label{fig:7532keV_Norm}The DWBA models of Fig. \ref{fig:7532keV_NoNorm} scaled down to the 7532 keV data to compare angular distributions. The $l=1$ models are a clear match for this state.}
\end{figure}

Finally, another transfer consideration comes from the fact that $^{39}$K has a non-zero ground state ($J^{\pi} = 3/2^{+}$). It is possible in this case that more than one single-particle state contributes to the differential cross section, resulting in a mixed-$\ell$ angular distribution. Theoretically, all allowed $\ell$ transitions can contribute, but it is usually impossible to distinguish between more than the two lowest $\ell$ transitions. The differential cross section decreases with increasing $\ell$, making each larger $\ell$ contribution more negligible \cite{Hodgson1971}. The mixed-$\ell$ angular distributions should then resemble a sum of two DWBA differential cross sections,
\begin{equation} \label{eqn:mixed_l}
\frac{d\sigma}{d\Omega}_{\mathrm{Exp}} = C^{2}S_{\ell_{1}} \frac{d\sigma}{d\Omega}_{\mathrm{DWBA},\ell_{1}} + C^{2}S_{\ell_{2}} \frac{d\sigma}{d\Omega}_{\mathrm{DWBA},\ell_{2}}.
\end{equation} 
The individual spectroscopic factors are given the common parameter $\alpha$ which determines their fractional contributions, as in
\begin{align}
C^{2}S_{\ell_{1}} &= \alpha C^{2}S, \\
C^{2}S_{\ell_{2}} &= (1 - \alpha) C^{2}S,
\end{align}
where $C^{2}S$ represents the joint spectroscopic factor. A $\chi^{2}$-minimization is performed in mixed-$\ell$ cases to determine the $\alpha$ and $C^{2}S$ parameters, and the individual $C^{2}S_{\ell_{1}}$ and $C^{2}S_{\ell_{2}}$ are then reported. Their statistical uncertainties are obtained from the variance-covariance matrix elements added in quadrature to the $d\sigma/d\Omega_{\mathrm{Exp}}$ uncertainties.

\subsection{Angular Distribution Results}

Figs. \ref{fig:diffcross_6panel_1}--\ref{fig:unbound_l} show the angular distributions extracted for each $^{40}$Ca state observed in the $^{39}\mathrm{K}(^{3}\mathrm{He}, d)^{40}\mathrm{Ca}$ experiment, with a few exceptions. The states observed above 9227 keV (9405--9603 keV) did not match any DWBA prescription. The experimental differential cross sections for these states rapidly decreased as a function of angle faster than even the lowest possible $\ell$ transition. This is likely due to these peaks being too near the edge of the focal plane detector, which is exacerbated with increasing angle. The detector efficiency is significantly reduced at the fringes, which makes yield measurements unreliable. Another factor could be the breakdown of the weak binding approximation in DWBA models for states with energies this high above the proton separation energy, albeit likely to a lesser degree, considering the good agreement with the nearby 9227 keV differential cross section. Energies were still extracted for states up to 9603 keV with relatively small residuals from ENSDF \cite{Chen2017} because detector efficiency does not affect centroid position, and DWBA calculations are not used in the energy calibration. Fig. \ref{fig:unbound_l} focuses on the proton-unbound $^{40}$Ca states of astrophysical interest ($E_{x} =$ 8359--8851 keV), which are below the lowest directly measured $^{39}\mathrm{K}(p, \gamma)^{40}\mathrm{Ca}$ resonance at $E_{x} =$ 8935 keV ($E_{r}^{\mathrm{c.m.}} =$ 606 keV). These are the states that are updated in the new reaction rate calculation, shown in Section \ref{sec:pg_reac_rate}. 

The angular distribution of each state is described in detail in the following discussion, where each excited state energy is labeled by its ENSDF \cite{Chen2017} energy for clarity. For proton-unbound states, the resonance energy $E_{r}^{\mathrm{c.m.}}$ is also provided based on the energy calibration of the present analysis. Spectroscopic factors are compared to literature in Section \ref{subsec:DWBA_Spec}, and the proton partial-widths of the states of astrophysical interest are discussed in Section \ref{subsec:partial_widths}.

\newpage

\begin{figure}[H]
\centering
\includegraphics[width=6.5in]{Chapter-6/figs/diffcross_6panel_1.png}
\caption{\label{fig:diffcross_6panel_1}Differential cross sections for 4491 keV through 6582 keV. See text for details.}
\end{figure}

\newpage

\begin{figure}[H]
\centering
\includegraphics[width=6.5in]{Chapter-6/figs/diffcross_6panel_2.png}
\caption{\label{fig:diffcross_6panel_2}Differential cross sections for 6750 keV through 7658 keV. See text for details.}
\end{figure}

\newpage

\begin{figure}[H]
\centering
\includegraphics[width=6.5in]{Chapter-6/figs/diffcross_6panel_3.png}
\caption{\label{fig:diffcross_6panel_3}Differential cross sections for 7694 keV through 8271 keV. See text for details.}
\end{figure}

\newpage

\begin{figure}[H]
\centering
\includegraphics[width=6.5in]{Chapter-6/figs/diffcross_6panel_4.png}
\caption{\label{fig:diffcross_6panel_4}Differential cross sections for 8935 keV through 9227 keV. See text for details.}
\end{figure}

\newpage

\begin{figure}[t]
\centering
%\begin{tikzpicture}
%%\hspace{0.3cm}
%\node at (-4,0) {\includegraphics[width=8.6cm]{Chapter-4/figs/unbound_l1.png}};
%\node at (4,0) {\includegraphics[width=8.6cm]{Chapter-4/figs/unbound_others.png}};
%\end{tikzpicture}
\includegraphics[width=6.5in]{Chapter-6/figs/unbound.png}
\caption{\label{fig:unbound_l}Differential cross-sections of proton-unbound $^{40}\mathrm{Ca}$ states observed in the present experiment below the lowest directly measured $(p,\gamma)$ resonance at 8935 keV. The left panel shows the $l=1$ distributions, while the right panel shows all other distributions.}
\end{figure}

\emph{$\mathbf{E_{x} = 4491}$} \textbf{keV:} \, 
The spin-parity of this state is assigned $J^{\pi} = 5^{-}$ by ENSDF \cite{Chen2017}, and its angular distribution is assigned $\ell=3$ from the present analysis. This is in agreement with all of the $(^{3}\mathrm{He}, d)$ measurements from Refs. \cite{Erskine1966,Seth1967,Forster1970,Cage1971} and the only $(d, n)$ measurement of Ref. \cite{Fuchs1969} in the literature.

\emph{$\mathbf{E_{x} = 5614}$} \textbf{keV:} \, 
The spin-parity of this state is assigned $J^{\pi} = 4^{-}$ by ENSDF. Its angular distribution is assigned $\ell=3$ from the present analysis, in agreement with the literature \cite{Erskine1966,Seth1967,Forster1970,Cage1971,Fuchs1969}.

\emph{$\mathbf{E_{x} = 5903}$} \textbf{keV:} \, 
The spin-parity of this state is assigned $J^{\pi} = 1^{-}$ by ENSDF. Its angular distribution is assigned $\ell=1$ from the present analysis, despite it being obscured by the 4915 keV $^{14}$N contaminant at $\theta_{\mathrm{lab}} = 7^{\circ}$ and $9^{\circ}$. This is in agreement with Refs. \cite{Erskine1966,Seth1967,Cage1971} but in disagreement with $(d,n)$ \cite{Fuchs1969}, where $\ell = 1+3$ is assigned. It was not resolved by Ref. \cite{Forster1970}.

\emph{$\mathbf{E_{x} = 6025}$} \textbf{keV:} \, 
The spin-parity of this state is assigned $J^{\pi} = 2^{-}$ by ENSDF, and its angular distribution is assigned $\ell=1+3$ from the present analysis. This is in agreement with Refs. \cite{Erskine1966,Seth1967} and $(d,n)$ \cite{Fuchs1969}, but in disagreement with Ref. \cite{Cage1971}, where $\ell=3$ is assigned. It was not resolved by Ref. \cite{Forster1970}.

\emph{$\mathbf{E_{x} = 6285}$} \textbf{keV:} \, 
The spin-parity of this state is assigned $J^{\pi} = 3^{-}$ by ENSDF, and its angular distribution is assigned $\ell=1$ from the present analysis. This is in agreement with Refs. \cite{Erskine1966,Forster1970,Cage1971}, but in disagreement with Ref. \cite{Seth1967} and $(d,n)$ \cite{Fuchs1969} where $\ell=1+3$ is assigned.

\emph{$\mathbf{E_{x} = 6582}$} \textbf{keV:} \, 
The spin-parity of this state is assigned $J^{\pi} = 3^{-}$ by ENSDF, and its angular distribution is assigned $\ell=1+3$ from the present analysis, with most of the contribution from $\ell=1$. It is obscured by the $^{13}$N ground state at $\theta_{\mathrm{lab}} \leq 7^{\circ}$. This is in agreement with $(d,n)$ \cite{Fuchs1969}, but in disagreement with $(^{3}\mathrm{He},d)$ \cite{Erskine1966,Seth1967,Forster1970,Cage1971}, which assign it purely $\ell=1$.

\emph{$\mathbf{E_{x} = 6750}$} \textbf{keV:} \, 
The spin-parity of this state is assigned $J^{\pi} = 2^{-}$ by ENSDF, and its angular distribution is assigned $\ell=1+3$ from the present analysis. It is obscured by the $^{13}$N ground state at $\theta_{\mathrm{lab}} = 11^{\circ}, 13^{\circ}$, and $15^{\circ}$. This state is widely disputed in the literature. It is in agreement with $(d,n)$ \cite{Fuchs1969}, but Refs. \cite{Erskine1966,Seth1967,Forster1970} assign it $\ell=1$, and Ref. \cite{Cage1971} assigns it $\ell=3$.

\emph{$\mathbf{E_{x} = 6950}$} \textbf{keV:} \, 
The spin-parity of this state is assigned $J^{\pi} = 1^{-}$ by ENSDF, and its angular distribution is assigned $\ell=1$ from the present analysis. It is obscured by the 5834 keV $^{14}$N state at $\theta_{\mathrm{lab}} = 15^{\circ}$ and $20^{\circ}$. The energy residuals are about 13 keV lower than ENSDF at $\theta_{\mathrm{lab}} = 5^{\circ}$ and $7^{\circ}$, which makes these measurements questionable, but the remaining angles show only a 2 keV residual from ENSDF. Even among the other angles alone, however, the angular distribution is clearly $\ell=1$ and not $\ell=3$. This assignment is in agreement with $(^{3}\mathrm{He},d)$ \cite{Erskine1966,Seth1967,Forster1970,Cage1971}, but in disagreement with $(d,n)$ \cite{Fuchs1969} where $\ell=1+3$ is assigned.

\emph{$\mathbf{E_{x} = 7113}$} \textbf{keV:} \, 
In ENSDF, the 7113.1(10) keV and 7113.73(5) keV states have spin-parities of $J^{\pi} = 1^{-}$ and $J^{\pi} = 4^{-}$, respectively. The present energy calibration yielded a state at 7109.4(29) keV, within the uncertainty of the $1^{-}$ state, but not the $4^{-}$ state. An angular distribution of $\ell=1+3$ is assigned in the present analysis, with the main contribution from $\ell=1$. This is impossible for a $4^{-}$ state, where only $\ell=3$ and/or $\ell=5$ can be observed. Both of these measurements indicate that this corresponds with the 7113.1(10) keV $1^{-}$ state, with possibly a small contribution from the 7113.73(5) keV $4^{-}$ state at $\ell=3$. This assignment is in agreement with $(d,n)$ \cite{Fuchs1969}, but in disagreement with $(^{3}\mathrm{He},d)$ \cite{Erskine1966,Seth1967,Forster1970,Cage1971}, where purely $\ell=1$ is assigned. This was the highest energy state measured by Ref. \cite{Forster1970}, so this reference is omitted from the remaining discussion.

\emph{$\mathbf{E_{x} = 7532}$} \textbf{keV:} \, 
The spin-parity of this state is assigned $J^{\pi} = 2^{-}$ by ENSDF, and its angular distribution is assigned $\ell=1$ from the present analysis. This is in agreement with Refs. \cite{Erskine1966,Cage1971}, but in disagreement with Ref. \cite{Seth1967} and $(d,n)$ \cite{Fuchs1969} where $\ell=1+3$ is assigned.

\emph{$\mathbf{E_{x} = 7623}$} \textbf{keV:} \, 
The spin-parity of this state is assigned $J^{\pi} = (2^{-},3,4^{+})$ by ENSDF and was not resolved by $(^{3}\mathrm{He},d)$ \cite{Erskine1966,Seth1967,Cage1971} or $(d,n)$ \cite{Fuchs1969}. This experiment has resolved this state for the first time through proton-transfer. Its angular distribution is assigned $\ell=1+3$ in the present analysis. This implies a $J^{\pi}$ assignment of $1^{-},2^{-},3^{-}$. Comparing with ENSDF, the $J^{\pi} = 3^{+}$ and $4^{+}$ assignments can be ruled out with this measurement.

\emph{$\mathbf{E_{x} = 7658}$} \textbf{keV:} \, 
The spin-parity of this state is assigned $J^{\pi} = 4^{-}$ by ENSDF, and its angular distribution is assigned $\ell=3$ in the present analysis, in agreement with the literature \cite{Erskine1966,Seth1967,Cage1971,Fuchs1969}.

\emph{$\mathbf{E_{x} = 7694}$} \textbf{keV:} \, 
The spin-parity of this state is assigned $J^{\pi} = 3^{-}$ by ENSDF, and its angular distribution is assigned $\ell=1+3$ in the present analysis. This state is disputed across the literature. The present assignment is in agreement with Ref. \cite{Erskine1966}, but in disagreement with Refs. \cite{Seth1967,Cage1971} and $(d,n)$ \cite{Fuchs1969}, where $\ell=3$ and $\ell=1$ are assigned, respectively.

\emph{$\mathbf{E_{x} = 7973}$} \textbf{keV:} \, 
The spin-parity of this state is assigned $J^{\pi} =$ (LE $3)^{-}$ by ENSDF based only on $(d,n)$ \cite{Fuchs1969}. It was not resolved by $(^{3}\mathrm{He},d)$ \cite{Erskine1966,Seth1967,Cage1971}. This state was obscured in this experiment by the $^{17}$F ground state at $\theta_{\mathrm{lab}} = 11^{\circ}$ and $13^{\circ}$. The present analysis assigns $\ell=1$, in agreement with $(d,n)$. This experiment alone suggests $J^{\pi} = 0^{-}, 1^{-}, 2^{-}, 3^{-}$.

\emph{$\mathbf{E_{x} = 8113}$} \textbf{keV:} \, 
The spin-parity of this state is assigned $J^{\pi} = 1^{-}$ by ENSDF, and it was not resolved by $(^{3}\mathrm{He},d)$ \cite{Erskine1966,Seth1967,Cage1971}. The present analysis assigns $\ell=1$, in disagreement with $(d,n)$ \cite{Fuchs1969}, where $\ell=1+3$ is assigned. However, Ref. \cite{Fuchs1969} observes a doublet at 8113 keV and 8135 keV, which is not observed in the present experiment. The presence of the 6859 keV $^{15}$O contaminant complicates matters because it overlaps with where the 8135 keV state is expected, making this state impossible to resolve. The 7029 keV $^{14}$N and 6793 keV $^{15}$O contaminants obscure the $\theta_{\mathrm{lab}} = 11^{\circ}$ measurement.

\emph{$\mathbf{E_{x} = 8188}$} \textbf{keV} (\emph{$\mathbf{E_{x} = 8196}$} \textbf{keV})\textbf{:} \, 
The spin-parity of the 8188 keV state is assigned $J^{\pi} = (3,4,5^{-})$ in ENSDF, and it was not resolved by $(^{3}\mathrm{He},d)$ \cite{Erskine1966,Seth1967,Cage1971}. No spectroscopic factor was reported from $(d,n)$ \cite{Fuchs1969} for a state observed at 8186 keV, but it was assigned a $\ell=0$ transition nonetheless. This would imply $J^{\pi} = 1^{+},2^{+}$, which conflicts with ENSDF. The present energy calibration found a state at 8199.3(27) keV, or a residual from the 8188 keV state of 11.8(28) keV. There is in fact a state at 8196 keV reported in ENSDF, but it has only been observed in $(p,t)$ \cite{Seth1974,Seth1977} and $(p,p')$ \cite{Nolen1975}, and no official $J^{\pi}$ assignment has been made. However, the $(p,t)$ measurement of Ref. \cite{Seth1977} makes a weak assignment of $\ell=(2)$ for this state, which for $(p,t)$ implies $J^{\pi} = (0,1)^{+}$. In the present experiment, there is ambiguity in the angular momentum between $\ell=0+2$ and $\ell=1$, with the $\ell=0+2$ transition being slightly more favorable. The spin-parities would be $J^{\pi} = 0^{+},1^{+}$ for the former transition and $J^{\pi} = 0^{-},1^{-},2^{-},3^{-}$ for the latter. It is possible that the $(d,n)$ \cite{Fuchs1969} assignment of $\ell=0$ applies to the 8196 keV state, a 10 keV residual, which would be more in alignment with the present measurement and Refs. \cite{Seth1974,Seth1977,Nolen1975}. This may even be likely, considering that Ref. \cite{Fuchs1969} also determined the energy of a nearby state to be 8371 keV, which is interpreted by ENSDF as coinciding with the 8359 keV state, a 12 keV residual. Observation of this state in the present experiment was obscured at $\theta_{\mathrm{lab}} = 11^{\circ}$ by the 6859 keV $^{15}$O state and at $15^{\circ}$ by the 7029 keV $^{14}$N and 6793 keV $^{15}$O states.

\emph{$\mathbf{E_{x} = 8271}$} \textbf{keV:} \, 
The spin-parity of this state is assigned $J^{\pi} =$ (LE $3)^{-}$ by ENSDF, and it was not resolved by $(^{3}\mathrm{He},d)$ \cite{Erskine1966,Seth1967,Cage1971}. The present analysis assigns $\ell=1$, in agreement with $(d,n)$ \cite{Fuchs1969}. These experiments alone suggest $J^{\pi} = 0^{-},1^{-},2^{-},3^{-}$.

\emph{$\mathbf{E_{x} = 8359}$} \textbf{keV}; \emph{$\mathbf{E^{\boldsymbol{\mathrm{c.m.}}}_{r} = 29}$} \textbf{keV:} \, 
This is the lowest energy proton-unbound state observed in the present experiment. The spin-parity of this state is assigned $J^{\pi} = (0, 1, 2)^{-}$ in ENSDF, and it was not resolved by $(^{3}\mathrm{He},d)$ \cite{Erskine1966,Seth1967,Cage1971}. The present analysis assigns $\ell=1$, in agreement with $(d,n)$ \cite{Fuchs1969}. The recent  $^{39}\mathrm{K}(p, \gamma)^{40}\mathrm{Ca}$ reaction rate evaluation from Ref. \cite{Longland2018} did not use the $(d,n)$ \cite{Fuchs1969} $C^{2}S$ measurement for a state at 8371 keV in their analysis because of the 12 keV energy discrepancy with ENSDF. This state was instead assigned a proton partial-width upper limit in the reaction rate calculation which the present work now replaces.

\emph{$\mathbf{E_{x} = 8425}$} \textbf{keV}; \emph{$\mathbf{E^{\boldsymbol{\mathrm{c.m.}}}_{r} = 96}$} \textbf{keV:} \, 
The spin-parity of this state is assigned $J^{\pi} = 2^{-}$ in ENSDF, and it has also since been confirmed via NRF \cite{Gribble2022}. The present work assigns $\ell=1+3$, in agreement with $(d, n)$ \cite{Fuchs1969}, but in disagreement with $(^{3}\mathrm{He}, d)$ \cite{Erskine1966,Seth1967,Cage1971}, where $\ell=3$ is assigned. Ref. \cite{Longland2018} uses the $(^{3}\mathrm{He},d)$ $C^{2}S$ measurement of Ref. \cite{Cage1971} in their reaction rate calculation. This state was obscured in the present experiment at $\theta_{\mathrm{lab}} = 5^{\circ}$ and $7^{\circ}$ by the $^{17}$F first excited state.

\emph{$\mathbf{E_{x} = 8484}$} \textbf{keV}; \emph{$\mathbf{E^{\boldsymbol{\mathrm{c.m.}}}_{r} = 154}$} \textbf{keV:} \, 
This weakly-populated state has not been observed in $(^{3}\mathrm{He}, d)$ \cite{Erskine1966,Seth1967,Cage1971} or $(d, n)$ \cite{Fuchs1969}. It has therefore been resolved for the first time in $^{39}\mathrm{K}+p$ from this experiment because the lowest energy resonance resolved by $(p,\gamma)$ \cite{Kikstra1990,Cheng1981,Leenhouts1966} is for $E_{r}^{\mathrm{c.m.}} = 606$ keV. Its spin-parity is assigned $J^{\pi} = (1^{-},2^{-},3^{-})$ in ENSDF. The present work assigns $\ell=1$ in support of the ENSDF $J^{\pi}$ assignment. This measurement replaces the proton partial-width upper limit of Ref. \cite{Longland2018} in the reaction rate calculation. This state was obscured in the present experiment at $\theta_{\mathrm{lab}} = 13^{\circ}$ by the $^{17}$F first excited state.

\emph{$\mathbf{E_{x} = 8551}$} \textbf{keV}; \emph{$\mathbf{E^{\boldsymbol{\mathrm{c.m.}}}_{r} = 221}$} \textbf{keV:} \, 
The spin-parity of this state is assigned $J^{\pi} = 5^{-}$ in ENSDF. The present work assigns $\ell=3$, in agreement with $(^{3}\mathrm{He}, d)$ \cite{Erskine1966,Seth1967,Cage1971} and $(d, n)$ \cite{Fuchs1969}.

\emph{$\mathbf{E_{x} = 8665}$} \textbf{keV}; \emph{$\mathbf{E^{\boldsymbol{\mathrm{c.m.}}}_{r} = 335}$} \textbf{keV:} \, 
The spin-parity of this state is assigned $J^{\pi} = 1^{-}$ in ENSDF, and it was not resolved by $(^{3}\mathrm{He},d)$ \cite{Erskine1966,Seth1967,Cage1971}. The present work assigns $\ell=1$ in agreement with $(d, n)$ \cite{Fuchs1969}.

\emph{$\mathbf{E_{x} = 8748}$} \textbf{keV}; \emph{$\mathbf{E^{\boldsymbol{\mathrm{c.m.}}}_{r} = 415}$} \textbf{keV:} \, 
This weakly-populated state was observed in $(d, n)$ \cite{Fuchs1969} as part of an unresolved doublet with the 8764 keV state, but no $\ell$ assignments were made. Neither state was resolved in $(^{3}\mathrm{He},d)$ \cite{Erskine1966,Seth1967,Cage1971} or $(p,\gamma)$ \cite{Kikstra1990,Cheng1981,Leenhouts1966}, but they have been resolved in the present experiment, making this the first time they have been resolved in $^{39}\mathrm{K}+p$ altogether. The spin-parity of the 8748 keV state is assigned $J^{\pi} = 2^{+}$ in ENSDF and has since been confirmed by NRF \cite{Gribble2022}. The present work assigns $\ell=2$ in support of the ENSDF $J^{\pi}$ assignment. This measurement replaces the proton partial-width upper limit of Ref. \cite{Longland2018} in the reaction rate calculation. The statistics were not sufficient to obtain fits for $\theta_{\mathrm{lab}} = 13^{\circ}$ for this state.

\emph{$\mathbf{E_{x} = 8764}$} \textbf{keV}; \emph{$\mathbf{E^{\boldsymbol{\mathrm{c.m.}}}_{r} = 439}$} \textbf{keV:} \, 
This state is part of a doublet with 8748 keV and was not resolved in $(^{3}\mathrm{He},d)$ \cite{Erskine1966,Seth1967,Cage1971}, $(d, n)$ \cite{Fuchs1969}, or $(p,\gamma)$ \cite{Kikstra1990,Cheng1981,Leenhouts1966}. Its spin-parity is assigned $J^{\pi} = 3^{-}$ in ENSDF, and the present work assigns $\ell=1$, in support of this assignment. This experiment is therefore the first to resolve this state for $^{39}\mathrm{K}+p$, and its measurement replaces the proton partial-width upper limit of Ref. \cite{Longland2018} in the reaction rate calculation.

\emph{$\mathbf{E_{x} = 8851}$} \textbf{keV}; \emph{$\mathbf{E^{\boldsymbol{\mathrm{c.m.}}}_{r} = 521}$} \textbf{keV:} \, 
The spin-parity of this state is assigned $J^{\pi} = 6^{-}, 7^{-}, 8^{-}$ in ENSDF based on $(p, p')$ \cite{Gruhn1972}. It was observed in $(d, n)$ \cite{Fuchs1969} with no $\ell$ assignment determined, and it was not resolved by $(^{3}\mathrm{He},d)$ \cite{Erskine1966,Seth1967,Cage1971} or $(p,\gamma)$ \cite{Kikstra1990,Cheng1981,Leenhouts1966}. The present experiment assigns $\ell=1+3$, making its resolution a first for $^{39}\mathrm{K}+p$. The $\ell$ assignment suggests $J^{\pi} = (1, 2, 3)^{-}$, in stark disagreement with ENSDF. The upper limit from Ref. \cite{Longland2018} for this state uses the ENSDF $J^{\pi}$ assignment in the reaction rate calculation. For this reason, the $(2J+1)\Gamma_{p}$ value from this work is about 2 orders of magnitude larger than the upper limit, as shown in Table \ref{tab:partial_widths}.

\emph{$\mathbf{E_{x} = 8935}$} \textbf{keV}; \emph{$\mathbf{E^{\boldsymbol{\mathrm{c.m.}}}_{r} = 608}$} \textbf{keV:} \, 
In ENSDF, the 8934.81(7) keV and 8938.4(9) keV states have spin-parities of $J^{\pi} = 2^{+}$ and $J^{\pi} = 0^{+}$, respectively. The 8934.81(7) keV state is the lowest energy proton-unbound state measured by the direct $(p,\gamma)$ \cite{Kikstra1990} experiment, and they do not observe the 8938.4(9) keV state. The present energy calibration yielded a state at 8936.3(27) keV, within the uncertainty of both states. Neither state is resolved by $(^{3}\mathrm{He},d)$ \cite{Erskine1966,Seth1967,Cage1971}. The present analysis assigns $\ell=1$, in agreement with the 8931 keV assignment from $(d,n)$ \cite{Fuchs1969}, and this joins Ref. \cite{Fuchs1969} in disagreement with ENSDF. These measurements imply $J^{\pi} = 0^{-},1^{-},2^{-},3^{-}$. It is worth noting that a $\ell=0+2$ assignment was considered in the present analysis, which would imply $J^{\pi} = 1^{+},2^{+}$, in agreement with $(p,\gamma)$ \cite{Kikstra1990}. However, the $\ell=1$ transition is significantly more likely.

\emph{$\mathbf{E_{x} = 8995}$} \textbf{keV}; \emph{$\mathbf{E^{\boldsymbol{\mathrm{c.m.}}}_{r} = 664}$} \textbf{keV:} \, 
The spin-parity of this state is assigned $J^{\pi} = (1^{-},2^{+})$ in ENSDF, and it was not resolved by $(^{3}\mathrm{He},d)$ \cite{Erskine1966,Seth1967,Cage1971} or $(d,n)$ \cite{Fuchs1969}. The angular momentum assignment in the present analysis is ambiguous between $\ell=0+2$ and $\ell=1$, which does not narrow down the $J^{\pi}$ assignment. Higher angle measurements of $\theta_{\mathrm{lab}} \geq 30^{\circ}$ could resolve the angular momentum. The state was obscured at $\theta_{\mathrm{lab}} = 5^{\circ}$ by the $^{13}$N first excited state, and the statistics were insufficient at $\theta_{\mathrm{lab}} = 20^{\circ}$.

\emph{$\mathbf{E_{x} = 9092}$} \textbf{keV}; \emph{$\mathbf{E^{\boldsymbol{\mathrm{c.m.}}}_{r} = 764}$} \textbf{keV:} \, 
The spin-parity of this state is assigned $J^{\pi} = 3^{-}$ in ENSDF, and it was not resolved by $(^{3}\mathrm{He},d)$ \cite{Erskine1966,Seth1967,Cage1971} or $(d,n)$ \cite{Fuchs1969}. The present analysis assigns $\ell=1$, despite the state being obscured at $\theta_{\mathrm{lab}} = 5$--$11^{\circ}$ by the $^{13}$N first excited state. This is in agreement with the $J^{\pi}$ assignment.

\emph{$\mathbf{E_{x} = 9136}$} \textbf{keV}; \emph{$\mathbf{E^{\boldsymbol{\mathrm{c.m.}}}_{r} = 810}$} \textbf{keV:} \, 
The spin-parity of this state is assigned $J^{\pi} = 2^{-},3^{-}$ in ENSDF, and it was not resolved by $(^{3}\mathrm{He},d)$ \cite{Erskine1966,Seth1967,Cage1971}. The present analysis assigns $\ell=1$, in agreement with $(d,n)$ \cite{Fuchs1969}, despite the state being obscured at $\theta_{\mathrm{lab}} = 5$--$11^{\circ}$ by the $^{13}$N first excited state. This is in agreement with the $J^{\pi}$ assignment.

\emph{$\mathbf{E_{x} = 9227}$} \textbf{keV}; \emph{$\mathbf{E^{\boldsymbol{\mathrm{c.m.}}}_{r} = 900}$} \textbf{keV:} \, 
In ENSDF, the 9226.69(5) keV and 9227.43(7) keV states have spin-parities of $J^{\pi} = (1^{-},2,3^{-})$ and $J^{\pi} = (1,2^{+})$, respectively. The present energy calibration yielded a state at 9227.8(34) keV, within the uncertainty of both states. Neither state is resolved by $(^{3}\mathrm{He},d)$ \cite{Erskine1966,Seth1967,Cage1971}. The angular momentum assignment in the present analysis is ambiguous between $\ell=0+2$ and $\ell=1$, which does not narrow down the $J^{\pi}$ assignment in either case. The $(d,n)$ measurement of Ref. \cite{Fuchs1969} unambiguously assigns $\ell=1$. The state was obscured at $\theta_{\mathrm{lab}} = 13^{\circ}$ and $15^{\circ}$ by the $^{13}$N first excited state.

\begin{figure}[t]
\centering
\includegraphics[width=6.5in]{Chapter-6/figs/5deg_9400keVGroup.png}
\caption{\label{fig:5deg_9400keVGroup}A Bayesian Monte Carlo fit for the 9405--9432 keV multiplet at $\theta_{\mathrm{lab}} = 5^{\circ}$, including the clearly distinguishable 9454 keV state. The multiplet is fit with 3 peaks in this experiment, with calibrated energies of 9403.7(31) keV, 9416.7(32) keV, and 9431.7(32) keV. This is better resolution than $(d,n)$ \cite{Fuchs1969} was able to achieve, where only 2 peaks in the multiplet were fit at energies of 9408 keV and 9431 keV.}
\end{figure}

\emph{$\mathbf{E_{x} = 9405-9432}$} \textbf{keV Multiplet:} \, 
This multiplet consists of the states 9405 keV ($J^{\pi} = 2^{-}$), 9406 keV ($J^{\pi} = 0^{+}$), 9412 keV (no $J^{\pi}$ assignment), 9419 keV ($J^{\pi} = 3^{-}$), 9429 keV ($J^{\pi} = (3,4)^{-}$), and 9432 keV ($J^{\pi} = 1^{-}$) in ENSDF. Even though these individual states are unresolvable at the energy resolution attainable with the focal plane detector, it was possible to consistently fit 3 peaks in this region with the Bayesian Monte Carlo procedure of Section \ref{sec:peak_fitting}, as shown in Fig. \ref{fig:5deg_9400keVGroup}. From the present energy calibration, the peaks result in energies of 9403.7(31) keV, 9416.7(32) keV, and 9431.7(32) keV. These can likely be assigned as 3 separate doublets from the 6 ENSDF states, 9405+9406 keV, 9412+9419 keV, and 9429+9432 keV. The $(d,n)$ measurement of Ref. \cite{Fuchs1969} obtained fits for only 2 peaks in this region, at 9408 keV and 9431 keV. Attempting fits with 2 peaks in this region resulted in much worse agreement with the present data. Angular distributions were not obtained for these states in the present analysis, or for any higher energy states, since the focal plane detector efficiency at the high energy fringe resulted in dramatic yield loss over increasing angle.

To summarize the results of this section, the present experiment has unambiguously resolved 13 new states in $^{39}\mathrm{K}(^{3}\mathrm{He},d)^{40}\mathrm{Ca}$, 6 new states in $^{39}$K proton-transfer, and 4 new states in $^{39}\mathrm{K}+p$ altogether. Table \ref{tab:new_states} lists the states that are newly resolved in these regimes. States in parentheses were given an ambiguous $\ell$ assignment in the present experiment.

\begin{table}[t]
\centering
\caption{\label{tab:new_states}The states resolved in the present experiment for the first time in the indicated regimes. Parentheses denote ambiguous $\ell$ assignments.}
\begin{tabular}{ccc}
\hline\midrule
New for $^{39}\mathrm{K}(^{3}\mathrm{He},d)^{40}\mathrm{Ca}$&New for $^{39}$K Proton-Transfer&New for $^{39}\mathrm{K}+p$\\ \midrule
7623 keV&7623 keV&\\
7973 keV&&\\
8113 keV&&\\
(8188 keV)&(8188 keV)&\\
8271 keV&&\\
8359 keV&&\\
8484 keV&8484 keV&8484 keV\\
8665 keV&&\\
8748 keV&8748 keV&8748 keV\\
8764 keV&8764 keV&8764 keV\\
8851 keV&8851 keV&8851 keV\\
8935 keV&&\\
(8995 keV)&(8995 keV)&\\
9092 keV&9092 keV&\\
9136 keV&&\\
(9227 keV)&&\\
\hline\hline
\end{tabular}
\end{table}

\subsection{Spectroscopic Factors} \label{subsec:DWBA_Spec}

Spectroscopic factors $C^{2}S_{^{39}\mathrm{K} + p}$ for each state were extracted from Eqns. \ref{eqn:spec_factor_dwba} and \ref{eqn:mixed_l}, using the angular momentum assignments and DWBA calculations from the previous section. These are summarized and compared with the $(^{3}\mathrm{He},d)$ \cite{Erskine1966,Seth1967,Forster1970,Cage1971} and $(d,n)$ \cite{Fuchs1969} measurements in the literature in Table \ref{tab:spec}. Doublets unresolved in the present analysis are given brackets around their ENSDF \cite{Chen2017} energies. Ambiguous $\ell$ assignments are also indicated with brackets. In mixed-$\ell$ or disputed-$\ell$ cases, each $\ell$-component is shown in parentheses on separate rows, where the lowest $\ell$ is always the first row. 

The $J^{\pi}$ values shown in the table were used in the present DWBA models to calculate $C^{2}S$. They were also used to convert the other measurements from their reported $(2J+1)C^{2}S$ values to $C^{2}S$ for consistency, since their chosen $J^{\pi}$ sometimes differed. The $J^{\pi}$ values that differ from ENDSF are indicated. This difference is usually an arbitrary choice of the lowest appropriate ENSDF $J^{\pi}$ value that is consistent with the present $\ell$-assignment. However, the choice sometimes genuinely conflicts with the ENSDF assignment, as is the case for the 8188 keV $1^{+}$ ambiguity, 8851 keV, and the 8935+8938 keV doublet.

The statistical uncertainty of $C^{2}S$ for the present experiment is also given in the table. An additional $30\%$ systematic uncertainty in $C^{2}S$ is assumed, which is typical of $(^{3}\mathrm{He},d)$ reactions in the literature \cite{Endt1977}. The main contribution to this systematic uncertainty is the choice of OMP parameters, especially for the $^{3}\mathrm{He} + {}^{39}\mathrm{K}$ incoming channel. This is due to the fact that the $^{39}\mathrm{K}(^{3}\mathrm{He},d)^{40}\mathrm{Ca}$ differential cross section was normalized to the $^{3}\mathrm{He} + {}^{39}\mathrm{K}$ OMP prediction. The OMP parameters for the $d + {}^{40}\mathrm{Ca}$ outgoing channel also contribute to the systematic uncertainty, albeit to a lesser degree. For example, the $^{23}\mathrm{Na}(^{3}\mathrm{He},d)^{24}\mathrm{Mg}$ measurement of Ref. \cite{Hale2004} varied the choice of OMP parameters within a reasonable range. They found a $26\%$ uncertainty associated with the $^{3}$He parameters alone and a $11\%$ uncertainty associated with the $d$ parameters alone. Added in quadrature, this is a $28\%$ uncertainty from the choice of OMP parameters, which agrees with the historical scatter between $(^{3}\mathrm{He},d)$ measurements of $C^{2}S$ \cite{Endt1977}.

%From Table \ref{tab:spec},

%In all but one case where a single $J^{\pi}$ value is given, it was used for the associated $C^{2}S$ calculation and the determination of the literature $C^{2}S$ values from $(2J+1)C^{2}S$. The one exception is the 8935+8938 keV doublet, where $J^{\pi}=2^{+}$ and $J^{\pi}=0^{+}$ according to ENSDF, respectively. This was observed as a $\ell=1$ transition in the present experiment, in agreement with $(d,n)$ \cite{Fuchs1969}, and $J^{\pi}=1^{-}$ was therefore used instead.

%\newpage

\begingroup % This and the following line are used to get table footnote marks to be alphabetic, but do this locally (not globally)
  \renewcommand*{\thefootnote}{\alph{footnote}}
  \renewcommand*\footnoterule{} % Get rid of the footnote rule
\begin{table}[!tp]
\centering
\begin{minipage}{\textwidth}
\centering
%\caption{\label{tab:spec}Spectroscopic factors of the present experiment, compared with that of the other $(^{3}\mathrm{He},d)$ \cite{Erskine1966,Seth1967,Forster1970,Cage1971} and $(d,n)$ \cite{Fuchs1969} measurements in the literature.}
\caption{\label{tab:spec}Spectroscopic factors compared with Refs. \cite{Erskine1966,Seth1967,Forster1970,Cage1971,Fuchs1969}.}
\resizebox{\linewidth}{!}{%
%\hspace{0.5cm}
% - 0.25cm
%\hrule width \hsize \kern 1mm \hrule width \hsize height 0.4pt
%\vspace{0.1cm}
\begin{tabular}{cllllllll}
\hline\midrule
$E_{x}$\footnotemark[1]&$J^{\pi}$\footnotemark[2]&$\ell$\footnotemark[2]&$C^{2}S_{^{39}\mathrm{K} + p}$\footnotemark[2]&\phantom{<}\cite{Erskine1966}&\phantom{<}\cite{Seth1967}&\cite{Forster1970}&\cite{Cage1971}\footnotemark[3]&\phantom{<}\cite{Fuchs1969}\\ \midrule
4491&$5^{-}$&$3$&0.454(25)&\phantom{<}0.609&\phantom{<}0.382&0.364&0.418&\phantom{<}0.340\\
5614&$4^{-}$&$3$&0.461(25)&\phantom{<}0.656&\phantom{<}0.389&0.378&0.389&\phantom{<}0.470\\
5903&$1^{-}$&$1$&0.019(1)&\phantom{<}0.037&\phantom{<}0.025&&0.015&\phantom{<}0.027 ($1$)\\
&&&&&&&&<0.067 ($3$)\\
6025&$2^{-}$&$1+3$&0.018(1) \phantom{a}($1$)&\phantom{<}0.027 ($1$)&\phantom{<}0.020 ($1$)&&&\phantom{<}0.030 ($1$)\\
&&&0.072(6) \phantom{a}($3$)&\phantom{<}0.180 ($3$)&\phantom{<}0.100 ($3$)&&0.105&\phantom{<}0.096 ($3$)\\
6285&$3^{-}$&$1$&0.167(9)&\phantom{<}0.357&\phantom{<}0.200 ($1$)&0.307&0.150&\phantom{<}0.246 ($1$)\\
&&&&&<0.050 ($3$)&&&<0.171 ($3$)\\
6582&$3^{-}$&$1+3$&0.047(3) \phantom{a}($1$)&\phantom{<}0.114&\phantom{<}0.086&0.107&0.060&\phantom{<}0.080 ($1$)\\
&&&0.069(9) \phantom{a}($3$)&&&&&<0.114 ($3$)\\
6750&$2^{-}$&$1+3$&0.024(2) \phantom{a}($1$)&\phantom{<}0.100&\phantom{<}0.096&0.094&&\phantom{<}0.027 ($1$)\\
&&&0.135(11) ($3$)&&&&0.190&\phantom{<}0.264 ($3$)\\
6950&$1^{-}$&$1$&0.134(8)&\phantom{<}0.275&\phantom{<}0.200&0.215&0.080&\phantom{<}0.227 ($1$)\\
&&&&&&&&<0.267 ($3$)\\
7113&$1^{-}$&$1+3$&0.150(8) \phantom{a}($1$)&\phantom{<}0.317&\phantom{<}0.257&0.233&0.257&\phantom{<}0.240 ($1$)\\
&&&0.132(23) ($3$)&&&&&<0.133 ($3$)\\
7532&$2^{-}$&$1$&0.227(12)&\phantom{<}0.436&\phantom{<}0.150 ($1$)&&0.095&\phantom{<}0.392 ($1$)\\
&&&&&\phantom{<}0.400 ($3$)&&&<0.080 ($3$)\\
7623&$2^{-}$\footnotemark[4]&$1+3$&0.005(1) \phantom{a}($1$)&&&&&\\
&&&0.037(3) \phantom{a}($3$)&&&&&\\
7658&$4^{-}$&$3$&0.375(20)&\phantom{<}0.483&\phantom{<}0.306&&0.306&\phantom{<}0.345\\
7694&$3^{-}$&$1+3$&0.027(4) \phantom{a}($1$)&<0.057 ($1$)&&&&\phantom{<}0.029\\
&&&0.279(20) ($3$)&\phantom{<}0.584 ($3$)&\phantom{<}0.414&&0.393&\\
7973&$3^{-}$&$1$&0.017(1)&&&&&\phantom{<}0.023\\
8113&$1^{-}$&$1$&0.026(2)&&&&&\phantom{<}0.033 ($1$)\\
&&&&&&&&\phantom{<}0.160 ($3$)\\
8188&$3^{-}$\footnotemark[4]&\hspace{-3mm}\ldelim \{ {3}{3mm}$1$&0.007(1)&&&&&\\
&$1^{+}$\footnotemark[5]&$0+2$&0.017(2) \phantom{a}($0$)&&&&&\\ %l=0 component
&&&0.006(1) \phantom{a}($2$)&&&&&\\ %l=2 component
8271&$3^{-}$&$1$&0.031(2)&&&&&\phantom{<}0.046\\
8359&$0^{-}$\footnotemark[4]&$1$&0.191(11)&&&&&\phantom{<}0.320\\
8425&$2^{-}$&$1+3$&0.037(6) \phantom{a}($1$)&&&&&\phantom{<}0.008 ($1$)\\
&&&0.246(21) ($3$)&\phantom{<}0.320&\phantom{<}0.500&&0.280&\phantom{<}0.288 ($3$)\\
8484&$1^{-}$\footnotemark[4]&$1$&0.031(2)&&&&&\\
8551&$5^{-}$&$3$&0.416(23)&\phantom{<}0.591&\phantom{<}0.455&&0.418&\phantom{<}0.356\\
8665&$1^{-}$&$1$&0.148(8)&&&&&\phantom{<}0.187\\
8748&$2^{+}$&$2$&0.004(1)&&&&&\\
8764&$3^{-}$&$1$&0.005(1)&&&&&\\
8851&$2^{-}$\footnotemark[5]&$1+3$&0.011(1) \phantom{a}($1$)&&&&&\\
&&&0.011(1) \phantom{a}($3$)&&&&&\\
\hspace{-3mm}\ldelim \{ {1.8}{3mm}8935&$1^{-}$\footnotemark[5]&$1$&0.057(3)&&&&&\phantom{<}0.147\\
8938&&&&&&&&\\
%\hline
\end{tabular}%
}
%\hrule width \hsize \kern 1mm \hrule width \hsize height 0.4pt
\vspace{-0.2cm}

\end{minipage}
\end{table}
\endgroup

\newpage

\addtocounter{table}{-1} % reset table numbering so that the 2nd table has the same counter number
\begingroup % This and the following line are used to get table footnote marks to be alphabetic, but do this locally (not globally)
  \renewcommand*{\thefootnote}{\alph{footnote}}
  \renewcommand*\footnoterule{} % Get rid of the footnote rule
\begin{table}[!tp]
\centering
\begin{minipage}{\textwidth}
\centering
%\caption{\label{tab:spec}Spectroscopic factors of the present experiment, compared with that of the other $(^{3}\mathrm{He},d)$ \cite{Erskine1966,Seth1967,Forster1970,Cage1971} and $(d,n)$ \cite{Fuchs1969} measurements in the literature.}
\caption{(continued)}
\resizebox{\linewidth}{!}{%
%\hspace{0.5cm}
% - 0.25cm
%\hrule width \hsize \kern 1mm \hrule width \hsize height 0.4pt
%\vspace{0.1cm}
\begin{tabular}{cllllllll}
\hline\midrule
$E_{x}$\footnotemark[1]&$J^{\pi}$\footnotemark[2]&$\ell$\footnotemark[2]&$C^{2}S_{^{39}\mathrm{K} + p}$\footnotemark[2]&\phantom{<}\cite{Erskine1966}&\phantom{<}\cite{Seth1967}&\cite{Forster1970}&\cite{Cage1971}\footnotemark[3]&\phantom{<}\cite{Fuchs1969}\\ \midrule
8995&$1^{-}$&\hspace{-3mm}\ldelim \{ {3}{3mm}$1$&0.020(1)&&&&&\\
&$2^{+}$&$0+2$&0.006(1) \phantom{a}($0$)&&&&&\\ %l=0 component
&&&0.006(1) \phantom{a}($2$)&&&&&\\ %l=2 component
9092&$3^{-}$&$1$&0.007(1)&&&&&\\
9136&$2^{-}$\footnotemark[4]&$1$&0.152(9)&&&&&\phantom{<}0.136\\
\hspace{-3mm}\ldelim \{ {1.8}{3mm}9226.69&$1^{-}$\footnotemark[4]&\hspace{-3mm}\ldelim \{ {3}{3mm}$1$&0.143(9)&&&&&\phantom{<}0.213\\
9227.43&$1^{+}$\footnotemark[4]&$0+2$&0.048(6) \phantom{a}($0$)&&&&&\\ %l=0 component
&&&0.071(6) \phantom{a}($2$)&&&&&\\ %l=2 component
%\hspace{-3mm}\ldelim \{ {1.8}{3mm}9404.85(19)&9403.7(31)&$2^{-}$\\
%9406.3(6)&&$0^{+}$\\
%\hspace{-3mm}\ldelim \{ {1.8}{3mm}9412.3(2)&9416.7(32)&\\
%9418.8(2)&&$3^{-}$\\
%\hspace{-3mm}\ldelim \{ {1.8}{3mm}9429.11(5)&9431.7(32)&$(3,4)^{-}$\\
%9432.46(18)&&$1^{-}$\\
%9453.95(5)&\emph{9454}\footnotemark[3]&$3^{-}$\\
%9537.8(5)&9538.4(34)&$1^{-}$\\
%\hspace{-3mm}\ldelim \{ {1.8}{3mm}9603.0(4)&9605.1(40)&$3^{-}$\\
%9604.6(4)&&$1^{-}$\\
% NOTE: Not including the 9662+9669 keV doublet's energy, since it was only extracted for 2 angles
%\hspace{-3mm}\ldelim \{ {2}{3mm}9662.2(2)&9673.4(31)&1345.2(31)&$3^{-}$\footnotemark[3]& & & \\
%9668.71(8)& & &$3^{-}$\footnotemark[3]& & &$1.73 \times 10^{4}$ \\
%\vspace{0.0001cm}
\hline\hline
\end{tabular}%
}
%\hrule width \hsize \kern 1mm \hrule width \hsize height 0.4pt
\vspace{-0.2cm}
\footnotetext[1]{From the ENSDF evaluation of Ref. \cite{Chen2017}. Energy is given in keV.}
\footnotetext[2]{Present experiment. Statistical uncertainties are given. A 30$\%$ systematic uncertainty is assumed.}
\footnotetext[3]{Using the modified DWBA results of Ref. \cite{Cage1971}.}
\footnotetext[4]{The lowest appropriate $J^{\pi}$ from the ENSDF assignment possibilities.}
\footnotetext[5]{Differs from the ENSDF $J^{\pi}$ assignment altogether.}
\end{minipage}
\end{table}
\endgroup

\subsection{Proton Partial Widths and Resonance Energies} \label{subsec:partial_widths}

The proton partial widths $\Gamma_{p}$ are small for the low proton energies $E_{r}$ considered in this work, $\Gamma_{p} << E_{r}$. Therefore, the $^{39}\mathrm{K}(p,\gamma)^{40}\mathrm{Ca}$ cross-section is considered to be dominated by narrow resonances, and the reaction rate can be calculated with Eqn. \ref{eqn:narrow_rate}. The nonresonant contributions and interference effects are expected to be negligible for this reaction \cite{Longland2018}. Proton partial widths are calculated using Eqn. \ref{eqn:partial_width}, given the spectroscopic factors $C^{2}S$ obtained in the previous section. The penetration factor $P(E)$ is easily computed from Eqn. \ref{eqn:pen_factor} once the center-of-mass resonance energies $E^{\mathrm{c.m.}}_{r}$ have been determined. These energies are calculated from the measured excitation energies $E_{x}$ in this work using the $^{40}$Ca proton separation energy $S_{p} = 8328.18(2)$ keV from Ref. \cite{Wang2021}. 

The final factor to consider in the $\Gamma_{p}$ calculation is the dimensionless single particle reduced width $\theta_{\mathrm{sp}}$, calculated from Eqn. \ref{eqn:single_particle_red_width} as $\theta_{\mathrm{sp}} = R \, |u_{\mathrm{sp}}(R)|^{2}/2$. The channel number $R$ is defined as
\begin{equation}
R = r_{0} (A^{1/3}_{t} + A^{1/3}_{p}) \, \mathrm{fm},
\end{equation}
where the standard choice of $r_{0} = 1.25$ fm is made here, $A_{t}$ is the target $(^{39}\mathrm{K})$ mass, and $A_{p}$ is the proton mass. $u_{\mathrm{sp}}$ is the single-particle radial wavefunction. As mentioned in Section \ref{subsec:particle_partial_widths}, we use the weak binding approximation for this analysis, which assumes that unbound states are weakly bound by $\lesssim$ 50 keV. The channel radius $R$ is chosen such that the wavefunctions for the bound state and unbound state at $R$ are equivalent \cite{Harrouz2023}. Therefore, $u_{\mathrm{sp}}(R)$ can be treated as the bound state radial wavefunction evaluated at $R$, which can be obtained from \texttt{Fresco}. Both the spectroscopic factor $C^{2}S$ and $\theta_{\mathrm{sp}}$ are obtained for unbound states using the weak binding approximation. It has been shown that $C^{2}S$ decreases with binding energy at roughly the same rate as $\theta_{\mathrm{sp}}$ increases \cite{Harrouz2023}, making their product largely independent of energy. Therefore, no extrapolation to specific resonance energies is required.

The systematic uncertainty of $\Gamma_{p}$ has been extensively studied in other works (e.g. Ref. \cite{Hale2004}). The largest contributor to this uncertainty comes from the choice of OMP parameters and is inherited from $C^{2}S$. The systematic uncertainty from the bound state parameters is expected to be low for $\Gamma_{p}$, since the two factors $C^{2}S$ and $\theta_{\mathrm{sp}}$ have opposing bound-state potential dependence. Hence, to a good approximation, the systematic uncertainty of $\Gamma_{p}$ can be treated as being derived entirely from $C^{2}S$. Therefore, the choice of a $30\%$ systematic uncertainty applies in this work to both $C^{2}S$ and $\Gamma_{p}$.

\begingroup % This and the following line are used to get table footnote marks to be alphabetic, but do this locally (not globally)
  \renewcommand*{\thefootnote}{\alph{footnote}}
  \renewcommand*\footnoterule{} % Get rid of the footnote rule

\begin{table}[t]
\centering
\begin{minipage}{\textwidth}
\centering
\caption{\label{tab:partial_widths}Center-of-mass resonance energies $E_{r}$, angular momentum assignments, and proton partial-widths obtained for proton-unbound $^{40}\mathrm{Ca}$ states in this experiment. The total proton partial-width uncertainty is reported, including a $30\%$ systematic uncertainty. The last column shows proton partial-width upper limits calculated by Ref. \cite{Longland2018}.}
% ...using the $J^{\pi}$ assignments from ENSDF.} <--- Not true
\resizebox{\linewidth}{!}{%
\begin{tabular}{llllllc}
\hline\midrule
%$E_{x}$ (ENSDF)&$E_{x}$\footnotemark[1]&$E_{r}$\footnotemark[1]&$J^{\pi}$ (ENSDF)&$J^{\pi}$\footnotemark[1]&$l$\footnotemark[1]&$(2J+1)\Gamma_{p}$\footnotemark[1]&$(2J+1)\Gamma_{p, UL}$\\ \midrule
$E_{x}$\footnotemark[1] [keV]&$E_{r}$\footnotemark[2] [keV]&$J^{\pi}$\footnotemark[1]&$J^{\pi}$\footnotemark[2]&$l$\footnotemark[2]&$(2J+1)\Gamma_{p}$\footnotemark[2] [eV]&$(2J+1)\Gamma_{p, UL}$ [eV]\\ \midrule
8358.9(6)&29.1(25)&$(0,1,2)^{-}$&$(0,1,2,3)^{-}$&1&$2.62(80) \times 10^{-38}$&$5.88 \times 10^{-36}$\\
8424.35(31)\footnotemark[3]\footnotemark[4]& \emph{96.17(31)}\footnotemark[3] &$2^{-}$&$(1,2,3)^{-}$&$1 + 3$&$5.4(18) \times 10^{-17} \phantom{a}(1)$&$1.47 \times 10^{-15}$\\
 & & & & &$3.9(12) \times 10^{-19} \phantom{a}(3)$& \\
8484.02(13)\footnotemark[4]&154.1(29)&$(1^{-},2^{-},3^{-})$&$(0,1,2,3)^{-}$&1&$8.1(25) \times 10^{-12}$&$1.16 \times 10^{-9}$\\
8551.1(7)&221.3(25)&$5^{-}$&$(1-5)^{-}$&3&$1.33(40) \times 10^{-9}$&$6.60 \times 10^{-9}$\\
8665.3(8)&334.7(26)&$1^{-}$&$(0,1,2,3)^{-}$&1&$1.56(48) \times 10^{-4}$&$1.10 \times 10^{-3}$\\
8748.59(19)\footnotemark[3]&415.4(28)&$2^{+}$&$(0-4)^{+}$&$2$&$1.82(57) \times 10^{-5}$&$1.81 \times 10^{-1}$\\
%8748.59(19)\footnotemark[3]&8743.6(28)&415.4(28)&$2^{+}$&\hspace{-3mm}\ldelim \{ {3}{3mm}$(1,2,3)^{-}$&$1 + 3$&$9.3(38) \times 10^{-5} \phantom{a}\phantom{a}(l=1)$&$1.81 \times 10^{-1}$\\
% & & & & & &$7.1(28) \times 10^{-7} \phantom{a}\phantom{a}(l=3)$& \\
% & & & &$(0-4)^{+}$&$2$&$2.27(77) \times 10^{-5}$& \\
8764.18(6)&439.1(27)&$3^{-}$&$(0,1,2,3)^{-}$&1&$7.1(22) \times 10^{-4}$&$1.20 \times 10^{-1}$\\
8850.6(9)&521.2(28)&$6^{-},7^{-},8^{-}$&$(1,2,3)^{-}$&$1 + 3$&$1.10(34) \times 10^{-2} \phantom{a}(1)$&$1.86 \times 10^{-4}$\\ %\footnotemark[5]
 & & & & &$2.13(65) \times 10^{-5} \phantom{a}(3)$& \\
\hspace{-3mm}\ldelim \{ {2}{3mm}8934.81(7)&608.1(27)&$2^{+}$&$(0,1,2,3)^{-}$&1&$2.31(77) \times 10^{-1}$&$2.07 \times 10^{1}$\\
8938.4(9)& &$0^{+}$& & & &$1.28 \times 10^{-1}$\\
8994.5(11)&664.0(25)&$(1^{-},2^{+})$&\hspace{-3mm}\ldelim \{ {3}{3mm}$(0,1,2,3)^{-}$&1&$2.30(77) \times 10^{-1}$& \\
& & &$(1,2)^{+}$&$0 + 2$&$4.7(16) \times 10^{-1} \phantom{a}\phantom{a}(0)$& \\
& & & & &$1.20(40) \times 10^{-2} \phantom{a}(2)$& \\
9091.7(6)&764.0(36)&$3^{-}$&$(0,1,2,3)^{-}$&1&$8.2(28) \times 10^{-1}$& \\
9135.66(5)\footnotemark[4]&809.6(34)&$(2,3)^{-}$&$(0,1,2,3)^{-}$&1&$2.33(78) \times 10^{1}$& \\
\hspace{-3mm}\ldelim \{ {2}{3mm}9226.69(5)&899.6(34)&$(1^{-},2,3^{-})$&\hspace{-3mm}\ldelim \{ {3}{3mm}$(0,1,2,3)^{-}$&1&$3.7(12) \times 10^{1}$& \\
9227.43(7)& &$(1,2^{+})$&$(1,2)^{+}$&$0 + 2$&$5.1(18) \times 10^{1} \phantom{a}\phantom{a}(0)$& \\
 & & & & &$2.27(77) \times 10^{0} \phantom{a}(2)$& \\
%%%\hspace{-3mm}\ldelim \{ {2}{3mm}9404.85(19)&9403.7(31)&1075.5(31)&$2^{-}$& & & & \\
%%%9406.3(6)& & &$0^{+}$& & & & \\
%%%\hspace{-3mm}\ldelim \{ {2}{3mm}9412.3(2)&9416.7(32)&1088.5(32)& & & & & \\
%%%9418.8(2)& & &$3^{-}$& & & & \\
%%%\hspace{-3mm}\ldelim \{ {2}{3mm}9429.11(5)&9431.7(32)&1103.5(32)&$(3,4)^{-}$& & & & \\
%%%9432.46(18)& & &$1^{-}$& & & & \\
%%%9453.95(5)\footnotemark[3]& & &$3^{-}$& & & & \\
%%%9537.8(5)&9538.4(34)&1210.2(34)&$1^{-}$& & & & \\
%%%\hspace{-3mm}\ldelim \{ {2}{3mm}9603.0(4)&9605.1(40)&1276.9(40)&$3^{-}$& & & & \\
%%%9604.6(4)& & &$1^{-}$& & & & \\
% NOTE: Not including the 9662+9669 keV doublet's energy, since it was only extracted for 2 angles
%\hspace{-3mm}\ldelim \{ {2}{3mm}9662.2(2)&9673.4(31)&1345.2(31)&$3^{-}$\footnotemark[3]& & & \\
%9668.71(8)& & &$3^{-}$\footnotemark[3]& & &$1.73 \times 10^{4}$ \\
\hline\hline
\end{tabular}
}
\footnotetext[1]{From ENSDF \cite{Chen2017}, unless otherwise indicated.}
\footnotetext[2]{From the present experiment, unless otherwise indicated.}
\footnotetext[3]{From Ref. \cite{Gribble2022}.}
\footnotetext[4]{Level used as a calibration point for at least one angle in this work. See text.}
%\footnotetext[4]{Differs from $(\gamma, \gamma')$ and/or $(p, \gamma)$ ENSDF \cite{Chen2017} literature. No $(^{3}\mathrm{He}, d)$ or $(d, n)$ assignment for this level.}
%\footnotetext[5]{Proton partial width upper limits calculated by Longland \emph{et al.} \cite{Longland2018}.}
%\footnotetext[5]{Longland \emph{et al.} \cite{Longland2018} uses the ENSDF \cite{Chen2017} assignment of $J^{\pi} = 6^{-},7^{-},8^{-}$ in this upper limit calculation.}
\end{minipage}
\end{table}
\endgroup

The proton partial widths in terms of $(2J+1)\Gamma_{p}$ and the center-of-mass resonance energies $E^{\mathrm{c.m.}}_{r}$ are summarized in Table \ref{tab:partial_widths}. The ENSDF \cite{Chen2017} excitation energies are given for reference, unless otherwise indicated. Doublets are indicated with brackets over the individual states. Brackets are also given to indicate ambiguous $\ell$ assignments. In mixed-$\ell$ cases, each component is given in parentheses next to the $(2J+1)\Gamma_{p}$ value. The full range of $J^{\pi}$ possibilities determined from the $l$-assignments of the present work is also provided. The $(2J+1)\Gamma_{p}$ upper limit calculations from Ref. \cite{Longland2018} are given in the last column.

For the remainder of this chapter, unbound states will be referred to solely as resonances, and the corresponding center-of-mass resonance energy will be used to identify them.

%%%%%%%%%%%%%%%%%%%%%%%%%%%%%%%%%%%%%%%%%%%%%%%%%%%%%%%%%%%%%%%%%%%%%%%%%%%%%%%%%%%%%%%%%%%%%
\section{New $^{39}\mathrm{\textbf{K}}(p,\gamma)^{40}\mathrm{\textbf{Ca}}$ Reaction Rate} \label{sec:pg_reac_rate}

\subsection{The Reaction Rate}

The Monte Carlo reaction rate code, \texttt{RatesMC} \cite{Longland2010a,RatesMC}, was used to calculate a new $^{39}\mathrm{K}(p, \gamma)^{40}\mathrm{Ca}$ reaction rate probability density from the resonance energies and proton partial-widths reported in the previous section. Among the resonances extracted in the $^{39}\mathrm{K}(^{3}\mathrm{He},d)^{40}\mathrm{Ca}$ experiment, only those that do not already have a directly measured $^{39}\mathrm{K}(p, \gamma)^{40}\mathrm{Ca}$ resonance strength $\omega \gamma$ from Refs. \cite{Kikstra1990,Cheng1981,Leenhouts1966} were modified from the most recent reaction rate evaluation of Ref. \cite{Longland2018}. That is, the only resonance strengths and resonance energies that were updated from those in Ref. \cite{Longland2018} are below the 608 keV (606 keV ENSDF energy) resonance. Otherwise, the $\omega \gamma$ expectation values and variances of the $^{39}\mathrm{K}(p, \gamma)^{40}\mathrm{Ca}$ experiments \cite{Kikstra1990,Cheng1981,Leenhouts1966} are adopted from Ref. \cite{Longland2018}.

\begin{figure}[!p]
\includegraphics[width=6.5in]{Chapter-6/figs/rateCompare.png} % 6.5 in is exact page width (8.5 in) minus the 1 in margins. Height scaled automatically to keep aspect ratio.
\caption{\label{fig:rateCompare}Comparison between the $^{39}\mathrm{K}(p, \gamma)^{40}\mathrm{Ca}$ reaction rate using the proton partial-widths and resonance energies of the present experiment (solid line, blue band) and the most recent evaluation of Ref. \cite{Longland2018} (dotted line, gray band). The reaction rate ratio is taken with respect to the median, recommended rate of Ref. \cite{Longland2018} for both calculations. The $1\sigma$ uncertainty bands are shown.}
\end{figure}

\begin{figure}[!p]
\includegraphics[width=6.5in]{Chapter-6/figs/uncCompare.png} % 6.5 in is exact page width (8.5 in) minus the 1 in margins. Height scaled automatically to keep aspect ratio.
\caption{\label{fig:uncCompare}Comparison between the $^{39}\mathrm{K}(p, \gamma)^{40}\mathrm{Ca}$ reaction rate uncertainty using the proton partial-widths and resonance energies of the present experiment (solid band) and the most recent evaluation of Ref. \cite{Longland2018} (dotted band). Each reaction rate ratio is taken with respect to their own median, recommended rate. The $1\sigma$ uncertainty bands are shown.}
\end{figure}

The new reaction rate as a function of temperature is compared with that of Ref. \cite{Longland2018} in Fig. \ref{fig:rateCompare}. The solid line and blue band represent the median, recommended rate and the $1\sigma$ confidence interval of this work, respectively. The dotted line and gray band represent that of Ref. \cite{Longland2018}, except with resonance energies calculated using $S_{p} = 8328.18(2)$ keV from Ref. \cite{Wang2021} for consistency, but with marginal effect. Both reaction rates are normalized to the median, recommended rate of Ref. \cite{Longland2018}. As mentioned in Section \ref{subsec:longland_rate}, the large uncertainty in the reaction rate of Ref. \cite{Longland2018}, between about 50 MK and 200 MK, overlaps with most of the relevant temperatures that reproduce the Mg--K anticorrelation in the globular cluster NGC 2419. Ref. \cite{Iliadis2016} found this temperature range to be $T \approx 80-260$ MK. Therefore, the new median, recommended rate is significantly larger at the lower end of this temperature range compared to the previous rate, peaking at a factor of thirteen at about 70 MK. 

Fig. \ref{fig:uncCompare} shows a direct comparison of the uncertainty between the new rate and that of Ref. \cite{Longland2018}. This plot is almost identical to that of Fig. \ref{fig:rateCompare}, except now each reaction rate is normalized to its own median, recommended rate. The total width of the $1\sigma$ uncertainty band is significantly reduced at the lower end of the astrophysically-important region, from a factor of 84 at 80 MK to just a factor of 2. This is a reduction of a factor of 42.

\subsection{Resonance Contributions} \label{subsec:resonance_contributions}

Resolving the 154 keV resonance is mostly responsible for the increase in the reaction rate and decrease in its uncertainty between about 55 MK and 110 MK. Likewise, the new $\ell=1+3$ assignment of the 96 keV resonance, which has replaced the $\ell=3$ assignment of the $(^{3}\mathrm{He},d)$ measurement of Ref. \cite{Cage1971} in this calculation, is mostly responsible for the increase in the rate and decrease in the uncertainty between about 20 MK and 55 MK. For this resonance, the $(d,n)$ measurement of Ref. \cite{Fuchs1969} is in agreement with the $\ell=1+3$ assignment. It is also important to note that using the previous $\ell=3$ assignment has a negligible impact on the significant results mentioned in the previous section at 70 MK and 80 MK. Therefore, the resolution of the 154 keV has made by far the largest impact of the reaction rate. 

At less than 20 MK, the rate increase is a result of the replacement of the 29 keV proton partial width upper limit in Ref. \cite{Longland2018}. The smaller rate increase and uncertainty reduction at temperatures higher than 110 MK are caused by a combination of the 335 keV, 415 keV, 439 keV, and 521 keV resonances. The latter three of these resonances have replaced proton partial width upper limits in Ref. \cite{Longland2018}. The 221 keV resonance has not made a significant impact on the new rate because the previous $C^{2}S$ value from Ref. \cite{Cage1971} is almost identical to the one obtained here. It was also found that the resonance energies calculated in this work have a negligible effect on the reaction rate compared to the proton partial-widths.

\begin{figure}[!p]
\includegraphics[width=6.5in]{Chapter-6/figs/Contrib_Fox.png} % 6.5 in is exact page width (8.5 in) minus the 1 in margins. Height scaled automatically to keep aspect ratio.
\caption{\label{fig:contrib}Individual resonance contributions to the $^{39}\mathrm{K}(p,\gamma)^{40}\mathrm{Ca}$ reaction rate, where a value of 1.0 implies that the given resonance contributes 100$\%$ to the reaction rate at that temperature. The labels correspond to center-of-mass resonance energies in keV. Resonances with shading or hatched lines have been measured and are shown with their $1\sigma$ uncertainty bands. Resonances with a single, dashed line are upper limit calculations and show their 84$\%$ $1\sigma$ value, for clarity. The resonances displayed are those that individually account for at least 10$\%$ of the total reaction rate at their corresponding temperatures. The remaining summed resonance contributions are represented by the dotted line.}
\end{figure}

Fig. \ref{fig:contrib} illustrates the individual contributions from each resonance to the new $^{39}\mathrm{K}(p,\gamma)^{40}\mathrm{Ca}$ reaction rate. A value of unity means that the given resonance contributes 100$\%$ to the reaction rate at that temperature. The labels at the top of the figure represent the center-of-mass resonance energies in keV and are located directly above each resonance. Resonances with a shaded, finite width are measured and show their $1\sigma$ uncertainty bands. Some measured resonances are represented with hatched lines instead for visible clarity. Upper limit resonances are shown with a single, dashed line at their 84th percentile value. The black, dotted line consists of all resonances that individually contribute less than 10$\%$ to the total reaction rate at their corresponding temperatures.

Compared with the previous contribution plot of Fig. \ref{fig:contrib_longland2018}, some key differences can be identified. Most importantly, the 154 keV resonance has been resolved, and its contribution dominates between about 55 MK and 110 MK. It has led to several additional constraints on the contribution of nearby resonances. For example, the 96 keV resonance contribution uncertainty is significantly constrained. The 221 keV resonance is also much smaller than it was previously. Finally, the 212 keV upper limit resonance has also been significantly reduced. The 335 keV resonance still dominates in the region between about 110 MK and 400 MK, and its lower 16th percentile value has increased slightly compared to the previous rate from the resolution of the 415 keV, 439 keV, and 521 keV resonances. The 415 keV resonance is not shown, as it contributes less than 10$\%$ to the reaction rate.

%%%%%%%%%%%%%%%%%%%%%%%%%%%%%%%%%%%%%%%%%%%%%%%%%%%%%%%%%%%%%%%%%%%%%%%%%%%%%%%%%%%%%%%%%%%%%
\section{New Constraints for the Mg--K Anticorrelation}

A nuclear reaction network was used to investigate the effect the new $^{39}\mathrm{K}(p,\gamma)^{40}\mathrm{Ca}$ reaction rate has on potassium abundance in hydrogen-burning environments. New conditions are obtained for the hydrogen-burning temperatures $T$ and densities $\rho$ which reproduce the Mg-K anticorrelation observed in the globular cluster NGC 2419. $T$-$\rho$ conditions that no longer reproduce the anticorrelation with the new rate are also obtained. 

The strategy to reproduce the Mg-K anticorrelation is similar to that of Ref. \cite{Iliadis2016}. A single network calculation is performed for a given $T$-$\rho$ condition, and the final $X(\mathrm{K})$ and $X(\mathrm{Mg})$ mass fractions are obtained. The mass fractions are subsequently mixed with varying amounts of pristine matter. This mixing is described by Eqn. \ref{eqn:mixing}, where one part of processed matter (final mass fractions) is diluted with $f$ parts of pristine matter (initial mass fractions). The values of $f$ are varied to produce a range of [K/Fe] and [Mg/Fe] abundances. If this range of abundances follows the observed anticorrelation of Fig. \ref{fig:MgK}, then the $T$-$\rho$ condition is considered a solution.

The hydrogen-burning reaction network consists of 213 nuclei from $^{1}$H to $^{55}$Cr and 2127 total reactions. Initial mass fractions are chosen such that $X_{\mathrm{init}}(^{1}\mathrm{H}) = 0.754$, $X_{\mathrm{init}}(^{4}\mathrm{He}) = 0.245$, and the remaining 0.001 is distributed among the other nuclei according to their relative solar system abundances. A grid of network calculations is performed, each with constant temperature $T$ and density $\rho$, terminating at a specified final hydrogen mass fraction $X_{\mathrm{last}}(^{1}\mathrm{H})$. The grid spans $T=5-300$ MK (in steps of 5 MK), $\rho=10^{-4}-10^{11}$ $\mathrm{g}/\mathrm{cm}^{3}$ (in logarithmic steps, 10 per order of magnitude), and $X_{\mathrm{last}}(^{1}\mathrm{H}) = 0.2-0.7$ (in steps of 0.1). For each network calculation, the final $X(\mathrm{K})$ and $X(\mathrm{Mg})$ are mixed by $f = 1000, 100, 30, 10, 3, 1, 0.1, 0.05, 0.02$, and $0$, and the new diluted [K/Fe] and [Mg/Fe] abundances are obtained. Their path through the [Mg/Fe] vs. [K/Fe] plot of Fig. \ref{fig:MgK} is tracked to test for a match with the observed abundances. Specifically, a solution is obtained if the path enters the box defined by the points 1.3 dex < [K/Fe] < 2.0 dex and -1.5 dex < [Mg/Fe] < -0.8 dex.

\begin{figure}[!p]
\includegraphics[width=6.5in]{Chapter-6/figs/MgK_xlast20.png}
\caption{\label{fig:MgK_Trho_Conditions}The $T$-$\rho$ conditions that match the observed Mg-K anticorrelation in the globular cluster NGC 2419, using both the new and previous $^{39}\mathrm{K}(p,\gamma)^{40}\mathrm{Ca}$ reaction rates. Solutions for both rates are shown in gray as the overlap region, while solutions unique to one rate are indicated with red (new) or blue (old). An $X_{\mathrm{last}}(^{1}\mathrm{H})$ value of 0.2 is used here.}
\end{figure}

Fig. \ref{fig:MgK_Trho_Conditions} shows the results of the grid search with a specific $X_{\mathrm{last}}(^{1}\mathrm{H})$ value of 0.2. Only $T$-$\rho$ solutions to the Mg-K anticorrelation are shown in the plot. The gray points are solutions for both the previous \cite{Longland2018} and the new $^{39}\mathrm{K}(p,\gamma)^{40}\mathrm{Ca}$ reaction rate. Red points are solutions with the new rate only, and blue points are no longer solutions with the new rate. The magenta point represents the maximum temperature, $T=104$ MK, of hot-bottom burning (HBB) in a 6 $M_{\odot}$ AGB star with metallicity $Z = 10^{-4}$ and a typical density of 10 $\mathrm{g}/\mathrm{cm}^{3}$ \cite{Karakas2010}. Similarly, the green point represents the maximum temperature, $T=136$ MK, of HBB in an 8 $M_{\odot}$ SAGB star with metallicity $Z = 10^{-4}$ and $\rho=9$ $\mathrm{g}/\mathrm{cm}^{3}$ \cite{Doherty2015}. 

Neither of these hydrogen-burning polluter star candidates reach a sufficient temperature of $\gtrsim$ 150 MK to be considered responsible for the Mg--K anticorrelation in NGC 2419. Because the new reaction rate does not introduce any new solutions on the left-hand side of the primary $T$-$\rho$ band, the likelihood for AGB or SAGB stars to be polluter star candidates for this globular cluster has not improved from the previous analysis of Ref. \cite{Iliadis2016}. The situation depicted by Fig. \ref{fig:Trho_Iliadis} remains the same. ONe and CO classical novae are the likely polluter candidates, as their $T$-$\rho$ evolution passes straight though the band. However, as mentioned in Section \ref{sec:NGC2419}, AGB stellar model parameters are presently very uncertain, and adjusting these parameters could increase the HBB temperature of SAGB stars enough for them to be considered likely polluter star candidates \cite{Iliadis2016}.

Fig. \ref{fig:MgK_Trho_Conditions} still has several interesting features. There are two separate regions that describe the Mg-K anticorrelation in NGC 2419. The smaller band runs from 190 MK to more than 300 MK and from $10^{8}$ $\mathrm{g}/\mathrm{cm}^{3}$ to less than $10^{4}$ $\mathrm{g}/\mathrm{cm}^{3}$. Meanwhile, the larger band spans 95 MK to at least 220 MK, where it diverges into two branches. This band spans most densities in the grid, from $10^{-4}$ $\mathrm{g}/\mathrm{cm}^{3}$ up to about $10^{9}$ $\mathrm{g}/\mathrm{cm}^{3}$. The distribution of red and blue points is also interesting. They are mostly associated with the outer edges of the bands, with no clear pattern for temperature or density. The sea of red points at 95 MK and 100 MK likely result from the 154 keV resonance acting to increase the rate at these temperatures. The reaction rate of Ref. \cite{Longland2018} likely produces too much potassium here, while the new increased destruction rate reduces its abundance enough the match the observations.

\begin{figure}[!p]
\includegraphics[width=5.5in]{Chapter-6/figs/MaxKSeparation_T100_rho4d7_xlast20_HDA.png}
\caption{\label{fig:MaxSeparation}The maximum change in potassium abundance when using the new $^{39}\mathrm{K}(p,\gamma)^{40}\mathrm{Ca}$ reaction rate compared to that of Ref. \cite{Longland2018}. All other reaction rates are provided by \texttt{Starlib} \cite{Sallaska2013}. A maximum decrease in [K/Fe] of 0.23 dex (a factor of 1.7 decrease) occurs at $T = 100$ MK, $\rho = 4 \times 10^{7}$ $\mathrm{g}/\mathrm{cm}^{3}$, and $X_{\mathrm{last}}(^{1}\mathrm{H}) = 0.2$. This configuration now reproduces the Mg--K anticorrelation in NGC 2419, whereas the rate of Ref. \cite{Longland2018} does not.}
\end{figure}

This hypothesis is confirmed when analyzing Fig. \ref{fig:MaxSeparation}. This plot shows the diluted [K/Fe] and [Mg/Fe] final abundances represented by crosses. Lines are drawn between each cross for clarity. The black crosses were obtained using the new $^{39}\mathrm{K}(p,\gamma)^{40}\mathrm{Ca}$ reaction rate, while the gray crosses are from the previous rate of Ref. \cite{Longland2018}, tabulated in \texttt{Starlib} \cite{Sallaska2013}. The dashed box represents the conditions for acceptable values of $T$-$\rho$, as the Mg-K anticorrelation is reproduced when the path enters roughly this box. This particular plot shows the specific scenario where the difference in the [K/Fe] abundances between the previous and new rate is maximized at the $f=0.02$ value, represented by the bottom-most crosses shown in the figure. The $f=0$ case is off-scale. This condition was found by ranking the [K/Fe] differences over the entire grid, including the $X_{\mathrm{last}}(^{1}\mathrm{H})$ range. The maximum reduction of -0.23 dex (a factor of 1.7 decrease in abundance) occurs at the solution $T = 100$ MK, $\rho=4 \times 10^{7}$ $\mathrm{g}/\mathrm{cm}^{3}$, and $X_{\mathrm{last}}(^{1}\mathrm{H}) = 0.2$, which is exactly where the sea of red points was located in Fig. \ref{fig:MgK_Trho_Conditions}. Hence this maximum reduction in [K/Fe] is a direct result of the newly resolved 154 keV resonance.

%%%%%%%%%%%%%%%%%%%%%%%%%%%%%%%%%%%%%%%%%%%%%%%%%%%%%%%%%%%%%%%%%%%%%%%%%%%%%%%%%%%%%%%%%%%%%
\section{Conclusions}

New spectroscopic factors and proton partial widths have been extracted from the present $^{39}\mathrm{K}(^{3}\mathrm{He},d)^{40}\mathrm{Ca}$ proton-transfer experiment. Among the states observed, 13 have been newly resolved for $^{39}\mathrm{K}(^{3}\mathrm{He},d)^{40}\mathrm{Ca}$, 6 have been newly resolved for $^{39}\mathrm{K}$ proton-transfer (including $^{39}\mathrm{K}(d,n)^{40}\mathrm{Ca}$), and 4 have been newly resolved for $^{39}\mathrm{K}+p$ (including $^{39}\mathrm{K}(p, \gamma)^{40}\mathrm{Ca}$). These latter 4 are the 154 keV, 415 keV, 439 keV, and 521 keV resonances, in terms of center-of-mass energy. Among these, the 154 keV resonance has by far the largest impact on the $^{39}\mathrm{K}(p, \gamma)^{40}\mathrm{Ca}$ reaction rate, increasing it by a factor of 13 at about 70 MK from the most recent evaluation of Ref. \cite{Longland2018}. The reduction in the total $1\sigma$ width of the reaction rate uncertainty from this resonance is an even more significant result. At 80 MK, this width is reduced from a factor of 84 to a factor of 2. Another major impact on the reaction rate comes from the present $\ell=1+3$ assignment of the 96 keV resonance, in agreement with $(d,n)$. This replaced the $\ell=3$ assignment by Ref. \cite{Cage1971} in the previous reaction rate calculation. The rate is increased by as much as a factor of 6 at about 25 MK from this $\ell=1+3$ assignment. However, its effect is negligible above 55 MK compared to that of the 154 keV resonance.

The significant increase in the $^{39}\mathrm{K}(p, \gamma)^{40}\mathrm{Ca}$ reaction rate below 110 MK has led to a decrease in [K/Fe] abundance from hydrogen-burning nuclear reaction networks by as much as 0.23 dex, or a factor of 1.7 at $T=100$ MK, $\rho = 4 \times 10^{7}$ $\mathrm{g}/\mathrm{cm}^{3}$ and $X_{\mathrm{last}}(^{1}\mathrm{H}) = 0.2$. Several more $T$-$\rho$ conditions have been found to reproduce the Mg--K anticorrelation in NGC 2419 with the new reaction rate. Likewise, several $T$-$\rho$ conditions have been found to no longer reproduce the Mg--K anticorrelation. However, AGB and SAGB stars have not been found to be any more likely polluter star candidates than with the previous reaction rate.

The 335 keV resonance still contributes overwhelmingly to the reaction rate between about 110 MK and 400 MK, and several unknown resonances with only upper limit proton partial widths are distributed around it. These unknown resonances almost certainty contribute to the uncertainty in its resonance contribution, and continued focus should be placed on resolving them to further constrain the $T$-$\rho$ conditions for hydrogen burning in NGC 2419.

%%%%%%%%%%%%%%%%%%%%%%%%%%%%%%%%%%%%%%%%%%%%%%%%%%%%%%%%%%%%%%%%%%%%%%%%%%%%%%%%%%%%%%%%%%%%%
